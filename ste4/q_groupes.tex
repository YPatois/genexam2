\element{grft}{
    \begin{question}{grft0}
        Quelle est la nature du groupe fonctionnel: hydroxyle~?
        \begin{reponses}
            \mauvaise{{\scriptsize\chemfig{[,.5]R-C([-2]=O)-OH}}}
            \mauvaise{{\scriptsize\chemfig{[,.5]R-C([-2]=O)-R}}}
            \mauvaise{{\scriptsize\chemfig{R-[,.5]NH_2}}}
            \bonne{{\scriptsize\chemfig{[,.5]R-OH}}}
        \end{reponses}
    \end{question}
}\element{grft}{
    \begin{question}{grftr0}
        Quel est le nom du groupe fonctionnel: {\scriptsize\chemfig{[,.5]R-OH}}~?
        \begin{reponses}
            \mauvaise{carbonyle}
            \mauvaise{amino}
            \mauvaise{carboxyle}
            \bonne{hydroxyle}
        \end{reponses}
    \end{question}
}\element{grft}{
    \begin{question}{grft1}
        Quelle est la nature du groupe fonctionnel: carbonyle~?
        \begin{reponses}
            \mauvaise{{\scriptsize\chemfig{[,.5]R-C([-2]=O)-OH}}}
            \mauvaise{{\scriptsize\chemfig{[,.5]R-OH}}}
            \mauvaise{{\scriptsize\chemfig{R-[,.5]NH_2}}}
            \bonne{{\scriptsize\chemfig{[,.5]R-C([-2]=O)-R}}}
        \end{reponses}
    \end{question}
}\element{grft}{
    \begin{question}{grftr1}
        Quel est le nom du groupe fonctionnel: {\scriptsize\chemfig{[,.5]R-C([-2]=O)-R}}~?
        \begin{reponses}
            \mauvaise{amino}
            \mauvaise{carboxyle}
            \mauvaise{hydroxyle}
            \bonne{carbonyle}
        \end{reponses}
    \end{question}
}\element{grft}{
    \begin{question}{grft2}
        Quelle est la nature du groupe fonctionnel: carboxyle~?
        \begin{reponses}
            \mauvaise{{\scriptsize\chemfig{[,.5]R-OH}}}
            \mauvaise{{\scriptsize\chemfig{R-[,.5]NH_2}}}
            \mauvaise{{\scriptsize\chemfig{[,.5]R-C([-2]=O)-R}}}
            \bonne{{\scriptsize\chemfig{[,.5]R-C([-2]=O)-OH}}}
        \end{reponses}
    \end{question}
}\element{grft}{
    \begin{question}{grftr2}
        Quel est le nom du groupe fonctionnel: {\scriptsize\chemfig{[,.5]R-C([-2]=O)-OH}}~?
        \begin{reponses}
            \mauvaise{carbonyle}
            \mauvaise{hydroxyle}
            \mauvaise{amino}
            \bonne{carboxyle}
        \end{reponses}
    \end{question}
}\element{grft}{
    \begin{question}{grft3}
        Quelle est la nature du groupe fonctionnel: amino~?
        \begin{reponses}
            \mauvaise{{\scriptsize\chemfig{[,.5]R-C([-2]=O)-OH}}}
            \mauvaise{{\scriptsize\chemfig{[,.5]R-C([-2]=O)-R}}}
            \mauvaise{{\scriptsize\chemfig{[,.5]R-OH}}}
            \bonne{{\scriptsize\chemfig{R-[,.5]NH_2}}}
        \end{reponses}
    \end{question}
}\element{grft}{
    \begin{question}{grftr3}
        Quel est le nom du groupe fonctionnel: {\scriptsize\chemfig{R-[,.5]NH_2}}~?
        \begin{reponses}
            \mauvaise{carboxyle}
            \mauvaise{hydroxyle}
            \mauvaise{carbonyle}
            \bonne{amino}
        \end{reponses}
    \end{question}
}