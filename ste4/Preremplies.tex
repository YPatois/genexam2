\documentclass[10pt,a4paper]{article}
\usepackage[a4paper, margin=2cm]{geometry}
\usepackage{needspace}
\usepackage[nobottomtitles]{titlesec}
\usepackage{multicol}
\usepackage{xcolor}
\usepackage{fp}
\usepackage{xfp}
\usepackage{numprint}
\usepackage{siunitx}
\usepackage{enumitem}
\usepackage[version=4]{mhchem}
\usepackage{chemfig}
\usepackage{mol2chemfig}
%\usepackage{WriteOnGrid}
\usepackage{frcursive}
\usepackage{csvsimple}%
\usepackage[francais,bloc]{automultiplechoice}

\def\multiSymbole{}

\DeclareSIUnit{\nothing}{\relax}

\FPseed=12

\baremeDefautS{e=0,v=0,b=1,m=-.25}
%\baremeDefautM{formula=(NBC>0 ? (NMC==0 ? 1:0):0)}

\graphicspath{ {./images/} }

\newenvironment{reponsesd}{
    \begin{multicols}{2}
    \begin{reponses} }{
    \end{reponses}
    \end{multicols}
}

\setlength{\columnseprule}{1pt}
\def\columnseprulecolor{\color{lightgray}}%

\let\oexplain\explain
\renewcommand{\explain}[1]{\oexplain{\textcolor{red}{#1}}}

\titleformat{\section}
  {\centering\hrule\vspace{2mm}}
  {\thesection}
  {1em}
  {}
  [\vspace{1mm}\hrule]

\titleformat{\subsection}
  {\em}
  {\thesubsection}
  {1em}
  {}

\newcommand{\sujet}{
\exemplaire{1}{%

\AMCsetFoot{\niveau{} \classe{} -- \prenom{}~\nom{} -- \thepage}

%%% debut de l'en-tête des copies :
\begin{center}
\vfill
\noindent{\large \bf Classe de \niveau{}°\classe{}}

\vspace*{3mm}

{\Large\bf Évaluation sur table STL - Physique-Chime 12/11/2025}

\vfill
\namefield{\noindent{}\fbox{\vspace*{3mm}
         \Huge\bf\prenom{}~\nom{}\normalsize{}%
         \vspace*{3mm}
      }}
\end{center}
\vfill

\begin{center}
\textbf{Durée : 30 minutes.}
\vspace*{5mm}

  Aucun document n'est autorisé.
  L'usage de la calculatrice est interdit.

  Les questions faisant apparaitre le symbole \multiSymbole{} peuvent
  présenter une ou plusieurs bonnes réponses. Les autres ont
  une unique bonne réponse.

  Des points négatifs seront affectés aux mauvaises réponses.
  \vspace*{5mm}

  \textbf{IMPORTANT: Utilisez un crayon de papier bien noir pour cocher les cases, et une gomme
  pour effacer délicatement en cas d'erreur. Ne raturez pas les cases.
  Si vous effacez le pourtour de la case, ne le redessinez pas!}

  \vspace*{5mm}

  Ne pas faire comme ceci (raturé, pas centré, trop pâle, ...):\\
  \includegraphics[width=4cm]{checkbox_bad.png}

  Mais comme cela (bien centré, bien foncé):\\
  \includegraphics[width=2cm]{checkbox_good.png}

  Si vous faites une erreur et que vous ne pouvez effacer, raturez la case bien ostensiblement.

\end{center}
\vspace{1ex}
\vfill
\pagebreak
%%% fin de l'en-tête

\restituegroupe{groupes}


\AMCassociation{\id}

	  } % End \exemplaire{1}{%
} % End \newcommand{\sujet}{


%%%%§§§§§§§§§§§§§§§§§§§§§§§§§§§§§§§§§
\newcommand{\allq}{}
\newcommand{\nball}[1]{#1}

\begin{document}
%%%Options
\AMCrandomseed{10}

\def\AMCformQuestion#1{{\sc Question #1 :}}

\setdefaultgroupmode{withoutreplacement}
%%% Fin Options

%%% elements
\element{falcn}{
    \begin{question}{falcn0}
        Quel est le nom de l'hydrocarbone \ce{CH4}~?
        \begin{reponses}
            \mauvaise{octane}
            \mauvaise{heptane}
            \mauvaise{hexane}
            \bonne{méthane}
        \end{reponses}
    \end{question}
}\element{falcn}{
    \begin{question}{falcnr0}
        Quelle est la formule de l'hydrocarbone méthane~?
        \begin{reponses}
            \mauvaise{\ce{C10H22}}
            \mauvaise{\ce{C7H16}}
            \mauvaise{\ce{C9H20}}
            \bonne{\ce{CH4}}
        \end{reponses}
    \end{question}
}\element{falcn}{
    \begin{question}{falcn1}
        Quel est le nom de l'hydrocarbone \ce{C2H6}~?
        \begin{reponses}
            \mauvaise{decane}
            \mauvaise{octane}
            \mauvaise{méthane}
            \bonne{éthane}
        \end{reponses}
    \end{question}
}\element{falcn}{
    \begin{question}{falcnr1}
        Quelle est la formule de l'hydrocarbone éthane~?
        \begin{reponses}
            \mauvaise{\ce{CH4}}
            \mauvaise{\ce{C4H10}}
            \mauvaise{\ce{C5H12}}
            \bonne{\ce{C2H6}}
        \end{reponses}
    \end{question}
}\element{falcn}{
    \begin{question}{falcn2}
        Quel est le nom de l'hydrocarbone \ce{C3H8}~?
        \begin{reponses}
            \mauvaise{méthane}
            \mauvaise{octane}
            \mauvaise{pentane}
            \bonne{propane}
        \end{reponses}
    \end{question}
}\element{falcn}{
    \begin{question}{falcnr2}
        Quelle est la formule de l'hydrocarbone propane~?
        \begin{reponses}
            \mauvaise{\ce{C8H18}}
            \mauvaise{\ce{C10H22}}
            \mauvaise{\ce{CH4}}
            \bonne{\ce{C3H8}}
        \end{reponses}
    \end{question}
}\element{falcn}{
    \begin{question}{falcn3}
        Quel est le nom de l'hydrocarbone \ce{C4H10}~?
        \begin{reponses}
            \mauvaise{nonane}
            \mauvaise{propane}
            \mauvaise{pentane}
            \bonne{butane}
        \end{reponses}
    \end{question}
}\element{falcn}{
    \begin{question}{falcnr3}
        Quelle est la formule de l'hydrocarbone butane~?
        \begin{reponses}
            \mauvaise{\ce{C9H20}}
            \mauvaise{\ce{C2H6}}
            \mauvaise{\ce{C8H18}}
            \bonne{\ce{C4H10}}
        \end{reponses}
    \end{question}
}\element{alcn}{
    \begin{question}{alcn4}
        Quel est le nom de l'hydrocarbone \ce{C5H12}~?
        \begin{reponses}
            \mauvaise{heptane}
            \mauvaise{decane}
            \mauvaise{propane}
            \bonne{pentane}
        \end{reponses}
    \end{question}
}\element{alcn}{
    \begin{question}{alcnr4}
        Quelle est la formule de l'hydrocarbone pentane~?
        \begin{reponses}
            \mauvaise{\ce{C6H14}}
            \mauvaise{\ce{C4H10}}
            \mauvaise{\ce{C9H20}}
            \bonne{\ce{C5H12}}
        \end{reponses}
    \end{question}
}\element{alcn}{
    \begin{question}{alcn5}
        Quel est le nom de l'hydrocarbone \ce{C6H14}~?
        \begin{reponses}
            \mauvaise{decane}
            \mauvaise{propane}
            \mauvaise{méthane}
            \bonne{hexane}
        \end{reponses}
    \end{question}
}\element{alcn}{
    \begin{question}{alcnr5}
        Quelle est la formule de l'hydrocarbone hexane~?
        \begin{reponses}
            \mauvaise{\ce{C9H20}}
            \mauvaise{\ce{C10H22}}
            \mauvaise{\ce{CH4}}
            \bonne{\ce{C6H14}}
        \end{reponses}
    \end{question}
}\element{alcn}{
    \begin{question}{alcn6}
        Quel est le nom de l'hydrocarbone \ce{C7H16}~?
        \begin{reponses}
            \mauvaise{octane}
            \mauvaise{hexane}
            \mauvaise{pentane}
            \bonne{heptane}
        \end{reponses}
    \end{question}
}\element{alcn}{
    \begin{question}{alcnr6}
        Quelle est la formule de l'hydrocarbone heptane~?
        \begin{reponses}
            \mauvaise{\ce{C9H20}}
            \mauvaise{\ce{C2H6}}
            \mauvaise{\ce{C6H14}}
            \bonne{\ce{C7H16}}
        \end{reponses}
    \end{question}
}\element{alcn}{
    \begin{question}{alcn7}
        Quel est le nom de l'hydrocarbone \ce{C8H18}~?
        \begin{reponses}
            \mauvaise{heptane}
            \mauvaise{butane}
            \mauvaise{pentane}
            \bonne{octane}
        \end{reponses}
    \end{question}
}\element{alcn}{
    \begin{question}{alcnr7}
        Quelle est la formule de l'hydrocarbone octane~?
        \begin{reponses}
            \mauvaise{\ce{C10H22}}
            \mauvaise{\ce{C6H14}}
            \mauvaise{\ce{C5H12}}
            \bonne{\ce{C8H18}}
        \end{reponses}
    \end{question}
}\element{alcn}{
    \begin{question}{alcn8}
        Quel est le nom de l'hydrocarbone \ce{C9H20}~?
        \begin{reponses}
            \mauvaise{propane}
            \mauvaise{octane}
            \mauvaise{decane}
            \bonne{nonane}
        \end{reponses}
    \end{question}
}\element{alcn}{
    \begin{question}{alcnr8}
        Quelle est la formule de l'hydrocarbone nonane~?
        \begin{reponses}
            \mauvaise{\ce{CH4}}
            \mauvaise{\ce{C2H6}}
            \mauvaise{\ce{C4H10}}
            \bonne{\ce{C9H20}}
        \end{reponses}
    \end{question}
}\element{alcn}{
    \begin{question}{alcn9}
        Quel est le nom de l'hydrocarbone \ce{C10H22}~?
        \begin{reponses}
            \mauvaise{éthane}
            \mauvaise{propane}
            \mauvaise{méthane}
            \bonne{decane}
        \end{reponses}
    \end{question}
}\element{alcn}{
    \begin{question}{alcnr9}
        Quelle est la formule de l'hydrocarbone decane~?
        \begin{reponses}
            \mauvaise{\ce{C6H14}}
            \mauvaise{\ce{C2H6}}
            \mauvaise{\ce{C8H18}}
            \bonne{\ce{C10H22}}
        \end{reponses}
    \end{question}
}
\element{alcl}{
    \begin{question}{alcl0}
        Quelle est la formule de: ethanol~?
        \begin{reponses}
            \mauvaise{{\small\ce{HO-CH2-CH2-CH2-CH2-CH3}}}
            \mauvaise{{\small\ce{CH3-CHOH-CH2-CH3}}}
            \mauvaise{{\small\ce{HO-CH2-CH2-CH2-CH2-CH2-CH3}}}
            \bonne{{\small\ce{CH3-CH2-OH}}}
        \end{reponses}
    \end{question}
}\element{alcl}{
    \begin{question}{alclr0}
        Quel est le nom de la molécule: {\small\ce{CH3-CH2-OH}}~?
        \begin{reponses}
            \mauvaise{pentan-2-ol}
            \mauvaise{pentan-3-ol}
            \mauvaise{hexan-1-ol}
            \bonne{ethanol}
        \end{reponses}
    \end{question}
}\element{alcl}{
    \begin{question}{alcl1}
        Quelle est la formule de: butan-1-ol~?
        \begin{reponses}
            \mauvaise{{\small\ce{CH3-CH2-CHOH-CH2-CH3}}}
            \mauvaise{{\small\ce{CH3-CHOH-CH2-CH2-CH3}}}
            \mauvaise{{\small\ce{CH3-CH2-CHOH-CH2-CH2-CH3}}}
            \bonne{{\small\ce{HO-CH2-CH2-CH2-CH3}}}
        \end{reponses}
    \end{question}
}\element{alcl}{
    \begin{question}{alclr1}
        Quel est le nom de la molécule: {\small\ce{HO-CH2-CH2-CH2-CH3}}~?
        \begin{reponses}
            \mauvaise{hexan-3-ol}
            \mauvaise{ethanol}
            \mauvaise{butan-2-ol}
            \bonne{butan-1-ol}
        \end{reponses}
    \end{question}
}\element{alcl}{
    \begin{question}{alcl2}
        Quelle est la formule de: butan-2-ol~?
        \begin{reponses}
            \mauvaise{{\small\ce{CH3-CHOH-CH2-CH2-CH2-CH3}}}
            \mauvaise{{\small\ce{CH3-CH2-CHOH-CH2-CH2-CH3}}}
            \mauvaise{{\small\ce{CH3-CH2-OH}}}
            \bonne{{\small\ce{CH3-CHOH-CH2-CH3}}}
        \end{reponses}
    \end{question}
}\element{alcl}{
    \begin{question}{alclr2}
        Quel est le nom de la molécule: {\small\ce{CH3-CHOH-CH2-CH3}}~?
        \begin{reponses}
            \mauvaise{pentan-2-ol}
            \mauvaise{pentan-1-ol}
            \mauvaise{hexan-1-ol}
            \bonne{butan-2-ol}
        \end{reponses}
    \end{question}
}\element{alcl}{
    \begin{question}{alcl3}
        Quelle est la formule de: pentan-1-ol~?
        \begin{reponses}
            \mauvaise{{\small\ce{CH3-CH2-OH}}}
            \mauvaise{{\small\ce{CH3-CHOH-CH2-CH2-CH2-CH3}}}
            \mauvaise{{\small\ce{CH3-CHOH-CH2-CH2-CH3}}}
            \bonne{{\small\ce{HO-CH2-CH2-CH2-CH2-CH3}}}
        \end{reponses}
    \end{question}
}\element{alcl}{
    \begin{question}{alclr3}
        Quel est le nom de la molécule: {\small\ce{HO-CH2-CH2-CH2-CH2-CH3}}~?
        \begin{reponses}
            \mauvaise{butan-1-ol}
            \mauvaise{hexan-1-ol}
            \mauvaise{hexan-3-ol}
            \bonne{pentan-1-ol}
        \end{reponses}
    \end{question}
}\element{alcl}{
    \begin{question}{alcl4}
        Quelle est la formule de: pentan-2-ol~?
        \begin{reponses}
            \mauvaise{{\small\ce{CH3-CHOH-CH2-CH2-CH2-CH3}}}
            \mauvaise{{\small\ce{CH3-CHOH-CH2-CH3}}}
            \mauvaise{{\small\ce{HO-CH2-CH2-CH2-CH2-CH2-CH3}}}
            \bonne{{\small\ce{CH3-CHOH-CH2-CH2-CH3}}}
        \end{reponses}
    \end{question}
}\element{alcl}{
    \begin{question}{alclr4}
        Quel est le nom de la molécule: {\small\ce{CH3-CHOH-CH2-CH2-CH3}}~?
        \begin{reponses}
            \mauvaise{hexan-1-ol}
            \mauvaise{hexan-2-ol}
            \mauvaise{butan-2-ol}
            \bonne{pentan-2-ol}
        \end{reponses}
    \end{question}
}\element{alcl}{
    \begin{question}{alcl5}
        Quelle est la formule de: pentan-3-ol~?
        \begin{reponses}
            \mauvaise{{\small\ce{CH3-CH2-CHOH-CH2-CH2-CH3}}}
            \mauvaise{{\small\ce{CH3-CHOH-CH2-CH2-CH3}}}
            \mauvaise{{\small\ce{HO-CH2-CH2-CH2-CH2-CH2-CH3}}}
            \bonne{{\small\ce{CH3-CH2-CHOH-CH2-CH3}}}
        \end{reponses}
    \end{question}
}\element{alcl}{
    \begin{question}{alclr5}
        Quel est le nom de la molécule: {\small\ce{CH3-CH2-CHOH-CH2-CH3}}~?
        \begin{reponses}
            \mauvaise{pentan-2-ol}
            \mauvaise{hexan-1-ol}
            \mauvaise{ethanol}
            \bonne{pentan-3-ol}
        \end{reponses}
    \end{question}
}\element{alcl}{
    \begin{question}{alcl6}
        Quelle est la formule de: hexan-1-ol~?
        \begin{reponses}
            \mauvaise{{\small\ce{CH3-CH2-CHOH-CH2-CH3}}}
            \mauvaise{{\small\ce{CH3-CH2-OH}}}
            \mauvaise{{\small\ce{CH3-CHOH-CH2-CH2-CH3}}}
            \bonne{{\small\ce{HO-CH2-CH2-CH2-CH2-CH2-CH3}}}
        \end{reponses}
    \end{question}
}\element{alcl}{
    \begin{question}{alclr6}
        Quel est le nom de la molécule: {\small\ce{HO-CH2-CH2-CH2-CH2-CH2-CH3}}~?
        \begin{reponses}
            \mauvaise{butan-1-ol}
            \mauvaise{pentan-2-ol}
            \mauvaise{ethanol}
            \bonne{hexan-1-ol}
        \end{reponses}
    \end{question}
}\element{alcl}{
    \begin{question}{alcl7}
        Quelle est la formule de: hexan-2-ol~?
        \begin{reponses}
            \mauvaise{{\small\ce{HO-CH2-CH2-CH2-CH2-CH3}}}
            \mauvaise{{\small\ce{CH3-CH2-CHOH-CH2-CH3}}}
            \mauvaise{{\small\ce{CH3-CHOH-CH2-CH2-CH3}}}
            \bonne{{\small\ce{CH3-CHOH-CH2-CH2-CH2-CH3}}}
        \end{reponses}
    \end{question}
}\element{alcl}{
    \begin{question}{alclr7}
        Quel est le nom de la molécule: {\small\ce{CH3-CHOH-CH2-CH2-CH2-CH3}}~?
        \begin{reponses}
            \mauvaise{pentan-3-ol}
            \mauvaise{ethanol}
            \mauvaise{butan-2-ol}
            \bonne{hexan-2-ol}
        \end{reponses}
    \end{question}
}\element{alcl}{
    \begin{question}{alcl8}
        Quelle est la formule de: hexan-3-ol~?
        \begin{reponses}
            \mauvaise{{\small\ce{HO-CH2-CH2-CH2-CH3}}}
            \mauvaise{{\small\ce{CH3-CHOH-CH2-CH3}}}
            \mauvaise{{\small\ce{CH3-CH2-OH}}}
            \bonne{{\small\ce{CH3-CH2-CHOH-CH2-CH2-CH3}}}
        \end{reponses}
    \end{question}
}\element{alcl}{
    \begin{question}{alclr8}
        Quel est le nom de la molécule: {\small\ce{CH3-CH2-CHOH-CH2-CH2-CH3}}~?
        \begin{reponses}
            \mauvaise{ethanol}
            \mauvaise{butan-2-ol}
            \mauvaise{pentan-2-ol}
            \bonne{hexan-3-ol}
        \end{reponses}
    \end{question}
}
\element{grft}{
    \begin{question}{grft0}
        Quelle est la nature du groupe fonctionnel: hydroxyle~?
        \begin{reponses}
            \mauvaise{{\scriptsize\chemfig{[,.5]R-C([-2]=O)-OH}}}
            \mauvaise{{\scriptsize\chemfig{[,.5]R-C([-2]=O)-R}}}
            \mauvaise{{\scriptsize\chemfig{R-[,.5]NH_2}}}
            \bonne{{\scriptsize\chemfig{[,.5]R-OH}}}
        \end{reponses}
    \end{question}
}\element{grft}{
    \begin{question}{grftr0}
        Quel est le nom du groupe fonctionnel: {\scriptsize\chemfig{[,.5]R-OH}}~?
        \begin{reponses}
            \mauvaise{carbonyle}
            \mauvaise{amino}
            \mauvaise{carboxyle}
            \bonne{hydroxyle}
        \end{reponses}
    \end{question}
}\element{grft}{
    \begin{question}{grft1}
        Quelle est la nature du groupe fonctionnel: carbonyle~?
        \begin{reponses}
            \mauvaise{{\scriptsize\chemfig{[,.5]R-C([-2]=O)-OH}}}
            \mauvaise{{\scriptsize\chemfig{[,.5]R-OH}}}
            \mauvaise{{\scriptsize\chemfig{R-[,.5]NH_2}}}
            \bonne{{\scriptsize\chemfig{[,.5]R-C([-2]=O)-R}}}
        \end{reponses}
    \end{question}
}\element{grft}{
    \begin{question}{grftr1}
        Quel est le nom du groupe fonctionnel: {\scriptsize\chemfig{[,.5]R-C([-2]=O)-R}}~?
        \begin{reponses}
            \mauvaise{amino}
            \mauvaise{carboxyle}
            \mauvaise{hydroxyle}
            \bonne{carbonyle}
        \end{reponses}
    \end{question}
}\element{grft}{
    \begin{question}{grft2}
        Quelle est la nature du groupe fonctionnel: carboxyle~?
        \begin{reponses}
            \mauvaise{{\scriptsize\chemfig{[,.5]R-OH}}}
            \mauvaise{{\scriptsize\chemfig{R-[,.5]NH_2}}}
            \mauvaise{{\scriptsize\chemfig{[,.5]R-C([-2]=O)-R}}}
            \bonne{{\scriptsize\chemfig{[,.5]R-C([-2]=O)-OH}}}
        \end{reponses}
    \end{question}
}\element{grft}{
    \begin{question}{grftr2}
        Quel est le nom du groupe fonctionnel: {\scriptsize\chemfig{[,.5]R-C([-2]=O)-OH}}~?
        \begin{reponses}
            \mauvaise{carbonyle}
            \mauvaise{hydroxyle}
            \mauvaise{amino}
            \bonne{carboxyle}
        \end{reponses}
    \end{question}
}\element{grft}{
    \begin{question}{grft3}
        Quelle est la nature du groupe fonctionnel: amino~?
        \begin{reponses}
            \mauvaise{{\scriptsize\chemfig{[,.5]R-C([-2]=O)-OH}}}
            \mauvaise{{\scriptsize\chemfig{[,.5]R-C([-2]=O)-R}}}
            \mauvaise{{\scriptsize\chemfig{[,.5]R-OH}}}
            \bonne{{\scriptsize\chemfig{R-[,.5]NH_2}}}
        \end{reponses}
    \end{question}
}\element{grft}{
    \begin{question}{grftr3}
        Quel est le nom du groupe fonctionnel: {\scriptsize\chemfig{R-[,.5]NH_2}}~?
        \begin{reponses}
            \mauvaise{carboxyle}
            \mauvaise{hydroxyle}
            \mauvaise{carbonyle}
            \bonne{amino}
        \end{reponses}
    \end{question}
}
\element{brut}{ % Ici pas de numérotation  
  \begin{question}{brut1}  
    Quel est la formule brute du Rouge para~?  
    {\small 
\chemfig{HO-[:30,,2]=^[:330](-[:30]-[:90](=_[:30.9,1.042]-[:330.4,1.042]=_[:270,1.042]-[:209.6,1.042]=_[:149.1,1.042])-[:150]=^[:210]-[:270])-[:270]N=[:210]N-[:270]=_[:330]-[:270]=_[:210](-[:270]\mcfbelow{N}{^{\mcfplus}}(=[:330]O)-[:210,,,2]^{\mcfminus}O)-[:150]=_[:\chemfig{HO-[:30,,2]=^[:330](-[:30]-[:90](=_[:30.9,1.042]-[:330.4,1.042]=_[:270,1.042]-[:209.6,1.042]=_[:149.1,1.042])-[:150]=^[:210]-[:270])-[:270]N=[:210]N-[:270]=_[:330]-[:270]=_[:210](-[:270]\mcfbelow{N}{^{\mcfplus}}(=[:330]O)-[:210,,,2]^{\mcfminus}O-[:150]=_[:90](-[:30])}}}
    \begin{reponses}  
      \mauvaise{\ce{C16H11O3N2}}  
      \mauvaise{\ce{C16H10O3N3}}  
      \mauvaise{\ce{C15H11O3N3}}  
      \bonne{\ce{C16H11O3N3}}  
    \end{reponses}  
    \explain{La formule brute correspond au nombre exact de chaque type d'atome dans la molécule. Ici, il y a 16 atomes de carbone, 11 d'hydrogène, 3 d'oxygène et 3 d'azote.}  
  \end{question}  

  \begin{question}{brut2}  
    Quel est la formule brute du Paracétamol~?  
    {\small \chemfig{OH-[:90,,1]C=^[:30,,,1]CH-[:90,,1,1]CH=^[:150,,1]C(-[:210,,,2]HC=^[:270,,2,2]HC-[:330,,2]\phantom{C})-[:90,,,2]HN-[:30,,2]C(=[:330]O)-[:90,,,1]CH_3}  }
    \begin{reponses}  
      \mauvaise{\ce{C8H8NO2}}  
      \mauvaise{\ce{C7H7NO2}}  
      \mauvaise{\ce{C7H9NO2}}  
      \bonne{\ce{C8H9NO2}}  
    \end{reponses}  
    \explain{La formule brute correspond au nombre exact de chaque type d'atome dans la molécule. Ici, il y a 8 atomes de carbone, 9 d'hydrogène, 1 d'azote et 2 d'oxygène.}  
  \end{question}  

  \begin{question}{brut3}  
    Quel est la formule brute de la Kétamine~?  
    {\small 
\chemfig{Cl-[:100]C-[:160]C(=_[:100,,,2]HC-[:40,,2]\mcfabove{C}{H}=_[:340,,,1]CH-[:280,,1,1]CH=_[:220,,1]\phantom{C})-[:220]C(-[:120]\mcfabove{N}{H}-[:180,,,2]H_3C)-[:330,,,1]CH_2-[:270,,1,1]CH_2-[:210,,1]\mcfbelow{C}{\mcfright{H}{_2}}-[:150,,,2]H_2C-[:90,,2]C(-[:30\chemfig{Cl-[:100]C-[:160]C(=_[:100,,,2]HC-[:40,,2]\mcfabove{C}{H}=_[:340,,,1]CH-[:280,,1,1]CH=_[:220,,1]\phantom{C})-[:220]C(-[:120]\mcfabove{N}{H}-[:180,,,2]H_3C)-[:330,,,1]CH_2-[:270,,1,1]CH_2-[:210,,1]\mcfbelow{C}{\mcfright{H}{_2}}-[:150,,,2]H_2C-[:9,,2]C(-[:30]\phantom{C})=[:150]O} }}
    \begin{reponses}  
      \mauvaise{\ce{C13H15ClNO}}  
      \mauvaise{\ce{C12H16ClNO}}  
      \mauvaise{\ce{C13H16ClN}}  
      \bonne{\ce{C13H16ClNO}}  
    \end{reponses}  
    \explain{La formule brute correspond au nombre exact de chaque type d'atome dans la molécule. Ici, il y a 13 atomes de carbone, 16 d'hydrogène, 1 de chlore, 1 d'azote et 1 d'oxygène.}  
  \end{question}  

  \begin{question}{brut4}  
    Quel est la formule brute de l'Ibuprofène~?  
    {\small \chemfig{OH-[:270,,1](=[:330]O)-[:210](-[:150])-[:270]=_[:330]-[:270]=_[:210](-[:150]=_[:90]-[:30])-[:270]-[:210](-[:270])-[:150]}   }
    \begin{reponses}  
      \mauvaise{\ce{C13H18O}}  
      \mauvaise{\ce{C14H18O2}}  
      \mauvaise{\ce{C12H18O2}}  
      \bonne{\ce{C13H18O2}}  
    \end{reponses}  
    \explain{La formule brute correspond au nombre exact de chaque type d'atome dans la molécule. Ici, il y a 13 atomes de carbone, 18 d'hydrogène et 2 d'oxygène.}  
  \end{question}  
}
\element{brut}{ % Ici pas de numérotation
  \begin{question}{brut5}
    Quel est la formule brute de l'acide citrique~?
    {\small \chemfig{OH-[:120,,1](-[:210]-[:270](=[:330]O)-[:210,,,2]HO)(-[:30](=[:90]O)-[:330,,,1]OH)-[:120]-[:180](-[:120,,,2]HO)=[:240]O} }
    \begin{reponses}
      \mauvaise{\ce{C6H8O7}}
      \mauvaise{\ce{C6H7O7}}
      \mauvaise{\ce{C6H6O7}}
      \bonne{\ce{C6H8O7}}
    \end{reponses}
    \explain{L'addition ajoute les deux quantités, ici 2 et 2.}
  \end{question}
}

\element{nomenca}{
    \begin{question}{nomenca0}
        \needspace{6cm}Quel est la formule topologique de: 4-éthylheptane~?
        \begin{reponses}
            \mauvaise{{\small\chemfig{-[:330](-[:270])-[:30]-[:330]-[:30]-[:330]}}}
            \mauvaise{{\small\chemfig{-[:210]-[:150]-[:210](-[:150]-[:210]-[:150])-[:270]-[:210]-[:270]}}}
            \mauvaise{{\small\chemfig{-[:90](-[:30]-[:330]-[:30])-[:150]-[:210]-[:150]}}}
            \bonne{{\small\chemfig{-[:30]-[:90](-[:30]-[:330]-[:30])-[:150]-[:210]-[:150]}}}
        \end{reponses}
    \end{question}
}\element{nomenca}{
    \begin{question}{nomencar0}
        \needspace{6cm}Quel est le nom de cette molécule: ~? 
         \smallskip
         \\ 
         {\small\chemfig{-[:30]-[:90](-[:30]-[:330]-[:30])-[:150]-[:210]-[:150]}}\begin{reponses}
            \mauvaise{2-méthylhexane}
            \mauvaise{2-éthylheptane}
            \mauvaise{3-méthyloctane}
            \bonne{4-éthylheptane}
        \end{reponses}
    \end{question}
}\element{nomenca}{
    \begin{question}{nomenca1}
        \needspace{6cm}Quel est la formule topologique de: 3-éthylheptane~?
        \begin{reponses}
            \mauvaise{{\small\chemfig{-[:330](-[:270])-[:30]-[:330]-[:30]-[:330]}}}
            \mauvaise{{\small\chemfig{-[:90](-[:150]-[:210])-[:30]-[:330]-[:30]-[:330]-[:30]}}}
            \mauvaise{{\small\chemfig{-[:210]-[:150]-[:210](-[:150]-[:210]-[:150])-[:270]-[:210]-[:270]}}}
            \bonne{{\small\chemfig{-[:30]-[:330](-[:270]-[:210])-[:30]-[:330]-[:30]-[:330]}}}
        \end{reponses}
    \end{question}
}\element{nomenca}{
    \begin{question}{nomencar1}
        \needspace{6cm}Quel est le nom de cette molécule: ~? 
         \smallskip
         \\ 
         {\small\chemfig{-[:30]-[:330](-[:270]-[:210])-[:30]-[:330]-[:30]-[:330]}}\begin{reponses}
            \mauvaise{4-méthylheptane}
            \mauvaise{4-propylheptane}
            \mauvaise{2-méthylhexane}
            \bonne{3-éthylheptane}
        \end{reponses}
    \end{question}
}\element{nomenca}{
    \begin{question}{nomenca4}
        \needspace{6cm}Quel est la formule topologique de: 4-méthylheptane~?
        \begin{reponses}
            \mauvaise{{\small\chemfig{-[:30]-[:90](-[:30]-[:330]-[:30])-[:150]-[:210]-[:150]}}}
            \mauvaise{{\small\chemfig{-[:330](-[:270])-[:30]-[:330]-[:30]-[:330]}}}
            \mauvaise{{\small\chemfig{-[:90](-[:150]-[:210])-[:30]-[:330]-[:30]-[:330]-[:30]}}}
            \bonne{{\small\chemfig{-[:90](-[:30]-[:330]-[:30])-[:150]-[:210]-[:150]}}}
        \end{reponses}
    \end{question}
}\element{nomenca}{
    \begin{question}{nomencar4}
        \needspace{6cm}Quel est le nom de cette molécule: ~? 
         \smallskip
         \\ 
         {\small\chemfig{-[:90](-[:30]-[:330]-[:30])-[:150]-[:210]-[:150]}}\begin{reponses}
            \mauvaise{4-propylheptane}
            \mauvaise{3-éthylheptane}
            \mauvaise{4-éthylheptane}
            \bonne{4-méthylheptane}
        \end{reponses}
    \end{question}
}\element{nomenca}{
    \begin{question}{nomenca5}
        \needspace{6cm}Quel est la formule topologique de: 4-propylheptane~?
        \begin{reponses}
            \mauvaise{{\small\chemfig{-[:90](-[:150]-[:210])-[:30]-[:330]-[:30]-[:330]-[:30]}}}
            \mauvaise{{\small\chemfig{-[:30]-[:90](-[:30]-[:330]-[:30])-[:150]-[:210]-[:150]}}}
            \mauvaise{{\small\chemfig{-[:330](-[:270])-[:30]-[:330]-[:30]-[:330]}}}
            \bonne{{\small\chemfig{-[:210]-[:150]-[:210](-[:150]-[:210]-[:150])-[:270]-[:210]-[:270]}}}
        \end{reponses}
    \end{question}
}\element{nomenca}{
    \begin{question}{nomencar5}
        \needspace{6cm}Quel est le nom de cette molécule: ~? 
         \smallskip
         \\ 
         {\small\chemfig{-[:210]-[:150]-[:210](-[:150]-[:210]-[:150])-[:270]-[:210]-[:270]}}\begin{reponses}
            \mauvaise{3-éthylheptane}
            \mauvaise{2-méthylhexane}
            \mauvaise{5-éthylheptane}
            \bonne{4-propylheptane}
        \end{reponses}
    \end{question}
}\element{nomenca}{
    \begin{question}{nomenca6}
        \needspace{6cm}Quel est la formule topologique de: 2-méthylhexane~?
        \begin{reponses}
            \mauvaise{{\small\chemfig{-[:90](-[:30]-[:330]-[:30])-[:150]-[:210]-[:150]}}}
            \mauvaise{{\small\chemfig{-[:210]-[:150]-[:210](-[:150]-[:210]-[:150])-[:270]-[:210]-[:270]}}}
            \mauvaise{{\small\chemfig{-[:30]-[:90](-[:30]-[:330]-[:30])-[:150]-[:210]-[:150]}}}
            \bonne{{\small\chemfig{-[:330](-[:270])-[:30]-[:330]-[:30]-[:330]}}}
        \end{reponses}
    \end{question}
}\element{nomenca}{
    \begin{question}{nomencar6}
        \needspace{6cm}Quel est le nom de cette molécule: ~? 
         \smallskip
         \\ 
         {\small\chemfig{-[:330](-[:270])-[:30]-[:330]-[:30]-[:330]}}\begin{reponses}
            \mauvaise{4-éthylheptane}
            \mauvaise{3-méthyloctane}
            \mauvaise{2-éthylheptane}
            \bonne{2-méthylhexane}
        \end{reponses}
    \end{question}
}\element{nomenca}{
    \begin{question}{nomenca7}
        \needspace{6cm}Quel est la formule topologique de: 3-méthyloctane~?
        \begin{reponses}
            \mauvaise{{\small\chemfig{-[:90](-[:30]-[:330]-[:30])-[:150]-[:210]-[:150]}}}
            \mauvaise{{\small\chemfig{-[:30]-[:330](-[:270]-[:210])-[:30]-[:330]-[:30]-[:330]}}}
            \mauvaise{{\small\chemfig{-[:30]-[:90](-[:30]-[:330]-[:30])-[:150]-[:210]-[:150]}}}
            \bonne{{\small\chemfig{-[:90](-[:150]-[:210])-[:30]-[:330]-[:30]-[:330]-[:30]}}}
        \end{reponses}
    \end{question}
}\element{nomenca}{
    \begin{question}{nomencar7}
        \needspace{6cm}Quel est le nom de cette molécule: ~? 
         \smallskip
         \\ 
         {\small\chemfig{-[:90](-[:150]-[:210])-[:30]-[:330]-[:30]-[:330]-[:30]}}\begin{reponses}
            \mauvaise{5-éthylheptane}
            \mauvaise{4-éthylheptane}
            \mauvaise{2-éthylheptane}
            \bonne{3-méthyloctane}
        \end{reponses}
    \end{question}
}\element{nomenca}{
    \begin{question}{nomenca8}
        \needspace{6cm}Quel est la formule topologique de: 2,3-diméthylpentane~?
        \begin{reponses}
            \mauvaise{{\small\chemfig{-[:300](-[:30]-[:330])(-[:300])-[:210]-[:150]}}}
            \mauvaise{{\small\chemfig{-[:330](-[:270])-[:30]-[:330]-[:30]}}}
            \mauvaise{{\small\chemfig{-[:330](-[:270])-[:30]-[:330](-[:30])-[:270]}}}
            \bonne{{\small\chemfig{-[:90](-[:30](-[:90])-[:330])-[:150]-[:210]}}}
        \end{reponses}
    \end{question}
}\element{nomenca}{
    \begin{question}{nomencar8}
        \needspace{6cm}Quel est le nom de cette molécule: ~? 
         \smallskip
         \\ 
         {\small\chemfig{-[:90](-[:30](-[:90])-[:330])-[:150]-[:210]}}\begin{reponses}
            \mauvaise{2-méthylpentane}
            \mauvaise{2,4-diméthylpentane}
            \mauvaise{3,3-diméthylpentane}
            \bonne{2,3-diméthylpentane}
        \end{reponses}
    \end{question}
}\element{nomenca}{
    \begin{question}{nomenca9}
        \needspace{6cm}Quel est la formule topologique de: 2,4-diméthylpentane~?
        \begin{reponses}
            \mauvaise{{\small\chemfig{-[:300](-[:30]-[:330])(-[:300])-[:210]-[:150]}}}
            \mauvaise{{\small\chemfig{-[:90](-[:30](-[:90])-[:330])-[:150]-[:210]}}}
            \mauvaise{{\small\chemfig{-[:330](-[:270])-[:30]-[:330]-[:30]}}}
            \bonne{{\small\chemfig{-[:330](-[:270])-[:30]-[:330](-[:30])-[:270]}}}
        \end{reponses}
    \end{question}
}\element{nomenca}{
    \begin{question}{nomencar9}
        \needspace{6cm}Quel est le nom de cette molécule: ~? 
         \smallskip
         \\ 
         {\small\chemfig{-[:330](-[:270])-[:30]-[:330](-[:30])-[:270]}}\begin{reponses}
            \mauvaise{2,3-diméthylpentane}
            \mauvaise{2-méthylpentane}
            \mauvaise{2,3-diéthylpentane}
            \bonne{2,4-diméthylpentane}
        \end{reponses}
    \end{question}
}\element{nomenca}{
    \begin{question}{nomenca10}
        \needspace{6cm}Quel est la formule topologique de: 3,3-diméthylpentane~?
        \begin{reponses}
            \mauvaise{{\small\chemfig{-[:330](-[:270])-[:30]-[:330](-[:30])-[:270]}}}
            \mauvaise{{\small\chemfig{-[:90](-[:30](-[:90])-[:330])-[:150]-[:210]}}}
            \mauvaise{{\small\chemfig{-[:330](-[:270])-[:30]-[:330]-[:30]}}}
            \bonne{{\small\chemfig{-[:300](-[:30]-[:330])(-[:300])-[:210]-[:150]}}}
        \end{reponses}
    \end{question}
}\element{nomenca}{
    \begin{question}{nomencar10}
        \needspace{6cm}Quel est le nom de cette molécule: ~? 
         \smallskip
         \\ 
         {\small\chemfig{-[:300](-[:30]-[:330])(-[:300])-[:210]-[:150]}}\begin{reponses}
            \mauvaise{2-méthylpentane}
            \mauvaise{2,4-diméthylpentane}
            \mauvaise{2,3-diéthylpentane}
            \bonne{3,3-diméthylpentane}
        \end{reponses}
    \end{question}
}\element{nomenca}{
    \begin{question}{nomenca11}
        \needspace{6cm}Quel est la formule topologique de: 2-méthylpentane~?
        \begin{reponses}
            \mauvaise{{\small\chemfig{-[:330](-[:270])-[:30]-[:330](-[:30])-[:270]}}}
            \mauvaise{{\small\chemfig{-[:300](-[:30]-[:330])(-[:300])-[:210]-[:150]}}}
            \mauvaise{{\small\chemfig{-[:90](-[:30](-[:90])-[:330])-[:150]-[:210]}}}
            \bonne{{\small\chemfig{-[:330](-[:270])-[:30]-[:330]-[:30]}}}
        \end{reponses}
    \end{question}
}\element{nomenca}{
    \begin{question}{nomencar11}
        \needspace{6cm}Quel est le nom de cette molécule: ~? 
         \smallskip
         \\ 
         {\small\chemfig{-[:330](-[:270])-[:30]-[:330]-[:30]}}\begin{reponses}
            \mauvaise{2,3-diméthylpentane}
            \mauvaise{2,3-diéthylpentane}
            \mauvaise{2,4-diméthylpentane}
            \bonne{2-méthylpentane}
        \end{reponses}
    \end{question}
}\element{nomenca}{
    \begin{question}{nomenca16}
        \needspace{6cm}Quel est la formule topologique de: butan-2-amine~?
        \begin{reponses}
            \mauvaise{{\small\chemfig{H_2N-[:330,,2](-[:270])-[:30]-[:330]-[:30]}}}
            \mauvaise{{\small\chemfig{H_2N-[:330,,2](-[:270])-[:30](-[:330])-[:90]}}}
            \mauvaise{{\small\chemfig{H_2N-[:30,,2]-[:330]-[:30]-[:330]}}}
            \bonne{{\small\chemfig{H_2N-[:330,,2](-[:270])-[:30]-[:330]}}}
        \end{reponses}
    \end{question}
}\element{nomenca}{
    \begin{question}{nomencar16}
        \needspace{6cm}Quel est le nom de cette molécule: ~? 
         \smallskip
         \\ 
         {\small\chemfig{H_2N-[:330,,2](-[:270])-[:30]-[:330]}}\begin{reponses}
            \mauvaise{2-amino-3-éthylbutane}
            \mauvaise{pentan-2-amine}
            \mauvaise{butan-1-amine}
            \bonne{butan-2-amine}
        \end{reponses}
    \end{question}
}\element{nomenca}{
    \begin{question}{nomenca17}
        \needspace{6cm}Quel est la formule topologique de: butan-1-amine~?
        \begin{reponses}
            \mauvaise{{\small\chemfig{H_2N-[:330,,2](-[:270])-[:30]-[:330]-[:30]}}}
            \mauvaise{{\small\chemfig{H_2N-[:330,,2](-[:270])-[:30]-[:330]}}}
            \mauvaise{{\small\chemfig{H_2N-[:330,,2](-[:270])-[:30](-[:330])-[:90]}}}
            \bonne{{\small\chemfig{H_2N-[:30,,2]-[:330]-[:30]-[:330]}}}
        \end{reponses}
    \end{question}
}\element{nomenca}{
    \begin{question}{nomencar17}
        \needspace{6cm}Quel est le nom de cette molécule: ~? 
         \smallskip
         \\ 
         {\small\chemfig{H_2N-[:30,,2]-[:330]-[:30]-[:330]}}\begin{reponses}
            \mauvaise{butan-2-amine}
            \mauvaise{2-amino-3-éthylbutane}
            \mauvaise{pentan-2-amine}
            \bonne{butan-1-amine}
        \end{reponses}
    \end{question}
}\element{nomenca}{
    \begin{question}{nomenca18}
        \needspace{6cm}Quel est la formule topologique de: 3-méthylbutan-2-amine~?
        \begin{reponses}
            \mauvaise{{\small\chemfig{H_2N-[:330,,2](-[:270])-[:30](-[:330,,,1]NH_2)-[:90]}}}
            \mauvaise{{\small\chemfig{H_2N-[:330,,2](-[:270])-[:30]-[:330]}}}
            \mauvaise{{\small\chemfig{H_2N-[:30,,2]-[:330]-[:30]-[:330]}}}
            \bonne{{\small\chemfig{H_2N-[:330,,2](-[:270])-[:30](-[:330])-[:90]}}}
        \end{reponses}
    \end{question}
}\element{nomenca}{
    \begin{question}{nomencar18}
        \needspace{6cm}Quel est le nom de cette molécule: ~? 
         \smallskip
         \\ 
         {\small\chemfig{H_2N-[:330,,2](-[:270])-[:30](-[:330])-[:90]}}\begin{reponses}
            \mauvaise{pentan-2-amine}
            \mauvaise{2-amino-3-éthylbutane}
            \mauvaise{butan-1-amine}
            \bonne{3-méthylbutan-2-amine}
        \end{reponses}
    \end{question}
}\element{nomenca}{
    \begin{question}{nomenca20}
        \needspace{6cm}Quel est la formule topologique de: pentan-2-amine~?
        \begin{reponses}
            \mauvaise{{\small\chemfig{H_2N-[:330,,2](-[:270])-[:30](-[:330])-[:90]}}}
            \mauvaise{{\small\chemfig{H_2N-[:330,,2](-[:270])-[:30](-[:330,,,1]NH_2)-[:90]}}}
            \mauvaise{{\small\chemfig{H_2N-[:330,,2](-[:270])-[:30]-[:330]}}}
            \bonne{{\small\chemfig{H_2N-[:330,,2](-[:270])-[:30]-[:330]-[:30]}}}
        \end{reponses}
    \end{question}
}\element{nomenca}{
    \begin{question}{nomencar20}
        \needspace{6cm}Quel est le nom de cette molécule: ~? 
         \smallskip
         \\ 
         {\small\chemfig{H_2N-[:330,,2](-[:270])-[:30]-[:330]-[:30]}}\begin{reponses}
            \mauvaise{butan-1-amine}
            \mauvaise{butan-2-amine}
            \mauvaise{3-méthylbutan-2-amine}
            \bonne{pentan-2-amine}
        \end{reponses}
    \end{question}
}\element{nomenca}{
    \begin{question}{nomenca21}
        \needspace{6cm}Quel est la formule topologique de: 2,3-diaminobutane~?
        \begin{reponses}
            \mauvaise{{\small\chemfig{H_2N-[:330,,2](-[:270])-[:30](-[:330])-[:90]}}}
            \mauvaise{{\small\chemfig{H_2N-[:330,,2](-[:270])-[:30]-[:330]}}}
            \mauvaise{{\small\chemfig{H_2N-[:30,,2]-[:330]-[:30]-[:330]}}}
            \bonne{{\small\chemfig{H_2N-[:330,,2](-[:270])-[:30](-[:330,,,1]NH_2)-[:90]}}}
        \end{reponses}
    \end{question}
}\element{nomenca}{
    \begin{question}{nomencar21}
        \needspace{6cm}Quel est le nom de cette molécule: ~? 
         \smallskip
         \\ 
         {\small\chemfig{H_2N-[:330,,2](-[:270])-[:30](-[:330,,,1]NH_2)-[:90]}}\begin{reponses}
            \mauvaise{butan-2-amine}
            \mauvaise{butan-1-amine}
            \mauvaise{2-amino-3-éthylbutane}
            \bonne{2,3-diaminobutane}
        \end{reponses}
    \end{question}
}\element{nomenca}{
    \begin{question}{nomenca24}
        \needspace{6cm}Quel est la formule topologique de: 2-méthylbutan-1-ol~?
        \begin{reponses}
            \mauvaise{{\small\chemfig{OH-[:210,,1]-[:150]-[:210](-[:150])-[:270]}}}
            \mauvaise{{\small\chemfig{OH-[:150,,1](=[:90]O)-[:210](-[:270])-[:150]-[:210]}}}
            \mauvaise{{\small\chemfig{O=[:30]-[:330](-[:270])-[:30]-[:330]}}}
            \bonne{{\small\chemfig{OH-[:150,,1]-[:210](-[:270])-[:150]-[:210]}}}
        \end{reponses}
    \end{question}
}\element{nomenca}{
    \begin{question}{nomencar24}
        \needspace{6cm}Quel est le nom de cette molécule: ~? 
         \smallskip
         \\ 
         {\small\chemfig{OH-[:150,,1]-[:210](-[:270])-[:150]-[:210]}}\begin{reponses}
            \mauvaise{3-pentylbutan-1-ol}
            \mauvaise{2-méthylbutanal}
            \mauvaise{3-éthylbutan-1-ol}
            \bonne{2-méthylbutan-1-ol}
        \end{reponses}
    \end{question}
}\element{nomenca}{
    \begin{question}{nomenca25}
        \needspace{6cm}Quel est la formule topologique de: 3-méthylbutan-1-ol~?
        \begin{reponses}
            \mauvaise{{\small\chemfig{O=[:30]-[:330](-[:270])-[:30]-[:330]}}}
            \mauvaise{{\small\chemfig{OH-[:150,,1]-[:210](-[:270])-[:150]-[:210]}}}
            \mauvaise{{\small\chemfig{OH-[:150,,1](=[:90]O)-[:210](-[:270])-[:150]-[:210]}}}
            \bonne{{\small\chemfig{OH-[:210,,1]-[:150]-[:210](-[:150])-[:270]}}}
        \end{reponses}
    \end{question}
}\element{nomenca}{
    \begin{question}{nomencar25}
        \needspace{6cm}Quel est le nom de cette molécule: ~? 
         \smallskip
         \\ 
         {\small\chemfig{OH-[:210,,1]-[:150]-[:210](-[:150])-[:270]}}\begin{reponses}
            \mauvaise{2-méthylbutan-1-ol}
            \mauvaise{acide 2-méthylbutanoïque}
            \mauvaise{3-pentylbutan-1-ol}
            \bonne{3-méthylbutan-1-ol}
        \end{reponses}
    \end{question}
}\element{nomenca}{
    \begin{question}{nomenca26}
        \needspace{6cm}Quel est la formule topologique de: acide 2-méthylbutanoïque~?
        \begin{reponses}
            \mauvaise{{\small\chemfig{O=[:30]-[:330](-[:270])-[:30]-[:330]}}}
            \mauvaise{{\small\chemfig{OH-[:150,,1]-[:210](-[:270])-[:150]-[:210]}}}
            \mauvaise{{\small\chemfig{OH-[:210,,1]-[:150]-[:210](-[:150])-[:270]}}}
            \bonne{{\small\chemfig{OH-[:150,,1](=[:90]O)-[:210](-[:270])-[:150]-[:210]}}}
        \end{reponses}
    \end{question}
}\element{nomenca}{
    \begin{question}{nomencar26}
        \needspace{6cm}Quel est le nom de cette molécule: ~? 
         \smallskip
         \\ 
         {\small\chemfig{OH-[:150,,1](=[:90]O)-[:210](-[:270])-[:150]-[:210]}}\begin{reponses}
            \mauvaise{2-méthylbutanal}
            \mauvaise{3-méthylbutan-1-ol}
            \mauvaise{2-méthylbutan-1-ol}
            \bonne{acide 2-méthylbutanoïque}
        \end{reponses}
    \end{question}
}\element{nomenca}{
    \begin{question}{nomenca27}
        \needspace{6cm}Quel est la formule topologique de: 2-méthylbutanal~?
        \begin{reponses}
            \mauvaise{{\small\chemfig{OH-[:210,,1]-[:150]-[:210](-[:150])-[:270]}}}
            \mauvaise{{\small\chemfig{OH-[:150,,1](=[:90]O)-[:210](-[:270])-[:150]-[:210]}}}
            \mauvaise{{\small\chemfig{OH-[:150,,1]-[:210](-[:270])-[:150]-[:210]}}}
            \bonne{{\small\chemfig{O=[:30]-[:330](-[:270])-[:30]-[:330]}}}
        \end{reponses}
    \end{question}
}\element{nomenca}{
    \begin{question}{nomencar27}
        \needspace{6cm}Quel est le nom de cette molécule: ~? 
         \smallskip
         \\ 
         {\small\chemfig{O=[:30]-[:330](-[:270])-[:30]-[:330]}}\begin{reponses}
            \mauvaise{acide 2-méthylbutanoïque}
            \mauvaise{3-méthylbutan-1-ol}
            \mauvaise{3-pentylbutan-1-ol}
            \bonne{2-méthylbutanal}
        \end{reponses}
    \end{question}
}\element{nomenca}{
    \begin{question}{nomenca32}
        \needspace{6cm}Quel est la formule topologique de: 3-méthylheptane~?
        \begin{reponses}
            \mauvaise{{\small\chemfig{-[:30]-[:330](-[:270]-[:210])-[:30]-[:330]-[:30]-[:330]}}}
            \mauvaise{{\small\chemfig{-[:90](-[:150]-[:210])-[:30]-[:330]-[:30]}}}
            \mauvaise{{\small\chemfig{-[:90](-[:150]-[:210])-[:30]-[:330]-[:30]-[:330]-[:30]}}}
            \bonne{{\small\chemfig{-[:90](-[:150]-[:210])-[:30]-[:330]-[:30]-[:330]}}}
        \end{reponses}
    \end{question}
}\element{nomenca}{
    \begin{question}{nomencar32}
        \needspace{6cm}Quel est le nom de cette molécule: ~? 
         \smallskip
         \\ 
         {\small\chemfig{-[:90](-[:150]-[:210])-[:30]-[:330]-[:30]-[:330]}}\begin{reponses}
            \mauvaise{3-éthylheptane}
            \mauvaise{4-méthylheptane}
            \mauvaise{5-méthylheptane}
            \bonne{3-méthylheptane}
        \end{reponses}
    \end{question}
}\element{nomenca}{
    \begin{question}{nomenca33}
        \needspace{6cm}Quel est la formule topologique de: 4-méthylheptane~?
        \begin{reponses}
            \mauvaise{{\small\chemfig{-[:90](-[:150]-[:210])-[:30]-[:330]-[:30]-[:330]}}}
            \mauvaise{{\small\chemfig{-[:90](-[:150]-[:210])-[:30]-[:330]-[:30]}}}
            \mauvaise{{\small\chemfig{-[:30]-[:330](-[:270]-[:210])-[:30]-[:330]-[:30]-[:330]}}}
            \bonne{{\small\chemfig{-[:90](-[:30]-[:330]-[:30])-[:150]-[:210]-[:150]}}}
        \end{reponses}
    \end{question}
}\element{nomenca}{
    \begin{question}{nomencar33}
        \needspace{6cm}Quel est le nom de cette molécule: ~? 
         \smallskip
         \\ 
         {\small\chemfig{-[:90](-[:30]-[:330]-[:30])-[:150]-[:210]-[:150]}}\begin{reponses}
            \mauvaise{3-méthylheptane}
            \mauvaise{5-méthylheptane}
            \mauvaise{3-méthylhexane}
            \bonne{4-méthylheptane}
        \end{reponses}
    \end{question}
}\element{nomenca}{
    \begin{question}{nomenca35}
        \needspace{6cm}Quel est la formule topologique de: 3-éthylheptane~?
        \begin{reponses}
            \mauvaise{{\small\chemfig{-[:90](-[:150]-[:210])-[:30]-[:330]-[:30]}}}
            \mauvaise{{\small\chemfig{-[:90](-[:150]-[:210])-[:30]-[:330]-[:30]-[:330]}}}
            \mauvaise{{\small\chemfig{-[:90](-[:30]-[:330]-[:30])-[:150]-[:210]-[:150]}}}
            \bonne{{\small\chemfig{-[:30]-[:330](-[:270]-[:210])-[:30]-[:330]-[:30]-[:330]}}}
        \end{reponses}
    \end{question}
}\element{nomenca}{
    \begin{question}{nomencar35}
        \needspace{6cm}Quel est le nom de cette molécule: ~? 
         \smallskip
         \\ 
         {\small\chemfig{-[:30]-[:330](-[:270]-[:210])-[:30]-[:330]-[:30]-[:330]}}\begin{reponses}
            \mauvaise{3-méthyloctane}
            \mauvaise{3-méthylheptane}
            \mauvaise{3-méthylhexane}
            \bonne{3-éthylheptane}
        \end{reponses}
    \end{question}
}\element{nomenca}{
    \begin{question}{nomenca36}
        \needspace{6cm}Quel est la formule topologique de: 3-méthylhexane~?
        \begin{reponses}
            \mauvaise{{\small\chemfig{-[:90](-[:150]-[:210])-[:30]-[:330]-[:30]-[:330]}}}
            \mauvaise{{\small\chemfig{-[:90](-[:150]-[:210])-[:30]-[:330]-[:30]-[:330]-[:30]}}}
            \mauvaise{{\small\chemfig{-[:30]-[:330](-[:270]-[:210])-[:30]-[:330]-[:30]-[:330]}}}
            \bonne{{\small\chemfig{-[:90](-[:150]-[:210])-[:30]-[:330]-[:30]}}}
        \end{reponses}
    \end{question}
}\element{nomenca}{
    \begin{question}{nomencar36}
        \needspace{6cm}Quel est le nom de cette molécule: ~? 
         \smallskip
         \\ 
         {\small\chemfig{-[:90](-[:150]-[:210])-[:30]-[:330]-[:30]}}\begin{reponses}
            \mauvaise{3-éthylheptane}
            \mauvaise{5-méthylheptane}
            \mauvaise{4-méthylheptane}
            \bonne{3-méthylhexane}
        \end{reponses}
    \end{question}
}\element{nomenca}{
    \begin{question}{nomenca37}
        \needspace{6cm}Quel est la formule topologique de: 3-méthyloctane~?
        \begin{reponses}
            \mauvaise{{\small\chemfig{-[:90](-[:150]-[:210])-[:30]-[:330]-[:30]-[:330]}}}
            \mauvaise{{\small\chemfig{-[:90](-[:150]-[:210])-[:30]-[:330]-[:30]}}}
            \mauvaise{{\small\chemfig{-[:30]-[:330](-[:270]-[:210])-[:30]-[:330]-[:30]-[:330]}}}
            \bonne{{\small\chemfig{-[:90](-[:150]-[:210])-[:30]-[:330]-[:30]-[:330]-[:30]}}}
        \end{reponses}
    \end{question}
}\element{nomenca}{
    \begin{question}{nomencar37}
        \needspace{6cm}Quel est le nom de cette molécule: ~? 
         \smallskip
         \\ 
         {\small\chemfig{-[:90](-[:150]-[:210])-[:30]-[:330]-[:30]-[:330]-[:30]}}\begin{reponses}
            \mauvaise{4-méthylheptane}
            \mauvaise{3-éthylheptane}
            \mauvaise{5-méthylheptane}
            \bonne{3-méthyloctane}
        \end{reponses}
    \end{question}
}\element{nomenca}{
    \begin{question}{nomenca40}
        \needspace{6cm}Quel est la formule topologique de: 2,3-diméthylpentanal~?
        \begin{reponses}
            \mauvaise{{\small\chemfig{O=[:330]-[:30]-[:330](-[:270])-[:30](-[:330])-[:90]}}}
            \mauvaise{{\small\chemfig{O=[:30]-[:330](-[:270])-[:30]-[:330](-[:30])-[:270]}}}
            \mauvaise{{\small\chemfig{O=[:150]-[:210](-[:270])-[:150]-[:210](-[:270])-[:150]-[:210]}}}
            \bonne{{\small\chemfig{O=[:210]-[:150](-[:90])-[:210](-[:270])-[:150]-[:210]}}}
        \end{reponses}
    \end{question}
}\element{nomenca}{
    \begin{question}{nomencar40}
        \needspace{6cm}Quel est le nom de cette molécule: ~? 
         \smallskip
         \\ 
         {\small\chemfig{O=[:210]-[:150](-[:90])-[:210](-[:270])-[:150]-[:210]}}\begin{reponses}
            \mauvaise{2,4-diméthylpentanal}
            \mauvaise{3,4-diméthylpentanal}
            \mauvaise{2-methyl-3-ethylpentanal}
            \bonne{2,3-diméthylpentanal}
        \end{reponses}
    \end{question}
}\element{nomenca}{
    \begin{question}{nomenca41}
        \needspace{6cm}Quel est la formule topologique de: 2,4-diméthylpentanal~?
        \begin{reponses}
            \mauvaise{{\small\chemfig{O=[:330]-[:30]-[:330](-[:270])-[:30](-[:330])-[:90]}}}
            \mauvaise{{\small\chemfig{O=[:150]-[:210](-[:270])-[:150]-[:210](-[:270])-[:150]-[:210]}}}
            \mauvaise{{\small\chemfig{O=[:210]-[:150](-[:90])-[:210](-[:270])-[:150]-[:210]}}}
            \bonne{{\small\chemfig{O=[:30]-[:330](-[:270])-[:30]-[:330](-[:30])-[:270]}}}
        \end{reponses}
    \end{question}
}\element{nomenca}{
    \begin{question}{nomencar41}
        \needspace{6cm}Quel est le nom de cette molécule: ~? 
         \smallskip
         \\ 
         {\small\chemfig{O=[:30]-[:330](-[:270])-[:30]-[:330](-[:30])-[:270]}}\begin{reponses}
            \mauvaise{2-methyl-3-ethylpentanal}
            \mauvaise{2,3-trimethylpentanal}
            \mauvaise{3,4-diméthylpentanal}
            \bonne{2,4-diméthylpentanal}
        \end{reponses}
    \end{question}
}\element{nomenca}{
    \begin{question}{nomenca42}
        \needspace{6cm}Quel est la formule topologique de: 2,4-diméthylhexanal~?
        \begin{reponses}
            \mauvaise{{\small\chemfig{O=[:330]-[:30]-[:330](-[:270])-[:30](-[:330])-[:90]}}}
            \mauvaise{{\small\chemfig{O=[:210]-[:150](-[:90])-[:210](-[:270])-[:150]-[:210]}}}
            \mauvaise{{\small\chemfig{O=[:30]-[:330](-[:270])-[:30]-[:330](-[:30])-[:270]}}}
            \bonne{{\small\chemfig{O=[:150]-[:210](-[:270])-[:150]-[:210](-[:270])-[:150]-[:210]}}}
        \end{reponses}
    \end{question}
}\element{nomenca}{
    \begin{question}{nomencar42}
        \needspace{6cm}Quel est le nom de cette molécule: ~? 
         \smallskip
         \\ 
         {\small\chemfig{O=[:150]-[:210](-[:270])-[:150]-[:210](-[:270])-[:150]-[:210]}}\begin{reponses}
            \mauvaise{2-methyl-3-ethylpentanal}
            \mauvaise{2,3-trimethylpentanal}
            \mauvaise{2,4-diméthylpentanal}
            \bonne{2,4-diméthylhexanal}
        \end{reponses}
    \end{question}
}\element{nomenca}{
    \begin{question}{nomenca43}
        \needspace{6cm}Quel est la formule topologique de: 3,4-diméthylpentanal~?
        \begin{reponses}
            \mauvaise{{\small\chemfig{O=[:150]-[:210](-[:270])-[:150]-[:210](-[:270])-[:150]-[:210]}}}
            \mauvaise{{\small\chemfig{O=[:210]-[:150](-[:90])-[:210](-[:270])-[:150]-[:210]}}}
            \mauvaise{{\small\chemfig{O=[:30]-[:330](-[:270])-[:30]-[:330](-[:30])-[:270]}}}
            \bonne{{\small\chemfig{O=[:330]-[:30]-[:330](-[:270])-[:30](-[:330])-[:90]}}}
        \end{reponses}
    \end{question}
}\element{nomenca}{
    \begin{question}{nomencar43}
        \needspace{6cm}Quel est le nom de cette molécule: ~? 
         \smallskip
         \\ 
         {\small\chemfig{O=[:330]-[:30]-[:330](-[:270])-[:30](-[:330])-[:90]}}\begin{reponses}
            \mauvaise{2,4-diméthylpentanal}
            \mauvaise{2,3-trimethylpentanal}
            \mauvaise{2,3-diméthylpentanal}
            \bonne{3,4-diméthylpentanal}
        \end{reponses}
    \end{question}
}\element{nomenca}{
    \begin{question}{nomenca48}
        \needspace{6cm}Quel est la formule topologique de: 3-éthyl-4-méthylhexan-2-one~?
        \begin{reponses}
            \mauvaise{{\small\chemfig{O=[:270](-[:210])-[:330](-[:30](-[:90])-[:330])-[:270]-[:210]}}}
            \mauvaise{{\small\chemfig{O=[:270](-[:210])-[:330](-[:270])-[:30](-[:330]-[:30])-[:90]-[:30]}}}
            \bonne{{\small\chemfig{O=[:270](-[:210])-[:330](-[:270]-[:210])-[:30](-[:90])-[:330]-[:30]}}}
        \end{reponses}
    \end{question}
}\element{nomenca}{
    \begin{question}{nomencar48}
        \needspace{6cm}Quel est le nom de cette molécule: ~? 
         \smallskip
         \\ 
         {\small\chemfig{O=[:270](-[:210])-[:330](-[:270]-[:210])-[:30](-[:90])-[:330]-[:30]}}\begin{reponses}
            \mauvaise{2,3-diméthyl-4-éthylhexan-5-one}
            \mauvaise{4-éthyl-3-méthylhexan-2-one}
            \mauvaise{3-propyl-4-methylhexan-2-one}
            \bonne{3-éthyl-4-méthylhexan-2-one}
        \end{reponses}
    \end{question}
}\element{nomenca}{
    \begin{question}{nomenca49}
        \needspace{6cm}Quel est la formule topologique de: 4-éthyl-3-méthylhexan-2-one~?
        \begin{reponses}
            \mauvaise{{\small\chemfig{O=[:270](-[:210])-[:330](-[:30](-[:90])-[:330])-[:270]-[:210]}}}
            \mauvaise{{\small\chemfig{O=[:270](-[:210])-[:330](-[:270]-[:210])-[:30](-[:90])-[:330]-[:30]}}}
            \bonne{{\small\chemfig{O=[:270](-[:210])-[:330](-[:270])-[:30](-[:330]-[:30])-[:90]-[:30]}}}
        \end{reponses}
    \end{question}
}\element{nomenca}{
    \begin{question}{nomencar49}
        \needspace{6cm}Quel est le nom de cette molécule: ~? 
         \smallskip
         \\ 
         {\small\chemfig{O=[:270](-[:210])-[:330](-[:270])-[:30](-[:330]-[:30])-[:90]-[:30]}}\begin{reponses}
            \mauvaise{2,3-diméthyl-4-éthylhexan-5-one}
            \mauvaise{3-propyl-4-methylhexan-2-one}
            \mauvaise{2-éthyl-5-methylheptan-2-one}
            \bonne{4-éthyl-3-méthylhexan-2-one}
        \end{reponses}
    \end{question}
}\element{nomenca}{
    \begin{question}{nomenca52}
        \needspace{6cm}Quel est la formule topologique de: 3-éthyl-4-méthylpentan-2-one~?
        \begin{reponses}
            \mauvaise{{\small\chemfig{O=[:270](-[:210])-[:330](-[:270]-[:210])-[:30](-[:90])-[:330]-[:30]}}}
            \mauvaise{{\small\chemfig{O=[:270](-[:210])-[:330](-[:270])-[:30](-[:330]-[:30])-[:90]-[:30]}}}
            \bonne{{\small\chemfig{O=[:270](-[:210])-[:330](-[:30](-[:90])-[:330])-[:270]-[:210]}}}
        \end{reponses}
    \end{question}
}\element{nomenca}{
    \begin{question}{nomencar52}
        \needspace{6cm}Quel est le nom de cette molécule: ~? 
         \smallskip
         \\ 
         {\small\chemfig{O=[:270](-[:210])-[:330](-[:30](-[:90])-[:330])-[:270]-[:210]}}\begin{reponses}
            \mauvaise{2-éthyl-5-methylheptan-2-one}
            \mauvaise{3-éthyl-4-méthylhexan-2-one}
            \mauvaise{4-éthyl-3-méthylhexan-2-one}
            \bonne{3-éthyl-4-méthylpentan-2-one}
        \end{reponses}
    \end{question}
}\element{nomenca}{
    \begin{question}{nomenca56}
        \needspace{6cm}Quel est la formule topologique de: acide 2-méthylpropanoïque~?
        \begin{reponses}
            \mauvaise{{\small\chemfig{OH-[:150,,1]-[:210](-[:150])-[:270]}}}
            \mauvaise{{\small\chemfig{OH-[:150,,1](=[:90]O)-[:210](-[:270])-[:150]-[:210](-[:150])-[:270]}}}
            \mauvaise{{\small\chemfig{HO-[:30,,2](=[:90]O)-[:330](-[:270]-[:210])-[:30](-[:90]-[:30])-[:330]%
            -[:30]-[:330]}}}
            \bonne{{\small\chemfig{OH-[:150,,1](=[:90]O)-[:210](-[:150])-[:270]}}}
        \end{reponses}
    \end{question}
}\element{nomenca}{
    \begin{question}{nomencar56}
        \needspace{6cm}Quel est le nom de cette molécule: ~? 
         \smallskip
         \\ 
         {\small\chemfig{OH-[:150,,1](=[:90]O)-[:210](-[:150])-[:270]}}\begin{reponses}
            \mauvaise{2-méthylpropan-1-ol}
            \mauvaise{acide 2,4-diméthylpentanoïque}
            \mauvaise{2-méthylpropanal}
            \bonne{acide 2-méthylpropanoïque}
        \end{reponses}
    \end{question}
}\element{nomenca}{
    \begin{question}{nomenca57}
        \needspace{6cm}Quel est la formule topologique de: 2-méthylpropan-1-ol~?
        \begin{reponses}
            \mauvaise{{\small\chemfig{OH-[:150,,1](=[:90]O)-[:210](-[:150])-[:270]}}}
            \mauvaise{{\small\chemfig{OH-[:150,,1](=[:90]O)-[:210](-[:270])-[:150]-[:210](-[:150])-[:270]}}}
            \mauvaise{{\small\chemfig{O=[:150]-[:210](-[:150])-[:270]}}}
            \bonne{{\small\chemfig{OH-[:150,,1]-[:210](-[:150])-[:270]}}}
        \end{reponses}
    \end{question}
}\element{nomenca}{
    \begin{question}{nomencar57}
        \needspace{6cm}Quel est le nom de cette molécule: ~? 
         \smallskip
         \\ 
         {\small\chemfig{OH-[:150,,1]-[:210](-[:150])-[:270]}}\begin{reponses}
            \mauvaise{acide 2-méthylpropanoïque}
            \mauvaise{acide 2-méthylbutanoïque}
            \mauvaise{acide 2,4-diméthylpentanoïque}
            \bonne{2-méthylpropan-1-ol}
        \end{reponses}
    \end{question}
}\element{nomenca}{
    \begin{question}{nomenca58}
        \needspace{6cm}Quel est la formule topologique de: acide 2,4-diméthylpentanoïque~?
        \begin{reponses}
            \mauvaise{{\small\chemfig{O=[:150]-[:210](-[:150])-[:270]}}}
            \mauvaise{{\small\chemfig{OH-[:150,,1]-[:210](-[:150])-[:270]}}}
            \mauvaise{{\small\chemfig{OH-[:150,,1](=[:90]O)-[:210](-[:150])-[:270]}}}
            \bonne{{\small\chemfig{OH-[:150,,1](=[:90]O)-[:210](-[:270])-[:150]-[:210](-[:150])-[:270]}}}
        \end{reponses}
    \end{question}
}\element{nomenca}{
    \begin{question}{nomencar58}
        \needspace{6cm}Quel est le nom de cette molécule: ~? 
         \smallskip
         \\ 
         {\small\chemfig{OH-[:150,,1](=[:90]O)-[:210](-[:270])-[:150]-[:210](-[:150])-[:270]}}\begin{reponses}
            \mauvaise{acide 2-méthylpropanoïque}
            \mauvaise{acide 2,3-diéthylhexanoïque}
            \mauvaise{2-méthylpropanal}
            \bonne{acide 2,4-diméthylpentanoïque}
        \end{reponses}
    \end{question}
}\element{nomenca}{
    \begin{question}{nomenca59}
        \needspace{6cm}Quel est la formule topologique de: 2-méthylpropanal~?
        \begin{reponses}
            \mauvaise{{\small\chemfig{OH-[:150,,1](=[:90]O)-[:210](-[:270])-[:150]-[:210]}}}
            \mauvaise{{\small\chemfig{OH-[:150,,1]-[:210](-[:150])-[:270]}}}
            \mauvaise{{\small\chemfig{OH-[:150,,1](=[:90]O)-[:210](-[:270])-[:150]-[:210](-[:150])-[:270]}}}
            \bonne{{\small\chemfig{O=[:150]-[:210](-[:150])-[:270]}}}
        \end{reponses}
    \end{question}
}\element{nomenca}{
    \begin{question}{nomencar59}
        \needspace{6cm}Quel est le nom de cette molécule: ~? 
         \smallskip
         \\ 
         {\small\chemfig{O=[:150]-[:210](-[:150])-[:270]}}\begin{reponses}
            \mauvaise{acide 2,3-diéthylhexanoïque}
            \mauvaise{2-méthylpropan-1-ol}
            \mauvaise{acide 2-méthylbutanoïque}
            \bonne{2-méthylpropanal}
        \end{reponses}
    \end{question}
}\element{nomenca}{
    \begin{question}{nomenca60}
        \needspace{6cm}Quel est la formule topologique de: acide 2-méthylbutanoïque~?
        \begin{reponses}
            \mauvaise{{\small\chemfig{HO-[:30,,2](=[:90]O)-[:330](-[:270]-[:210])-[:30](-[:90]-[:30])-[:330]%
            -[:30]-[:330]}}}
            \mauvaise{{\small\chemfig{OH-[:150,,1]-[:210](-[:150])-[:270]}}}
            \mauvaise{{\small\chemfig{OH-[:150,,1](=[:90]O)-[:210](-[:270])-[:150]-[:210](-[:150])-[:270]}}}
            \bonne{{\small\chemfig{OH-[:150,,1](=[:90]O)-[:210](-[:270])-[:150]-[:210]}}}
        \end{reponses}
    \end{question}
}\element{nomenca}{
    \begin{question}{nomencar60}
        \needspace{6cm}Quel est le nom de cette molécule: ~? 
         \smallskip
         \\ 
         {\small\chemfig{OH-[:150,,1](=[:90]O)-[:210](-[:270])-[:150]-[:210]}}\begin{reponses}
            \mauvaise{2-méthylpropanal}
            \mauvaise{acide 2,3-diéthylhexanoïque}
            \mauvaise{acide 2-méthylpropanoïque}
            \bonne{acide 2-méthylbutanoïque}
        \end{reponses}
    \end{question}
}\element{nomenca}{
    \begin{question}{nomenca61}
        \needspace{6cm}Quel est la formule topologique de: acide 2,3-diéthylhexanoïque~?
        \begin{reponses}
            \mauvaise{{\small\chemfig{OH-[:150,,1]-[:210](-[:150])-[:270]}}}
            \mauvaise{{\small\chemfig{O=[:150]-[:210](-[:150])-[:270]}}}
            \mauvaise{{\small\chemfig{OH-[:150,,1](=[:90]O)-[:210](-[:270])-[:150]-[:210]}}}
            \bonne{{\small\chemfig{HO-[:30,,2](=[:90]O)-[:330](-[:270]-[:210])-[:30](-[:90]-[:30])-[:330]%
            -[:30]-[:330]}}}
        \end{reponses}
    \end{question}
}\element{nomenca}{
    \begin{question}{nomencar61}
        \needspace{6cm}Quel est le nom de cette molécule: ~? 
         \smallskip
         \\ 
         {\small\chemfig{HO-[:30,,2](=[:90]O)-[:330](-[:270]-[:210])-[:30](-[:90]-[:30])-[:330]%
        -[:30]-[:330]}}\begin{reponses}
            \mauvaise{acide 2,4-diméthylpentanoïque}
            \mauvaise{2-méthylpropanal}
            \mauvaise{acide 2-méthylbutanoïque}
            \bonne{acide 2,3-diéthylhexanoïque}
        \end{reponses}
    \end{question}
}

%%% fin des elements
\element{groupes}{
\section{Nomenclature des alcanes linéaires (simple)}
\begin{multicols}{2}
\insertgroup[2]{falcn}
\insertgroup[2]{alcn}
\end{multicols}

\section{Un peu d’alcool...}
\begin{multicols}{2}
\insertgroup[4]{alcl}
\end{multicols}

\section{Groupes fonctionnels...}
\begin{multicols}{2}
\insertgroup[4]{grft}
\end{multicols}

\section{Comptons les hydrogènes...}
\begin{multicols}{2}
\insertgroup[2]{brut}
\end{multicols}

\section{Nomenclature (moins simple)}
\begin{multicols}{2}
\insertgroup[4]{nomenca}
\end{multicols}

\section{Nomenclature stéréochimique}
\begin{multicols}{2}
%\insertgroup[4]{nomenca}
\end{multicols}

\AMCcleardoublepage
}

\csvreader[head to column names]{liste.csv}{}{\sujet}

\end{document}
