\element{nomenca}{
    \begin{question}{nomenca0}
        Quel est la formule topologique de: 4-éthylheptane~?
        \begin{reponses}
            \mauvaise{{\small\chemfig{H_2N-[:330,,2](-[:270])-[:30]-[:330]}}}
            \mauvaise{{\small\chemfig{-[:90](-[:150]-[:210])-[:30](-[:90])-[:330]-[:30]}}}
            \mauvaise{{\small\chemfig{O=[:210]-[:150](-[:90])-[:210](-[:270])-[:150]-[:210]}}}
            \bonne{{\small\chemfig{-[:30]-[:90](-[:30]-[:330]-[:30])-[:150]-[:210]-[:150]}}}
        \end{reponses}
    \end{question}
}\element{nomenca}{
    \begin{question}{nomencar0}
        Quel est le nom de cette molécule: ~?
         \\ {\small\chemfig{-[:30]-[:90](-[:30]-[:330]-[:30])-[:150]-[:210]-[:150]}}\begin{reponses}
            \mauvaise{2-méthylbutan-1-ol}
            \mauvaise{3-méthylheptane}
            \mauvaise{2,3-diméthylpentanal}
            \bonne{4-éthylheptane}
        \end{reponses}
    \end{question}
}\element{nomenca}{
    \begin{question}{nomenca1}
        Quel est la formule topologique de: 2,3-diméthylpentane~?
        \begin{reponses}
            \mauvaise{{\small\chemfig{-[:90](-[:150]-[:210])-[:30]-[:330]-[:30]-[:330]}}}
            \mauvaise{{\small\chemfig{OH-[:150,,1]-[:210](-[:270])-[:150]-[:210]}}}
            \mauvaise{{\small\chemfig{OH-[:150,,1](=[:90]O)-[:210](-[:270])-[:150]}}}
            \bonne{{\small\chemfig{-[:90](-[:30](-[:90])-[:330])-[:150]-[:210]}}}
        \end{reponses}
    \end{question}
}\element{nomenca}{
    \begin{question}{nomencar1}
        Quel est le nom de cette molécule: ~?
         \\ {\small\chemfig{-[:90](-[:30](-[:90])-[:330])-[:150]-[:210]}}\begin{reponses}
            \mauvaise{Acide 2-méthylpropanoïque}
            \mauvaise{4-éthylheptane}
            \mauvaise{3,4-diméthylhexane}
            \bonne{2,3-diméthylpentane}
        \end{reponses}
    \end{question}
}\element{nomenca}{
    \begin{question}{nomenca2}
        Quel est la formule topologique de: 3,4-diméthylhexane~?
        \begin{reponses}
            \mauvaise{{\small\chemfig{O=[:270](-[:210])-[:330](-[:270]-[:210])-[:30](-[:90])-[:330]-[:30]}}}
            \mauvaise{{\small\chemfig{OH-[:150,,1](=[:90]O)-[:210](-[:270])-[:150]}}}
            \mauvaise{{\small\chemfig{-[:30]-[:90](-[:30]-[:330]-[:30])-[:150]-[:210]-[:150]}}}
            \bonne{{\small\chemfig{-[:90](-[:150]-[:210])-[:30](-[:90])-[:330]-[:30]}}}
        \end{reponses}
    \end{question}
}\element{nomenca}{
    \begin{question}{nomencar2}
        Quel est le nom de cette molécule: ~?
         \\ {\small\chemfig{-[:90](-[:150]-[:210])-[:30](-[:90])-[:330]-[:30]}}\begin{reponses}
            \mauvaise{2-méthylbutan-1-ol}
            \mauvaise{butan-2-amine}
            \mauvaise{2,3-diméthylpentanal}
            \bonne{3,4-diméthylhexane}
        \end{reponses}
    \end{question}
}\element{nomenca}{
    \begin{question}{nomenca3}
        Quel est la formule topologique de: butan-2-amine~?
        \begin{reponses}
            \mauvaise{{\small\chemfig{-[:30]-[:90](-[:30]-[:330]-[:30])-[:150]-[:210]-[:150]}}}
            \mauvaise{{\small\chemfig{O=[:270](-[:210])-[:330](-[:270]-[:210])-[:30](-[:90])-[:330]-[:30]}}}
            \mauvaise{{\small\chemfig{OH-[:150,,1]-[:210](-[:270])-[:150]-[:210]}}}
            \bonne{{\small\chemfig{H_2N-[:330,,2](-[:270])-[:30]-[:330]}}}
        \end{reponses}
    \end{question}
}\element{nomenca}{
    \begin{question}{nomencar3}
        Quel est le nom de cette molécule: ~?
         \\ {\small\chemfig{H_2N-[:330,,2](-[:270])-[:30]-[:330]}}\begin{reponses}
            \mauvaise{2,3-diméthylpentane}
            \mauvaise{2,3-diméthylpentanal}
            \mauvaise{Acide 2-méthylpropanoïque}
            \bonne{butan-2-amine}
        \end{reponses}
    \end{question}
}\element{nomenca}{
    \begin{question}{nomenca4}
        Quel est la formule topologique de: 2-méthylbutan-1-ol~?
        \begin{reponses}
            \mauvaise{{\small\chemfig{O=[:270](-[:210])-[:330](-[:270]-[:210])-[:30](-[:90])-[:330]-[:30]}}}
            \mauvaise{{\small\chemfig{-[:90](-[:150]-[:210])-[:30](-[:90])-[:330]-[:30]}}}
            \mauvaise{{\small\chemfig{O=[:210]-[:150](-[:90])-[:210](-[:270])-[:150]-[:210]}}}
            \bonne{{\small\chemfig{OH-[:150,,1]-[:210](-[:270])-[:150]-[:210]}}}
        \end{reponses}
    \end{question}
}\element{nomenca}{
    \begin{question}{nomencar4}
        Quel est le nom de cette molécule: ~?
         \\ {\small\chemfig{OH-[:150,,1]-[:210](-[:270])-[:150]-[:210]}}\begin{reponses}
            \mauvaise{2,3-diméthylpentanal}
            \mauvaise{3-éthyl-4-méthylhexan-2-one}
            \mauvaise{3,4-diméthylhexane}
            \bonne{2-méthylbutan-1-ol}
        \end{reponses}
    \end{question}
}\element{nomenca}{
    \begin{question}{nomenca5}
        Quel est la formule topologique de: 3-méthylheptane~?
        \begin{reponses}
            \mauvaise{{\small\chemfig{OH-[:150,,1](=[:90]O)-[:210](-[:270])-[:150]}}}
            \mauvaise{{\small\chemfig{OH-[:150,,1]-[:210](-[:270])-[:150]-[:210]}}}
            \mauvaise{{\small\chemfig{O=[:210]-[:150](-[:90])-[:210](-[:270])-[:150]-[:210]}}}
            \bonne{{\small\chemfig{-[:90](-[:150]-[:210])-[:30]-[:330]-[:30]-[:330]}}}
        \end{reponses}
    \end{question}
}\element{nomenca}{
    \begin{question}{nomencar5}
        Quel est le nom de cette molécule: ~?
         \\ {\small\chemfig{-[:90](-[:150]-[:210])-[:30]-[:330]-[:30]-[:330]}}\begin{reponses}
            \mauvaise{2-méthylbutan-1-ol}
            \mauvaise{2,3-diméthylpentanal}
            \mauvaise{4-éthylheptane}
            \bonne{3-méthylheptane}
        \end{reponses}
    \end{question}
}\element{nomenca}{
    \begin{question}{nomenca6}
        Quel est la formule topologique de: 2,3-diméthylpentanal~?
        \begin{reponses}
            \mauvaise{{\small\chemfig{-[:90](-[:150]-[:210])-[:30]-[:330]-[:30]-[:330]}}}
            \mauvaise{{\small\chemfig{-[:30]-[:90](-[:30]-[:330]-[:30])-[:150]-[:210]-[:150]}}}
            \mauvaise{{\small\chemfig{OH-[:150,,1]-[:210](-[:270])-[:150]-[:210]}}}
            \bonne{{\small\chemfig{O=[:210]-[:150](-[:90])-[:210](-[:270])-[:150]-[:210]}}}
        \end{reponses}
    \end{question}
}\element{nomenca}{
    \begin{question}{nomencar6}
        Quel est le nom de cette molécule: ~?
         \\ {\small\chemfig{O=[:210]-[:150](-[:90])-[:210](-[:270])-[:150]-[:210]}}\begin{reponses}
            \mauvaise{2,3-diméthylpentane}
            \mauvaise{2-méthylbutan-1-ol}
            \mauvaise{4-éthylheptane}
            \bonne{2,3-diméthylpentanal}
        \end{reponses}
    \end{question}
}\element{nomenca}{
    \begin{question}{nomenca7}
        Quel est la formule topologique de: 3-éthyl-4-méthylhexan-2-one~?
        \begin{reponses}
            \mauvaise{{\small\chemfig{H_2N-[:330,,2](-[:270])-[:30]-[:330]}}}
            \mauvaise{{\small\chemfig{-[:90](-[:150]-[:210])-[:30]-[:330]-[:30]-[:330]}}}
            \mauvaise{{\small\chemfig{OH-[:150,,1]-[:210](-[:270])-[:150]-[:210]}}}
            \bonne{{\small\chemfig{O=[:270](-[:210])-[:330](-[:270]-[:210])-[:30](-[:90])-[:330]-[:30]}}}
        \end{reponses}
    \end{question}
}\element{nomenca}{
    \begin{question}{nomencar7}
        Quel est le nom de cette molécule: ~?
         \\ {\small\chemfig{O=[:270](-[:210])-[:330](-[:270]-[:210])-[:30](-[:90])-[:330]-[:30]}}\begin{reponses}
            \mauvaise{3-méthylheptane}
            \mauvaise{4-éthylheptane}
            \mauvaise{3,4-diméthylhexane}
            \bonne{3-éthyl-4-méthylhexan-2-one}
        \end{reponses}
    \end{question}
}\element{nomenca}{
    \begin{question}{nomenca8}
        Quel est la formule topologique de: Acide 2-méthylpropanoïque~?
        \begin{reponses}
            \mauvaise{{\small\chemfig{-[:90](-[:30](-[:90])-[:330])-[:150]-[:210]}}}
            \mauvaise{{\small\chemfig{-[:90](-[:150]-[:210])-[:30](-[:90])-[:330]-[:30]}}}
            \mauvaise{{\small\chemfig{-[:30]-[:90](-[:30]-[:330]-[:30])-[:150]-[:210]-[:150]}}}
            \bonne{{\small\chemfig{OH-[:150,,1](=[:90]O)-[:210](-[:270])-[:150]}}}
        \end{reponses}
    \end{question}
}\element{nomenca}{
    \begin{question}{nomencar8}
        Quel est le nom de cette molécule: ~?
         \\ {\small\chemfig{OH-[:150,,1](=[:90]O)-[:210](-[:270])-[:150]}}\begin{reponses}
            \mauvaise{4-éthylheptane}
            \mauvaise{3,4-diméthylhexane}
            \mauvaise{2-méthylbutan-1-ol}
            \bonne{Acide 2-méthylpropanoïque}
        \end{reponses}
    \end{question}
}