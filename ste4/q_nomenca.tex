\element{nomenca}{
    \begin{question}{nomenca0}
        \needspace{6cm}Quel est la formule topologique de: 4-éthylheptane~?
        \begin{reponses}
            \mauvaise{{\small\chemfig{-[:330](-[:270])-[:30]-[:330]-[:30]-[:330]}}}
            \mauvaise{{\small\chemfig{-[:210]-[:150]-[:210](-[:150]-[:210]-[:150])-[:270]-[:210]-[:270]}}}
            \mauvaise{{\small\chemfig{-[:90](-[:30]-[:330]-[:30])-[:150]-[:210]-[:150]}}}
            \bonne{{\small\chemfig{-[:30]-[:90](-[:30]-[:330]-[:30])-[:150]-[:210]-[:150]}}}
        \end{reponses}
    \end{question}
}\element{nomenca}{
    \begin{question}{nomencar0}
        \needspace{6cm}Quel est le nom de cette molécule: ~? 
         \smallskip
         \\ 
         {\small\chemfig{-[:30]-[:90](-[:30]-[:330]-[:30])-[:150]-[:210]-[:150]}}\begin{reponses}
            \mauvaise{2-méthylhexane}
            \mauvaise{2-éthylheptane}
            \mauvaise{3-méthyloctane}
            \bonne{4-éthylheptane}
        \end{reponses}
    \end{question}
}\element{nomenca}{
    \begin{question}{nomenca1}
        \needspace{6cm}Quel est la formule topologique de: 3-éthylheptane~?
        \begin{reponses}
            \mauvaise{{\small\chemfig{-[:330](-[:270])-[:30]-[:330]-[:30]-[:330]}}}
            \mauvaise{{\small\chemfig{-[:90](-[:150]-[:210])-[:30]-[:330]-[:30]-[:330]-[:30]}}}
            \mauvaise{{\small\chemfig{-[:210]-[:150]-[:210](-[:150]-[:210]-[:150])-[:270]-[:210]-[:270]}}}
            \bonne{{\small\chemfig{-[:30]-[:330](-[:270]-[:210])-[:30]-[:330]-[:30]-[:330]}}}
        \end{reponses}
    \end{question}
}\element{nomenca}{
    \begin{question}{nomencar1}
        \needspace{6cm}Quel est le nom de cette molécule: ~? 
         \smallskip
         \\ 
         {\small\chemfig{-[:30]-[:330](-[:270]-[:210])-[:30]-[:330]-[:30]-[:330]}}\begin{reponses}
            \mauvaise{4-méthylheptane}
            \mauvaise{4-propylheptane}
            \mauvaise{2-méthylhexane}
            \bonne{3-éthylheptane}
        \end{reponses}
    \end{question}
}\element{nomenca}{
    \begin{question}{nomenca4}
        \needspace{6cm}Quel est la formule topologique de: 4-méthylheptane~?
        \begin{reponses}
            \mauvaise{{\small\chemfig{-[:30]-[:90](-[:30]-[:330]-[:30])-[:150]-[:210]-[:150]}}}
            \mauvaise{{\small\chemfig{-[:330](-[:270])-[:30]-[:330]-[:30]-[:330]}}}
            \mauvaise{{\small\chemfig{-[:90](-[:150]-[:210])-[:30]-[:330]-[:30]-[:330]-[:30]}}}
            \bonne{{\small\chemfig{-[:90](-[:30]-[:330]-[:30])-[:150]-[:210]-[:150]}}}
        \end{reponses}
    \end{question}
}\element{nomenca}{
    \begin{question}{nomencar4}
        \needspace{6cm}Quel est le nom de cette molécule: ~? 
         \smallskip
         \\ 
         {\small\chemfig{-[:90](-[:30]-[:330]-[:30])-[:150]-[:210]-[:150]}}\begin{reponses}
            \mauvaise{4-propylheptane}
            \mauvaise{3-éthylheptane}
            \mauvaise{4-éthylheptane}
            \bonne{4-méthylheptane}
        \end{reponses}
    \end{question}
}\element{nomenca}{
    \begin{question}{nomenca5}
        \needspace{6cm}Quel est la formule topologique de: 4-propylheptane~?
        \begin{reponses}
            \mauvaise{{\small\chemfig{-[:90](-[:150]-[:210])-[:30]-[:330]-[:30]-[:330]-[:30]}}}
            \mauvaise{{\small\chemfig{-[:30]-[:90](-[:30]-[:330]-[:30])-[:150]-[:210]-[:150]}}}
            \mauvaise{{\small\chemfig{-[:330](-[:270])-[:30]-[:330]-[:30]-[:330]}}}
            \bonne{{\small\chemfig{-[:210]-[:150]-[:210](-[:150]-[:210]-[:150])-[:270]-[:210]-[:270]}}}
        \end{reponses}
    \end{question}
}\element{nomenca}{
    \begin{question}{nomencar5}
        \needspace{6cm}Quel est le nom de cette molécule: ~? 
         \smallskip
         \\ 
         {\small\chemfig{-[:210]-[:150]-[:210](-[:150]-[:210]-[:150])-[:270]-[:210]-[:270]}}\begin{reponses}
            \mauvaise{3-éthylheptane}
            \mauvaise{2-méthylhexane}
            \mauvaise{5-éthylheptane}
            \bonne{4-propylheptane}
        \end{reponses}
    \end{question}
}\element{nomenca}{
    \begin{question}{nomenca6}
        \needspace{6cm}Quel est la formule topologique de: 2-méthylhexane~?
        \begin{reponses}
            \mauvaise{{\small\chemfig{-[:90](-[:30]-[:330]-[:30])-[:150]-[:210]-[:150]}}}
            \mauvaise{{\small\chemfig{-[:210]-[:150]-[:210](-[:150]-[:210]-[:150])-[:270]-[:210]-[:270]}}}
            \mauvaise{{\small\chemfig{-[:30]-[:90](-[:30]-[:330]-[:30])-[:150]-[:210]-[:150]}}}
            \bonne{{\small\chemfig{-[:330](-[:270])-[:30]-[:330]-[:30]-[:330]}}}
        \end{reponses}
    \end{question}
}\element{nomenca}{
    \begin{question}{nomencar6}
        \needspace{6cm}Quel est le nom de cette molécule: ~? 
         \smallskip
         \\ 
         {\small\chemfig{-[:330](-[:270])-[:30]-[:330]-[:30]-[:330]}}\begin{reponses}
            \mauvaise{4-éthylheptane}
            \mauvaise{3-méthyloctane}
            \mauvaise{2-éthylheptane}
            \bonne{2-méthylhexane}
        \end{reponses}
    \end{question}
}\element{nomenca}{
    \begin{question}{nomenca7}
        \needspace{6cm}Quel est la formule topologique de: 3-méthyloctane~?
        \begin{reponses}
            \mauvaise{{\small\chemfig{-[:90](-[:30]-[:330]-[:30])-[:150]-[:210]-[:150]}}}
            \mauvaise{{\small\chemfig{-[:30]-[:330](-[:270]-[:210])-[:30]-[:330]-[:30]-[:330]}}}
            \mauvaise{{\small\chemfig{-[:30]-[:90](-[:30]-[:330]-[:30])-[:150]-[:210]-[:150]}}}
            \bonne{{\small\chemfig{-[:90](-[:150]-[:210])-[:30]-[:330]-[:30]-[:330]-[:30]}}}
        \end{reponses}
    \end{question}
}\element{nomenca}{
    \begin{question}{nomencar7}
        \needspace{6cm}Quel est le nom de cette molécule: ~? 
         \smallskip
         \\ 
         {\small\chemfig{-[:90](-[:150]-[:210])-[:30]-[:330]-[:30]-[:330]-[:30]}}\begin{reponses}
            \mauvaise{5-éthylheptane}
            \mauvaise{4-éthylheptane}
            \mauvaise{2-éthylheptane}
            \bonne{3-méthyloctane}
        \end{reponses}
    \end{question}
}\element{nomenca}{
    \begin{question}{nomenca8}
        \needspace{6cm}Quel est la formule topologique de: 2,3-diméthylpentane~?
        \begin{reponses}
            \mauvaise{{\small\chemfig{-[:300](-[:30]-[:330])(-[:300])-[:210]-[:150]}}}
            \mauvaise{{\small\chemfig{-[:330](-[:270])-[:30]-[:330]-[:30]}}}
            \mauvaise{{\small\chemfig{-[:330](-[:270])-[:30]-[:330](-[:30])-[:270]}}}
            \bonne{{\small\chemfig{-[:90](-[:30](-[:90])-[:330])-[:150]-[:210]}}}
        \end{reponses}
    \end{question}
}\element{nomenca}{
    \begin{question}{nomencar8}
        \needspace{6cm}Quel est le nom de cette molécule: ~? 
         \smallskip
         \\ 
         {\small\chemfig{-[:90](-[:30](-[:90])-[:330])-[:150]-[:210]}}\begin{reponses}
            \mauvaise{2-méthylpentane}
            \mauvaise{2,4-diméthylpentane}
            \mauvaise{3,3-diméthylpentane}
            \bonne{2,3-diméthylpentane}
        \end{reponses}
    \end{question}
}\element{nomenca}{
    \begin{question}{nomenca9}
        \needspace{6cm}Quel est la formule topologique de: 2,4-diméthylpentane~?
        \begin{reponses}
            \mauvaise{{\small\chemfig{-[:300](-[:30]-[:330])(-[:300])-[:210]-[:150]}}}
            \mauvaise{{\small\chemfig{-[:90](-[:30](-[:90])-[:330])-[:150]-[:210]}}}
            \mauvaise{{\small\chemfig{-[:330](-[:270])-[:30]-[:330]-[:30]}}}
            \bonne{{\small\chemfig{-[:330](-[:270])-[:30]-[:330](-[:30])-[:270]}}}
        \end{reponses}
    \end{question}
}\element{nomenca}{
    \begin{question}{nomencar9}
        \needspace{6cm}Quel est le nom de cette molécule: ~? 
         \smallskip
         \\ 
         {\small\chemfig{-[:330](-[:270])-[:30]-[:330](-[:30])-[:270]}}\begin{reponses}
            \mauvaise{2,3-diméthylpentane}
            \mauvaise{2-méthylpentane}
            \mauvaise{2,3-diéthylpentane}
            \bonne{2,4-diméthylpentane}
        \end{reponses}
    \end{question}
}\element{nomenca}{
    \begin{question}{nomenca10}
        \needspace{6cm}Quel est la formule topologique de: 3,3-diméthylpentane~?
        \begin{reponses}
            \mauvaise{{\small\chemfig{-[:330](-[:270])-[:30]-[:330](-[:30])-[:270]}}}
            \mauvaise{{\small\chemfig{-[:90](-[:30](-[:90])-[:330])-[:150]-[:210]}}}
            \mauvaise{{\small\chemfig{-[:330](-[:270])-[:30]-[:330]-[:30]}}}
            \bonne{{\small\chemfig{-[:300](-[:30]-[:330])(-[:300])-[:210]-[:150]}}}
        \end{reponses}
    \end{question}
}\element{nomenca}{
    \begin{question}{nomencar10}
        \needspace{6cm}Quel est le nom de cette molécule: ~? 
         \smallskip
         \\ 
         {\small\chemfig{-[:300](-[:30]-[:330])(-[:300])-[:210]-[:150]}}\begin{reponses}
            \mauvaise{2-méthylpentane}
            \mauvaise{2,4-diméthylpentane}
            \mauvaise{2,3-diéthylpentane}
            \bonne{3,3-diméthylpentane}
        \end{reponses}
    \end{question}
}\element{nomenca}{
    \begin{question}{nomenca11}
        \needspace{6cm}Quel est la formule topologique de: 2-méthylpentane~?
        \begin{reponses}
            \mauvaise{{\small\chemfig{-[:330](-[:270])-[:30]-[:330](-[:30])-[:270]}}}
            \mauvaise{{\small\chemfig{-[:300](-[:30]-[:330])(-[:300])-[:210]-[:150]}}}
            \mauvaise{{\small\chemfig{-[:90](-[:30](-[:90])-[:330])-[:150]-[:210]}}}
            \bonne{{\small\chemfig{-[:330](-[:270])-[:30]-[:330]-[:30]}}}
        \end{reponses}
    \end{question}
}\element{nomenca}{
    \begin{question}{nomencar11}
        \needspace{6cm}Quel est le nom de cette molécule: ~? 
         \smallskip
         \\ 
         {\small\chemfig{-[:330](-[:270])-[:30]-[:330]-[:30]}}\begin{reponses}
            \mauvaise{2,3-diméthylpentane}
            \mauvaise{2,3-diéthylpentane}
            \mauvaise{2,4-diméthylpentane}
            \bonne{2-méthylpentane}
        \end{reponses}
    \end{question}
}\element{nomenca}{
    \begin{question}{nomenca16}
        \needspace{6cm}Quel est la formule topologique de: butan-2-amine~?
        \begin{reponses}
            \mauvaise{{\small\chemfig{H_2N-[:330,,2](-[:270])-[:30]-[:330]-[:30]}}}
            \mauvaise{{\small\chemfig{H_2N-[:330,,2](-[:270])-[:30](-[:330])-[:90]}}}
            \mauvaise{{\small\chemfig{H_2N-[:30,,2]-[:330]-[:30]-[:330]}}}
            \bonne{{\small\chemfig{H_2N-[:330,,2](-[:270])-[:30]-[:330]}}}
        \end{reponses}
    \end{question}
}\element{nomenca}{
    \begin{question}{nomencar16}
        \needspace{6cm}Quel est le nom de cette molécule: ~? 
         \smallskip
         \\ 
         {\small\chemfig{H_2N-[:330,,2](-[:270])-[:30]-[:330]}}\begin{reponses}
            \mauvaise{2-amino-3-éthylbutane}
            \mauvaise{pentan-2-amine}
            \mauvaise{butan-1-amine}
            \bonne{butan-2-amine}
        \end{reponses}
    \end{question}
}\element{nomenca}{
    \begin{question}{nomenca17}
        \needspace{6cm}Quel est la formule topologique de: butan-1-amine~?
        \begin{reponses}
            \mauvaise{{\small\chemfig{H_2N-[:330,,2](-[:270])-[:30]-[:330]-[:30]}}}
            \mauvaise{{\small\chemfig{H_2N-[:330,,2](-[:270])-[:30]-[:330]}}}
            \mauvaise{{\small\chemfig{H_2N-[:330,,2](-[:270])-[:30](-[:330])-[:90]}}}
            \bonne{{\small\chemfig{H_2N-[:30,,2]-[:330]-[:30]-[:330]}}}
        \end{reponses}
    \end{question}
}\element{nomenca}{
    \begin{question}{nomencar17}
        \needspace{6cm}Quel est le nom de cette molécule: ~? 
         \smallskip
         \\ 
         {\small\chemfig{H_2N-[:30,,2]-[:330]-[:30]-[:330]}}\begin{reponses}
            \mauvaise{butan-2-amine}
            \mauvaise{2-amino-3-éthylbutane}
            \mauvaise{pentan-2-amine}
            \bonne{butan-1-amine}
        \end{reponses}
    \end{question}
}\element{nomenca}{
    \begin{question}{nomenca18}
        \needspace{6cm}Quel est la formule topologique de: 3-méthylbutan-2-amine~?
        \begin{reponses}
            \mauvaise{{\small\chemfig{H_2N-[:330,,2](-[:270])-[:30](-[:330,,,1]NH_2)-[:90]}}}
            \mauvaise{{\small\chemfig{H_2N-[:330,,2](-[:270])-[:30]-[:330]}}}
            \mauvaise{{\small\chemfig{H_2N-[:30,,2]-[:330]-[:30]-[:330]}}}
            \bonne{{\small\chemfig{H_2N-[:330,,2](-[:270])-[:30](-[:330])-[:90]}}}
        \end{reponses}
    \end{question}
}\element{nomenca}{
    \begin{question}{nomencar18}
        \needspace{6cm}Quel est le nom de cette molécule: ~? 
         \smallskip
         \\ 
         {\small\chemfig{H_2N-[:330,,2](-[:270])-[:30](-[:330])-[:90]}}\begin{reponses}
            \mauvaise{pentan-2-amine}
            \mauvaise{2-amino-3-éthylbutane}
            \mauvaise{butan-1-amine}
            \bonne{3-méthylbutan-2-amine}
        \end{reponses}
    \end{question}
}\element{nomenca}{
    \begin{question}{nomenca20}
        \needspace{6cm}Quel est la formule topologique de: pentan-2-amine~?
        \begin{reponses}
            \mauvaise{{\small\chemfig{H_2N-[:330,,2](-[:270])-[:30](-[:330])-[:90]}}}
            \mauvaise{{\small\chemfig{H_2N-[:330,,2](-[:270])-[:30](-[:330,,,1]NH_2)-[:90]}}}
            \mauvaise{{\small\chemfig{H_2N-[:330,,2](-[:270])-[:30]-[:330]}}}
            \bonne{{\small\chemfig{H_2N-[:330,,2](-[:270])-[:30]-[:330]-[:30]}}}
        \end{reponses}
    \end{question}
}\element{nomenca}{
    \begin{question}{nomencar20}
        \needspace{6cm}Quel est le nom de cette molécule: ~? 
         \smallskip
         \\ 
         {\small\chemfig{H_2N-[:330,,2](-[:270])-[:30]-[:330]-[:30]}}\begin{reponses}
            \mauvaise{butan-1-amine}
            \mauvaise{butan-2-amine}
            \mauvaise{3-méthylbutan-2-amine}
            \bonne{pentan-2-amine}
        \end{reponses}
    \end{question}
}\element{nomenca}{
    \begin{question}{nomenca21}
        \needspace{6cm}Quel est la formule topologique de: 2,3-diaminobutane~?
        \begin{reponses}
            \mauvaise{{\small\chemfig{H_2N-[:330,,2](-[:270])-[:30](-[:330])-[:90]}}}
            \mauvaise{{\small\chemfig{H_2N-[:330,,2](-[:270])-[:30]-[:330]}}}
            \mauvaise{{\small\chemfig{H_2N-[:30,,2]-[:330]-[:30]-[:330]}}}
            \bonne{{\small\chemfig{H_2N-[:330,,2](-[:270])-[:30](-[:330,,,1]NH_2)-[:90]}}}
        \end{reponses}
    \end{question}
}\element{nomenca}{
    \begin{question}{nomencar21}
        \needspace{6cm}Quel est le nom de cette molécule: ~? 
         \smallskip
         \\ 
         {\small\chemfig{H_2N-[:330,,2](-[:270])-[:30](-[:330,,,1]NH_2)-[:90]}}\begin{reponses}
            \mauvaise{butan-2-amine}
            \mauvaise{butan-1-amine}
            \mauvaise{2-amino-3-éthylbutane}
            \bonne{2,3-diaminobutane}
        \end{reponses}
    \end{question}
}\element{nomenca}{
    \begin{question}{nomenca24}
        \needspace{6cm}Quel est la formule topologique de: 2-méthylbutan-1-ol~?
        \begin{reponses}
            \mauvaise{{\small\chemfig{OH-[:210,,1]-[:150]-[:210](-[:150])-[:270]}}}
            \mauvaise{{\small\chemfig{OH-[:150,,1](=[:90]O)-[:210](-[:270])-[:150]-[:210]}}}
            \mauvaise{{\small\chemfig{O=[:30]-[:330](-[:270])-[:30]-[:330]}}}
            \bonne{{\small\chemfig{OH-[:150,,1]-[:210](-[:270])-[:150]-[:210]}}}
        \end{reponses}
    \end{question}
}\element{nomenca}{
    \begin{question}{nomencar24}
        \needspace{6cm}Quel est le nom de cette molécule: ~? 
         \smallskip
         \\ 
         {\small\chemfig{OH-[:150,,1]-[:210](-[:270])-[:150]-[:210]}}\begin{reponses}
            \mauvaise{3-pentylbutan-1-ol}
            \mauvaise{2-méthylbutanal}
            \mauvaise{3-éthylbutan-1-ol}
            \bonne{2-méthylbutan-1-ol}
        \end{reponses}
    \end{question}
}\element{nomenca}{
    \begin{question}{nomenca25}
        \needspace{6cm}Quel est la formule topologique de: 3-méthylbutan-1-ol~?
        \begin{reponses}
            \mauvaise{{\small\chemfig{O=[:30]-[:330](-[:270])-[:30]-[:330]}}}
            \mauvaise{{\small\chemfig{OH-[:150,,1]-[:210](-[:270])-[:150]-[:210]}}}
            \mauvaise{{\small\chemfig{OH-[:150,,1](=[:90]O)-[:210](-[:270])-[:150]-[:210]}}}
            \bonne{{\small\chemfig{OH-[:210,,1]-[:150]-[:210](-[:150])-[:270]}}}
        \end{reponses}
    \end{question}
}\element{nomenca}{
    \begin{question}{nomencar25}
        \needspace{6cm}Quel est le nom de cette molécule: ~? 
         \smallskip
         \\ 
         {\small\chemfig{OH-[:210,,1]-[:150]-[:210](-[:150])-[:270]}}\begin{reponses}
            \mauvaise{2-méthylbutan-1-ol}
            \mauvaise{acide 2-méthylbutanoïque}
            \mauvaise{3-pentylbutan-1-ol}
            \bonne{3-méthylbutan-1-ol}
        \end{reponses}
    \end{question}
}\element{nomenca}{
    \begin{question}{nomenca26}
        \needspace{6cm}Quel est la formule topologique de: acide 2-méthylbutanoïque~?
        \begin{reponses}
            \mauvaise{{\small\chemfig{O=[:30]-[:330](-[:270])-[:30]-[:330]}}}
            \mauvaise{{\small\chemfig{OH-[:150,,1]-[:210](-[:270])-[:150]-[:210]}}}
            \mauvaise{{\small\chemfig{OH-[:210,,1]-[:150]-[:210](-[:150])-[:270]}}}
            \bonne{{\small\chemfig{OH-[:150,,1](=[:90]O)-[:210](-[:270])-[:150]-[:210]}}}
        \end{reponses}
    \end{question}
}\element{nomenca}{
    \begin{question}{nomencar26}
        \needspace{6cm}Quel est le nom de cette molécule: ~? 
         \smallskip
         \\ 
         {\small\chemfig{OH-[:150,,1](=[:90]O)-[:210](-[:270])-[:150]-[:210]}}\begin{reponses}
            \mauvaise{2-méthylbutanal}
            \mauvaise{3-méthylbutan-1-ol}
            \mauvaise{2-méthylbutan-1-ol}
            \bonne{acide 2-méthylbutanoïque}
        \end{reponses}
    \end{question}
}\element{nomenca}{
    \begin{question}{nomenca27}
        \needspace{6cm}Quel est la formule topologique de: 2-méthylbutanal~?
        \begin{reponses}
            \mauvaise{{\small\chemfig{OH-[:210,,1]-[:150]-[:210](-[:150])-[:270]}}}
            \mauvaise{{\small\chemfig{OH-[:150,,1](=[:90]O)-[:210](-[:270])-[:150]-[:210]}}}
            \mauvaise{{\small\chemfig{OH-[:150,,1]-[:210](-[:270])-[:150]-[:210]}}}
            \bonne{{\small\chemfig{O=[:30]-[:330](-[:270])-[:30]-[:330]}}}
        \end{reponses}
    \end{question}
}\element{nomenca}{
    \begin{question}{nomencar27}
        \needspace{6cm}Quel est le nom de cette molécule: ~? 
         \smallskip
         \\ 
         {\small\chemfig{O=[:30]-[:330](-[:270])-[:30]-[:330]}}\begin{reponses}
            \mauvaise{acide 2-méthylbutanoïque}
            \mauvaise{3-méthylbutan-1-ol}
            \mauvaise{3-pentylbutan-1-ol}
            \bonne{2-méthylbutanal}
        \end{reponses}
    \end{question}
}\element{nomenca}{
    \begin{question}{nomenca32}
        \needspace{6cm}Quel est la formule topologique de: 3-méthylheptane~?
        \begin{reponses}
            \mauvaise{{\small\chemfig{-[:30]-[:330](-[:270]-[:210])-[:30]-[:330]-[:30]-[:330]}}}
            \mauvaise{{\small\chemfig{-[:90](-[:150]-[:210])-[:30]-[:330]-[:30]}}}
            \mauvaise{{\small\chemfig{-[:90](-[:150]-[:210])-[:30]-[:330]-[:30]-[:330]-[:30]}}}
            \bonne{{\small\chemfig{-[:90](-[:150]-[:210])-[:30]-[:330]-[:30]-[:330]}}}
        \end{reponses}
    \end{question}
}\element{nomenca}{
    \begin{question}{nomencar32}
        \needspace{6cm}Quel est le nom de cette molécule: ~? 
         \smallskip
         \\ 
         {\small\chemfig{-[:90](-[:150]-[:210])-[:30]-[:330]-[:30]-[:330]}}\begin{reponses}
            \mauvaise{3-éthylheptane}
            \mauvaise{4-méthylheptane}
            \mauvaise{5-méthylheptane}
            \bonne{3-méthylheptane}
        \end{reponses}
    \end{question}
}\element{nomenca}{
    \begin{question}{nomenca33}
        \needspace{6cm}Quel est la formule topologique de: 4-méthylheptane~?
        \begin{reponses}
            \mauvaise{{\small\chemfig{-[:90](-[:150]-[:210])-[:30]-[:330]-[:30]-[:330]}}}
            \mauvaise{{\small\chemfig{-[:90](-[:150]-[:210])-[:30]-[:330]-[:30]}}}
            \mauvaise{{\small\chemfig{-[:30]-[:330](-[:270]-[:210])-[:30]-[:330]-[:30]-[:330]}}}
            \bonne{{\small\chemfig{-[:90](-[:30]-[:330]-[:30])-[:150]-[:210]-[:150]}}}
        \end{reponses}
    \end{question}
}\element{nomenca}{
    \begin{question}{nomencar33}
        \needspace{6cm}Quel est le nom de cette molécule: ~? 
         \smallskip
         \\ 
         {\small\chemfig{-[:90](-[:30]-[:330]-[:30])-[:150]-[:210]-[:150]}}\begin{reponses}
            \mauvaise{3-méthylheptane}
            \mauvaise{5-méthylheptane}
            \mauvaise{3-méthylhexane}
            \bonne{4-méthylheptane}
        \end{reponses}
    \end{question}
}\element{nomenca}{
    \begin{question}{nomenca35}
        \needspace{6cm}Quel est la formule topologique de: 3-éthylheptane~?
        \begin{reponses}
            \mauvaise{{\small\chemfig{-[:90](-[:150]-[:210])-[:30]-[:330]-[:30]}}}
            \mauvaise{{\small\chemfig{-[:90](-[:150]-[:210])-[:30]-[:330]-[:30]-[:330]}}}
            \mauvaise{{\small\chemfig{-[:90](-[:30]-[:330]-[:30])-[:150]-[:210]-[:150]}}}
            \bonne{{\small\chemfig{-[:30]-[:330](-[:270]-[:210])-[:30]-[:330]-[:30]-[:330]}}}
        \end{reponses}
    \end{question}
}\element{nomenca}{
    \begin{question}{nomencar35}
        \needspace{6cm}Quel est le nom de cette molécule: ~? 
         \smallskip
         \\ 
         {\small\chemfig{-[:30]-[:330](-[:270]-[:210])-[:30]-[:330]-[:30]-[:330]}}\begin{reponses}
            \mauvaise{3-méthyloctane}
            \mauvaise{3-méthylheptane}
            \mauvaise{3-méthylhexane}
            \bonne{3-éthylheptane}
        \end{reponses}
    \end{question}
}\element{nomenca}{
    \begin{question}{nomenca36}
        \needspace{6cm}Quel est la formule topologique de: 3-méthylhexane~?
        \begin{reponses}
            \mauvaise{{\small\chemfig{-[:90](-[:150]-[:210])-[:30]-[:330]-[:30]-[:330]}}}
            \mauvaise{{\small\chemfig{-[:90](-[:150]-[:210])-[:30]-[:330]-[:30]-[:330]-[:30]}}}
            \mauvaise{{\small\chemfig{-[:30]-[:330](-[:270]-[:210])-[:30]-[:330]-[:30]-[:330]}}}
            \bonne{{\small\chemfig{-[:90](-[:150]-[:210])-[:30]-[:330]-[:30]}}}
        \end{reponses}
    \end{question}
}\element{nomenca}{
    \begin{question}{nomencar36}
        \needspace{6cm}Quel est le nom de cette molécule: ~? 
         \smallskip
         \\ 
         {\small\chemfig{-[:90](-[:150]-[:210])-[:30]-[:330]-[:30]}}\begin{reponses}
            \mauvaise{3-éthylheptane}
            \mauvaise{5-méthylheptane}
            \mauvaise{4-méthylheptane}
            \bonne{3-méthylhexane}
        \end{reponses}
    \end{question}
}\element{nomenca}{
    \begin{question}{nomenca37}
        \needspace{6cm}Quel est la formule topologique de: 3-méthyloctane~?
        \begin{reponses}
            \mauvaise{{\small\chemfig{-[:90](-[:150]-[:210])-[:30]-[:330]-[:30]-[:330]}}}
            \mauvaise{{\small\chemfig{-[:90](-[:150]-[:210])-[:30]-[:330]-[:30]}}}
            \mauvaise{{\small\chemfig{-[:30]-[:330](-[:270]-[:210])-[:30]-[:330]-[:30]-[:330]}}}
            \bonne{{\small\chemfig{-[:90](-[:150]-[:210])-[:30]-[:330]-[:30]-[:330]-[:30]}}}
        \end{reponses}
    \end{question}
}\element{nomenca}{
    \begin{question}{nomencar37}
        \needspace{6cm}Quel est le nom de cette molécule: ~? 
         \smallskip
         \\ 
         {\small\chemfig{-[:90](-[:150]-[:210])-[:30]-[:330]-[:30]-[:330]-[:30]}}\begin{reponses}
            \mauvaise{4-méthylheptane}
            \mauvaise{3-éthylheptane}
            \mauvaise{5-méthylheptane}
            \bonne{3-méthyloctane}
        \end{reponses}
    \end{question}
}\element{nomenca}{
    \begin{question}{nomenca40}
        \needspace{6cm}Quel est la formule topologique de: 2,3-diméthylpentanal~?
        \begin{reponses}
            \mauvaise{{\small\chemfig{O=[:330]-[:30]-[:330](-[:270])-[:30](-[:330])-[:90]}}}
            \mauvaise{{\small\chemfig{O=[:30]-[:330](-[:270])-[:30]-[:330](-[:30])-[:270]}}}
            \mauvaise{{\small\chemfig{O=[:150]-[:210](-[:270])-[:150]-[:210](-[:270])-[:150]-[:210]}}}
            \bonne{{\small\chemfig{O=[:210]-[:150](-[:90])-[:210](-[:270])-[:150]-[:210]}}}
        \end{reponses}
    \end{question}
}\element{nomenca}{
    \begin{question}{nomencar40}
        \needspace{6cm}Quel est le nom de cette molécule: ~? 
         \smallskip
         \\ 
         {\small\chemfig{O=[:210]-[:150](-[:90])-[:210](-[:270])-[:150]-[:210]}}\begin{reponses}
            \mauvaise{2,4-diméthylpentanal}
            \mauvaise{3,4-diméthylpentanal}
            \mauvaise{2-methyl-3-ethylpentanal}
            \bonne{2,3-diméthylpentanal}
        \end{reponses}
    \end{question}
}\element{nomenca}{
    \begin{question}{nomenca41}
        \needspace{6cm}Quel est la formule topologique de: 2,4-diméthylpentanal~?
        \begin{reponses}
            \mauvaise{{\small\chemfig{O=[:330]-[:30]-[:330](-[:270])-[:30](-[:330])-[:90]}}}
            \mauvaise{{\small\chemfig{O=[:150]-[:210](-[:270])-[:150]-[:210](-[:270])-[:150]-[:210]}}}
            \mauvaise{{\small\chemfig{O=[:210]-[:150](-[:90])-[:210](-[:270])-[:150]-[:210]}}}
            \bonne{{\small\chemfig{O=[:30]-[:330](-[:270])-[:30]-[:330](-[:30])-[:270]}}}
        \end{reponses}
    \end{question}
}\element{nomenca}{
    \begin{question}{nomencar41}
        \needspace{6cm}Quel est le nom de cette molécule: ~? 
         \smallskip
         \\ 
         {\small\chemfig{O=[:30]-[:330](-[:270])-[:30]-[:330](-[:30])-[:270]}}\begin{reponses}
            \mauvaise{2-methyl-3-ethylpentanal}
            \mauvaise{2,3-trimethylpentanal}
            \mauvaise{3,4-diméthylpentanal}
            \bonne{2,4-diméthylpentanal}
        \end{reponses}
    \end{question}
}\element{nomenca}{
    \begin{question}{nomenca42}
        \needspace{6cm}Quel est la formule topologique de: 2,4-diméthylhexanal~?
        \begin{reponses}
            \mauvaise{{\small\chemfig{O=[:330]-[:30]-[:330](-[:270])-[:30](-[:330])-[:90]}}}
            \mauvaise{{\small\chemfig{O=[:210]-[:150](-[:90])-[:210](-[:270])-[:150]-[:210]}}}
            \mauvaise{{\small\chemfig{O=[:30]-[:330](-[:270])-[:30]-[:330](-[:30])-[:270]}}}
            \bonne{{\small\chemfig{O=[:150]-[:210](-[:270])-[:150]-[:210](-[:270])-[:150]-[:210]}}}
        \end{reponses}
    \end{question}
}\element{nomenca}{
    \begin{question}{nomencar42}
        \needspace{6cm}Quel est le nom de cette molécule: ~? 
         \smallskip
         \\ 
         {\small\chemfig{O=[:150]-[:210](-[:270])-[:150]-[:210](-[:270])-[:150]-[:210]}}\begin{reponses}
            \mauvaise{2-methyl-3-ethylpentanal}
            \mauvaise{2,3-trimethylpentanal}
            \mauvaise{2,4-diméthylpentanal}
            \bonne{2,4-diméthylhexanal}
        \end{reponses}
    \end{question}
}\element{nomenca}{
    \begin{question}{nomenca43}
        \needspace{6cm}Quel est la formule topologique de: 3,4-diméthylpentanal~?
        \begin{reponses}
            \mauvaise{{\small\chemfig{O=[:150]-[:210](-[:270])-[:150]-[:210](-[:270])-[:150]-[:210]}}}
            \mauvaise{{\small\chemfig{O=[:210]-[:150](-[:90])-[:210](-[:270])-[:150]-[:210]}}}
            \mauvaise{{\small\chemfig{O=[:30]-[:330](-[:270])-[:30]-[:330](-[:30])-[:270]}}}
            \bonne{{\small\chemfig{O=[:330]-[:30]-[:330](-[:270])-[:30](-[:330])-[:90]}}}
        \end{reponses}
    \end{question}
}\element{nomenca}{
    \begin{question}{nomencar43}
        \needspace{6cm}Quel est le nom de cette molécule: ~? 
         \smallskip
         \\ 
         {\small\chemfig{O=[:330]-[:30]-[:330](-[:270])-[:30](-[:330])-[:90]}}\begin{reponses}
            \mauvaise{2,4-diméthylpentanal}
            \mauvaise{2,3-trimethylpentanal}
            \mauvaise{2,3-diméthylpentanal}
            \bonne{3,4-diméthylpentanal}
        \end{reponses}
    \end{question}
}\element{nomenca}{
    \begin{question}{nomenca48}
        \needspace{6cm}Quel est la formule topologique de: 3-éthyl-4-méthylhexan-2-one~?
        \begin{reponses}
            \mauvaise{{\small\chemfig{O=[:270](-[:210])-[:330](-[:30](-[:90])-[:330])-[:270]-[:210]}}}
            \mauvaise{{\small\chemfig{O=[:270](-[:210])-[:330](-[:270])-[:30](-[:330]-[:30])-[:90]-[:30]}}}
            \bonne{{\small\chemfig{O=[:270](-[:210])-[:330](-[:270]-[:210])-[:30](-[:90])-[:330]-[:30]}}}
        \end{reponses}
    \end{question}
}\element{nomenca}{
    \begin{question}{nomencar48}
        \needspace{6cm}Quel est le nom de cette molécule: ~? 
         \smallskip
         \\ 
         {\small\chemfig{O=[:270](-[:210])-[:330](-[:270]-[:210])-[:30](-[:90])-[:330]-[:30]}}\begin{reponses}
            \mauvaise{2,3-diméthyl-4-éthylhexan-5-one}
            \mauvaise{4-éthyl-3-méthylhexan-2-one}
            \mauvaise{3-propyl-4-methylhexan-2-one}
            \bonne{3-éthyl-4-méthylhexan-2-one}
        \end{reponses}
    \end{question}
}\element{nomenca}{
    \begin{question}{nomenca49}
        \needspace{6cm}Quel est la formule topologique de: 4-éthyl-3-méthylhexan-2-one~?
        \begin{reponses}
            \mauvaise{{\small\chemfig{O=[:270](-[:210])-[:330](-[:30](-[:90])-[:330])-[:270]-[:210]}}}
            \mauvaise{{\small\chemfig{O=[:270](-[:210])-[:330](-[:270]-[:210])-[:30](-[:90])-[:330]-[:30]}}}
            \bonne{{\small\chemfig{O=[:270](-[:210])-[:330](-[:270])-[:30](-[:330]-[:30])-[:90]-[:30]}}}
        \end{reponses}
    \end{question}
}\element{nomenca}{
    \begin{question}{nomencar49}
        \needspace{6cm}Quel est le nom de cette molécule: ~? 
         \smallskip
         \\ 
         {\small\chemfig{O=[:270](-[:210])-[:330](-[:270])-[:30](-[:330]-[:30])-[:90]-[:30]}}\begin{reponses}
            \mauvaise{2,3-diméthyl-4-éthylhexan-5-one}
            \mauvaise{3-propyl-4-methylhexan-2-one}
            \mauvaise{2-éthyl-5-methylheptan-2-one}
            \bonne{4-éthyl-3-méthylhexan-2-one}
        \end{reponses}
    \end{question}
}\element{nomenca}{
    \begin{question}{nomenca52}
        \needspace{6cm}Quel est la formule topologique de: 3-éthyl-4-méthylpentan-2-one~?
        \begin{reponses}
            \mauvaise{{\small\chemfig{O=[:270](-[:210])-[:330](-[:270]-[:210])-[:30](-[:90])-[:330]-[:30]}}}
            \mauvaise{{\small\chemfig{O=[:270](-[:210])-[:330](-[:270])-[:30](-[:330]-[:30])-[:90]-[:30]}}}
            \bonne{{\small\chemfig{O=[:270](-[:210])-[:330](-[:30](-[:90])-[:330])-[:270]-[:210]}}}
        \end{reponses}
    \end{question}
}\element{nomenca}{
    \begin{question}{nomencar52}
        \needspace{6cm}Quel est le nom de cette molécule: ~? 
         \smallskip
         \\ 
         {\small\chemfig{O=[:270](-[:210])-[:330](-[:30](-[:90])-[:330])-[:270]-[:210]}}\begin{reponses}
            \mauvaise{2-éthyl-5-methylheptan-2-one}
            \mauvaise{3-éthyl-4-méthylhexan-2-one}
            \mauvaise{4-éthyl-3-méthylhexan-2-one}
            \bonne{3-éthyl-4-méthylpentan-2-one}
        \end{reponses}
    \end{question}
}\element{nomenca}{
    \begin{question}{nomenca56}
        \needspace{6cm}Quel est la formule topologique de: acide 2-méthylpropanoïque~?
        \begin{reponses}
            \mauvaise{{\small\chemfig{OH-[:150,,1]-[:210](-[:150])-[:270]}}}
            \mauvaise{{\small\chemfig{OH-[:150,,1](=[:90]O)-[:210](-[:270])-[:150]-[:210](-[:150])-[:270]}}}
            \mauvaise{{\small\chemfig{HO-[:30,,2](=[:90]O)-[:330](-[:270]-[:210])-[:30](-[:90]-[:30])-[:330]%
            -[:30]-[:330]}}}
            \bonne{{\small\chemfig{OH-[:150,,1](=[:90]O)-[:210](-[:150])-[:270]}}}
        \end{reponses}
    \end{question}
}\element{nomenca}{
    \begin{question}{nomencar56}
        \needspace{6cm}Quel est le nom de cette molécule: ~? 
         \smallskip
         \\ 
         {\small\chemfig{OH-[:150,,1](=[:90]O)-[:210](-[:150])-[:270]}}\begin{reponses}
            \mauvaise{2-méthylpropan-1-ol}
            \mauvaise{acide 2,4-diméthylpentanoïque}
            \mauvaise{2-méthylpropanal}
            \bonne{acide 2-méthylpropanoïque}
        \end{reponses}
    \end{question}
}\element{nomenca}{
    \begin{question}{nomenca57}
        \needspace{6cm}Quel est la formule topologique de: 2-méthylpropan-1-ol~?
        \begin{reponses}
            \mauvaise{{\small\chemfig{OH-[:150,,1](=[:90]O)-[:210](-[:150])-[:270]}}}
            \mauvaise{{\small\chemfig{OH-[:150,,1](=[:90]O)-[:210](-[:270])-[:150]-[:210](-[:150])-[:270]}}}
            \mauvaise{{\small\chemfig{O=[:150]-[:210](-[:150])-[:270]}}}
            \bonne{{\small\chemfig{OH-[:150,,1]-[:210](-[:150])-[:270]}}}
        \end{reponses}
    \end{question}
}\element{nomenca}{
    \begin{question}{nomencar57}
        \needspace{6cm}Quel est le nom de cette molécule: ~? 
         \smallskip
         \\ 
         {\small\chemfig{OH-[:150,,1]-[:210](-[:150])-[:270]}}\begin{reponses}
            \mauvaise{acide 2-méthylpropanoïque}
            \mauvaise{acide 2-méthylbutanoïque}
            \mauvaise{acide 2,4-diméthylpentanoïque}
            \bonne{2-méthylpropan-1-ol}
        \end{reponses}
    \end{question}
}\element{nomenca}{
    \begin{question}{nomenca58}
        \needspace{6cm}Quel est la formule topologique de: acide 2,4-diméthylpentanoïque~?
        \begin{reponses}
            \mauvaise{{\small\chemfig{O=[:150]-[:210](-[:150])-[:270]}}}
            \mauvaise{{\small\chemfig{OH-[:150,,1]-[:210](-[:150])-[:270]}}}
            \mauvaise{{\small\chemfig{OH-[:150,,1](=[:90]O)-[:210](-[:150])-[:270]}}}
            \bonne{{\small\chemfig{OH-[:150,,1](=[:90]O)-[:210](-[:270])-[:150]-[:210](-[:150])-[:270]}}}
        \end{reponses}
    \end{question}
}\element{nomenca}{
    \begin{question}{nomencar58}
        \needspace{6cm}Quel est le nom de cette molécule: ~? 
         \smallskip
         \\ 
         {\small\chemfig{OH-[:150,,1](=[:90]O)-[:210](-[:270])-[:150]-[:210](-[:150])-[:270]}}\begin{reponses}
            \mauvaise{acide 2-méthylpropanoïque}
            \mauvaise{acide 2,3-diéthylhexanoïque}
            \mauvaise{2-méthylpropanal}
            \bonne{acide 2,4-diméthylpentanoïque}
        \end{reponses}
    \end{question}
}\element{nomenca}{
    \begin{question}{nomenca59}
        \needspace{6cm}Quel est la formule topologique de: 2-méthylpropanal~?
        \begin{reponses}
            \mauvaise{{\small\chemfig{OH-[:150,,1](=[:90]O)-[:210](-[:270])-[:150]-[:210]}}}
            \mauvaise{{\small\chemfig{OH-[:150,,1]-[:210](-[:150])-[:270]}}}
            \mauvaise{{\small\chemfig{OH-[:150,,1](=[:90]O)-[:210](-[:270])-[:150]-[:210](-[:150])-[:270]}}}
            \bonne{{\small\chemfig{O=[:150]-[:210](-[:150])-[:270]}}}
        \end{reponses}
    \end{question}
}\element{nomenca}{
    \begin{question}{nomencar59}
        \needspace{6cm}Quel est le nom de cette molécule: ~? 
         \smallskip
         \\ 
         {\small\chemfig{O=[:150]-[:210](-[:150])-[:270]}}\begin{reponses}
            \mauvaise{acide 2,3-diéthylhexanoïque}
            \mauvaise{2-méthylpropan-1-ol}
            \mauvaise{acide 2-méthylbutanoïque}
            \bonne{2-méthylpropanal}
        \end{reponses}
    \end{question}
}\element{nomenca}{
    \begin{question}{nomenca60}
        \needspace{6cm}Quel est la formule topologique de: acide 2-méthylbutanoïque~?
        \begin{reponses}
            \mauvaise{{\small\chemfig{HO-[:30,,2](=[:90]O)-[:330](-[:270]-[:210])-[:30](-[:90]-[:30])-[:330]%
            -[:30]-[:330]}}}
            \mauvaise{{\small\chemfig{OH-[:150,,1]-[:210](-[:150])-[:270]}}}
            \mauvaise{{\small\chemfig{OH-[:150,,1](=[:90]O)-[:210](-[:270])-[:150]-[:210](-[:150])-[:270]}}}
            \bonne{{\small\chemfig{OH-[:150,,1](=[:90]O)-[:210](-[:270])-[:150]-[:210]}}}
        \end{reponses}
    \end{question}
}\element{nomenca}{
    \begin{question}{nomencar60}
        \needspace{6cm}Quel est le nom de cette molécule: ~? 
         \smallskip
         \\ 
         {\small\chemfig{OH-[:150,,1](=[:90]O)-[:210](-[:270])-[:150]-[:210]}}\begin{reponses}
            \mauvaise{2-méthylpropanal}
            \mauvaise{acide 2,3-diéthylhexanoïque}
            \mauvaise{acide 2-méthylpropanoïque}
            \bonne{acide 2-méthylbutanoïque}
        \end{reponses}
    \end{question}
}\element{nomenca}{
    \begin{question}{nomenca61}
        \needspace{6cm}Quel est la formule topologique de: acide 2,3-diéthylhexanoïque~?
        \begin{reponses}
            \mauvaise{{\small\chemfig{OH-[:150,,1]-[:210](-[:150])-[:270]}}}
            \mauvaise{{\small\chemfig{O=[:150]-[:210](-[:150])-[:270]}}}
            \mauvaise{{\small\chemfig{OH-[:150,,1](=[:90]O)-[:210](-[:270])-[:150]-[:210]}}}
            \bonne{{\small\chemfig{HO-[:30,,2](=[:90]O)-[:330](-[:270]-[:210])-[:30](-[:90]-[:30])-[:330]%
            -[:30]-[:330]}}}
        \end{reponses}
    \end{question}
}\element{nomenca}{
    \begin{question}{nomencar61}
        \needspace{6cm}Quel est le nom de cette molécule: ~? 
         \smallskip
         \\ 
         {\small\chemfig{HO-[:30,,2](=[:90]O)-[:330](-[:270]-[:210])-[:30](-[:90]-[:30])-[:330]%
        -[:30]-[:330]}}\begin{reponses}
            \mauvaise{acide 2,4-diméthylpentanoïque}
            \mauvaise{2-méthylpropanal}
            \mauvaise{acide 2-méthylbutanoïque}
            \bonne{acide 2,3-diéthylhexanoïque}
        \end{reponses}
    \end{question}
}