\element{tampons}{
    \begin{question}{tampons1}
        Une solution tampon est composée~:
        \begin{reponses}
            \mauvaise{D'un acide fort et d'une base forte}
            \mauvaise{D'un sel et d'un solvant}
            \mauvaise{D'une base faible et d'un acide fort}
            \bonne{D'un acide faible et de sa base conjuguée}
        \end{reponses}
        \explain{Une solution tampon est composée d'un acide faible et de sa base conjuguée, ce qui permet de résister aux variations de pH.}
    \end{question}
}

\element{tampons}{
    \begin{question}{tampons2}
        La fonction principale d'une solution tampon est~:
        \begin{reponses}
            \mauvaise{Changer radicalement le pH}
            \mauvaise{Neutraliser les acides forts}
            \mauvaise{Augmenter la température}
            \bonne{Maintenir le pH constant}
        \end{reponses}
        \explain{La fonction principale d'une solution tampon est de maintenir le pH constant malgré l'ajout de petites quantités d'acide ou de base.}
    \end{question}
}

\element{tampons}{
    \begin{question}{tampons3}
        Une solution tampon est efficace lorsque~:
        \begin{reponses}
            \mauvaise{Le pH est très acide}
            \mauvaise{Le pH est très basique}
            \mauvaise{Les concentrations de l'acide et de la base sont très différentes}
            \bonne{Le pH est proche du pKa du couple acide/base}
        \end{reponses}
        \explain{Une solution tampon est efficace lorsque le pH est proche du pKa du couple acide/base, car c'est dans cette zone que le pouvoir tampon est maximal.}
    \end{question}
}

\element{tampons}{
    \begin{question}{tampons4}
        Un exemple de solution tampon est~:
        \begin{reponses}
            \mauvaise{$\ce{HCl}$ et $\ce{NaOH}$}
            \mauvaise{$\ce{NaCl}$ et $\ce{H2O}$}
            \mauvaise{$\ce{H2SO4}$ et $\ce{KOH}$}
            \bonne{$\ce{CH3COOH}$ et $\ce{CH3COONa}$}
        \end{reponses}
        \explain{Un exemple classique de solution tampon est le mélange d'acide acétique ($\ce{CH3COOH}$) et d'acétate de sodium ($\ce{CH3COONa}$), car il s'agit d'un acide faible et de sa base conjuguée.}
    \end{question}
}

\element{tampons}{
    \begin{question}{tampons5}
        La capacité tampon dépend de~:
        \begin{reponses}
            \mauvaise{La température uniquement}
            \mauvaise{La couleur de la solution}
            \mauvaise{La pression atmosphérique}
            \bonne{La concentration de l'acide faible et de sa base conjuguée}
        \end{reponses}
        \explain{La capacité tampon dépend principalement de la concentration de l'acide faible et de sa base conjuguée, car c'est leur rapport qui détermine l'efficacité du tampon.}
    \end{question}
}

\element{tampons}{
    \begin{question}{tampons6}
        Si on ajoute une petite quantité d'acide à une solution tampon, le pH~:
        \begin{reponses}
            \mauvaise{Augmente fortement}
            \mauvaise{Diminue fortement}
            \mauvaise{Reste inchangé}
            \bonne{Varie très légèrement}
        \end{reponses}
        \explain{Si on ajoute une petite quantité d'acide à une solution tampon, le pH varie très légèrement grâce à l'équilibre entre l'acide faible et sa base conjuguée qui absorbe les variations.}
    \end{question}
}

\element{tampons}{
    \begin{question}{tampons7}
        Une solution tampon est utilisée pour~:
        \begin{reponses}
            \mauvaise{Chauffer rapidement une solution}
            \mauvaise{Précipiter les sels}
            \mauvaise{Changer la couleur d'une solution}
            \bonne{Stabiliser le pH d'une solution}
        \end{reponses}
        \explain{Une solution tampon est utilisée pour stabiliser le pH d'une solution, en limitant les variations de pH lors de l'ajout de petites quantités d'acide ou de base.}
    \end{question}
}

\element{tampons}{
    \begin{question}{tampons8}
        La zone de tamponnement efficace est généralement~:
        \begin{reponses}
            \mauvaise{Entre pH 0 et pH 2}
            \mauvaise{Entre pH 12 et pH 14}
            \mauvaise{À pH 7 uniquement}
            \bonne{À $\pm 1$ unité de pH autour du pKa}
        \end{reponses}
        \explain{La zone de tamponnement efficace est généralement à $\pm 1$ unité de pH autour du pKa, car c'est dans cette plage que le pouvoir tampon est optimal.}
    \end{question}
}
