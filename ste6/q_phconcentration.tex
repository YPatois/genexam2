\element{phconcentration}{
    \begin{question}{phconcentration1}
        La formule du pH est~:
        \begin{reponses}
            \mauvaise{$\ce{pH = log[H3O+]}$}
            \mauvaise{$\ce{pH = [H3O+]}$}
            \mauvaise{$\ce{pH = -[H3O+]}$}
            \bonne{$\ce{pH = -log[H3O+]}$}
        \end{reponses}
        \explain{La formule correcte du pH est $\ce{pH = -log[H3O+]}$ car le pH est défini comme l'opposé du logarithme de la concentration en ions hydronium.}
    \end{question}
}

\element{phconcentration}{
    \begin{question}{phconcentration2}
        Si $\ce{[H3O+] = \qty{1.0e-3}{\mole\per\litre}}$, alors le pH est~:
        \begin{reponses}
            \mauvaise{11}
            \mauvaise{7}
            \mauvaise{1}
            \bonne{3}
        \end{reponses}
        \explain{Si $\ce{[H3O+] = \qty{1.0e-3}{\mole\per\litre}}$, alors $\ce{pH = -log(1.0e-3) = 3}$.}
    \end{question}
}

\element{phconcentration}{
    \begin{question}{phconcentration3}
        Si le pH d'une solution est 5, alors $\ce{[H3O+]}$ est~:
        \begin{reponses}
            \mauvaise{$\qty{1.0e-7}{\mole\per\litre}$}
            \mauvaise{$\qty{1.0e-9}{\mole\per\litre}$}
            \mauvaise{$\qty{5.0}{\mole\per\litre}$}
            \bonne{$\qty{1.0e-5}{\mole\per\litre}$}
        \end{reponses}
        \explain{Si le pH est 5, alors $\ce{[H3O+] = 10^{-pH} = 10^{-5} = \qty{1.0e-5}{\mole\per\litre}}$.}
    \end{question}
}

\element{phconcentration}{
    \begin{question}{phconcentration4}
        Une solution de pH 9 est~:
        \begin{reponses}
            \mauvaise{Neutre}
            \mauvaise{Acide}
            \mauvaise{Très acide}
            \bonne{Basique}
        \end{reponses}
        \explain{Une solution de pH 9 est basique car un pH supérieur à 7 indique une solution basique.}
    \end{question}
}

\element{phconcentration}{
    \begin{question}{phconcentration5}
        Le pH d'une solution neutre à $\qty{25}{\celsius}$ est~:
        \begin{reponses}
            \mauvaise{0}
            \mauvaise{14}
            \mauvaise{1}
            \bonne{7}
        \end{reponses}
        \explain{À $\qty{25}{\celsius}$, une solution neutre a un pH de 7 car les concentrations en ions $\ce{H3O+}$ et $\ce{OH-}$ sont égales.}
    \end{question}
}

\element{phconcentration}{
    \begin{question}{phconcentration6}
        Si $\ce{[H3O+]}$ augmente, le pH~:
        \begin{reponses}
            \mauvaise{Augmente}
            \mauvaise{Reste constant}
            \mauvaise{Devient nul}
            \bonne{Diminue}
        \end{reponses}
        \explain{Si $\ce{[H3O+]}$ augmente, le pH diminue car le pH est inversement proportionnel à la concentration en ions $\ce{H3O+}$.}
    \end{question}
}

\element{phconcentration}{
    \begin{question}{phconcentration7}
        Le produit ionique de l'eau $K_e$ à $\qty{25}{\celsius}$ est~:
        \begin{reponses}
            \mauvaise{$1.0 \times 10^{-7}$}
            \mauvaise{$1.0 \times 10^{-12}$}
            \mauvaise{$1.0 \times 10^{-2}$}
            \bonne{$1.0 \times 10^{-14}$}
        \end{reponses}
        \explain{À $\qty{25}{\celsius}$, le produit ionique de l'eau $K_e$ est $1.0 \times 10^{-14}$ car $K_e = \ce{[H3O+][OH-]}$ et vaut $10^{-14}$ à cette température.}
    \end{question}
}

\element{phconcentration}{
    \begin{question}{phconcentration8}
        Si $\ce{[OH-] = \qty{1.0e-5}{\mole\per\litre}}$, alors le pH est~:
        \begin{reponses}
            \mauvaise{5}
            \mauvaise{7}
            \mauvaise{14}
            \bonne{9}
        \end{reponses}
        \explain{Si $\ce{[OH-] = \qty{1.0e-5}{\mole\per\litre}}$, alors $\ce{[H3O+] = \frac{K_e}{[OH-]} = \frac{10^{-14}}{10^{-5}} = 10^{-9}}$ et $\ce{pH = -log(10^{-9}) = 9}$.}
    \end{question}
}
