%%AMC:latex_engine=pdflatex --shell-escape
\documentclass[12pt,a4paper]{article}
\usepackage{needspace}
\usepackage[nobottomtitles]{titlesec}
\usepackage{multicol}
\usepackage{xcolor}
\usepackage{fp}
\usepackage{xfp}
\usepackage{WriteOnGrid}
\usepackage{csvsimple}%
\usepackage[francais,bloc]{automultiplechoice}

\FPseed=11

\baremeDefautS{e=0,v=0,b=1,m=-.25}
\baremeDefautM{e=0,v=0,b=.25,m=-.25}

\graphicspath{ {./images/} }

\newenvironment{reponsesd}{
    \begin{multicols}{2}
    \begin{reponses} }{
    \end{reponses}
    \end{multicols}
}

\setlength{\columnseprule}{1pt}
\def\columnseprulecolor{\color{lightgray}}%

\let\oexplain\explain
\renewcommand{\explain}[1]{\oexplain{\textcolor{red}{#1}}}

\titleformat{\section}
  {\centering\hrule\vspace{2mm}}
  {\thesection}
  {1em}
  {}
  [\vspace{1mm}\hrule]

\titleformat{\subsection}
  {\em}
  {\thesubsection}
  {1em}
  {}

\newcommand{\sujet}{
\exemplaire{1}{%

\AMCsetFoot{\niveau{}°\classe{} -- \prenom{}~\nom{} -- \thepage}

%%% debut de l'en-tête des copies :
\begin{center}
\vfill
\noindent{\large \bf Classe de \niveau{}°\classe{}}

\vspace*{3mm}

{\Large\bf Évaluation sur table - Énergie 03/06/05/2024}

\vfill
\namefield{\noindent{}\fbox{\vspace*{3mm}
         \Huge\bf\prenom{}~\nom{}\normalsize{}%
         \vspace*{3mm}
      }}
\end{center}
\vfill

\begin{center}
\textbf{Durée : 30 minutes.}
\vspace*{5mm}

  Aucun document n'est autorisé.
  L'usage de la calculatrice est interdit.

  Les questions faisant apparaitre le symbole \multiSymbole{} peuvent
  présenter une ou plusieurs bonnes réponses. Les autres ont
  une unique bonne réponse.

  Des points négatifs seront affectés aux mauvaises réponses.
  \vspace*{5mm}

  \textbf{IMPORTANT: Utilisez un crayon de papier bien noir pour cocher les cases, et une gomme pour effacer délicatement en cas d'erreur. Ne raturez pas les cases.}

  \vspace*{5mm}

  Ne pas faire comme ceci (pas centré, trop pâle, raturé):\\
  \includegraphics[width=4cm]{checkbox_bad.png}

  Mais comme cela (bien centré, bien foncé):\\
  \includegraphics[width=2cm]{checkbox_good.png}


\end{center}
\vspace{1ex}
\vfill
\pagebreak
%%% fin de l'en-tête

\restituegroupe{groupes}


\AMCassociation{\id}

	  } % End \exemplaire{1}{%
} % End \newcommand{\sujet}{


%%%%§§§§§§§§§§§§§§§§§§§§§§§§§§§§§§§§§
\newcommand{\allq}{--all}

\begin{document}
%%%Options
\AMCrandomseed{10}

\def\AMCformQuestion#1{{\sc Question #1 :}}

\setdefaultgroupmode{withoutreplacement}
%%% Fin Options

%%% elements

\element{generalites}{
  \begin{questionmult}{gen1}
    Les humains ont besoin d'énergie pour~:
    \begin{reponses}
      \bonne{Cuisiner}
      \bonne{Se déplacer}
      \bonne{S'éclairer}
      \mauvaise{Les humains n'ont pas besoin d'énergie}
    \end{reponses}
    \explain{Toute action demande de l'énergie.}
  \end{questionmult}
}

\element{generalites}{
  \begin{questionmult}{gen2}
    Les humains ont besoin d'énergie pour~:
    \begin{reponses}
      \bonne{Se nourrir}
      \bonne{Se chauffer}
      \bonne{Voyager}
      \mauvaise{Les humains n'ont pas besoin d'énergie}
    \end{reponses}
    \explain{Toute action demande de l'énergie.}
  \end{questionmult}
}

\element{sources}{
  \begin{questionmult}{source1}
    Les énergies exploités par l'humanité~:
    \begin{reponses}
      \bonne{Se trouvent dans des sources d'énergies}
      \bonne{Se chauffer}\bareme{d=.25,b=0}
      \bonne{Voyager}\bareme{d=.25,b=0}
      \mauvaise{Surgissent spontanément}
    \end{reponses}
    \explain{erreur d'énoncé ici, les deux réponses \emph{Se chauffer} et \emph{Voyager} n'ont pas de sens ici. On trouve de l'énergie exploitable par l'humanité dans des sources d'énergie.}
  \end{questionmult}
}

\element{sources}{
  \begin{questionmult}{source2}
    Une source d'énergie, c'est~:
    \begin{reponses}
      \bonne{Un phénomène naturel}
      \bonne{Exploitée par les humains pour en tirer de l'énergie}
      \mauvaise{Une passion qui anime l'humanité}
      \mauvaise{Une eau vivifiante qui jaillit d'une montagne}
    \end{reponses}
  \end{questionmult}
}

\element{carbone}{
  \begin{questionmult}{carb1}
    Coche les sources d'énergie carbonées~:
    \begin{reponses}
      \FPeval\ySeed{trunc(random*1000,0)}
      \input{"|./genexo.py --seed \ySeed~ --mode CARB1 --level 0 --qref toto \allq"}
    \end{reponses}
    \explain{Ici l'énoncé était imprécis: on aurait dû lire \emph{Les sources d'énergie carbonnées \bf{fossiles}}} 
  \end{questionmult}
}

\element{carbone}{
  \begin{questionmult}{carb2}
    Les sources d'énergie carbonées~:
    \begin{reponses}
      \FPeval\ySeed{trunc(random*1000,0)}
      \input{"|./genexo.py --seed \ySeed~ --mode CARB2 --level 0 --qref toto \allq"}
    \end{reponses}
    \explain{Ici l'énoncé était imprécis: on aurait dû lire \emph{Les sources d'énergie carbonnées \bf{fossiles}}}
  \end{questionmult}
}

\FPeval\ySeed{trunc(random*1000,0)}
\input{"|./genexo.py --seed \ySeed~ --mode FNR --level 0 --qref toto \allq"}

\FPeval\ySeed{trunc(random*1000,0)}
\input{"|./genexo.py --seed \ySeed~ --mode FSNR --level 0 --qref toto \allq"}


\element{enrn}{
  \begin{questionmult}{nrrn}
    Coche les sources d'énergie renouvelables dans la liste~:
    \begin{reponses}
      \FPeval\ySeed{trunc(random*1000,0)}
      \input{"|./genexo.py --seed \ySeed~ --mode FRNR --level 0 --qref toto \allq"}
    \end{reponses}
    \explain{Les énergies renouvelables sont celles qui soit ne s'épuiseront pas à l'échelle de l'humanité (comme le Soleil) soit se renouvellent à notre échelle (comme le bois). Le charbon ou le pétrole, par exemple mettent des dizaines de millions d'années à se former, ce n'est pas renouvelable.}
  \end{questionmult}
}


%%%% fin des elements

\element{groupes}{
\section{Généralités}
\restituegroupe[1]{generalites}

\section{Énergies carbonées}
\begin{multicols}{2}
\restituegroupe{carbone}
\end{multicols}

\section{Sources et formes d'énergies}
\begin{multicols}{2}
\restituegroupe[8]{autofnr}
\restituegroupe[8]{autofsnr}
\restituegroupe{enrn}
\end{multicols}

\section{Pour aller plus loin}
\emph{Ce problème est plus difficile et rapporte moins de points, ne le faire que lorsque le reste est terminé, c'est un bonus!}

\subsection{En quoi les énergies carbonées sont-elles un problème pour l'avenir de l'humanité~?}
\begin{EnvQuadrillage}[NbCarreaux=21x6,Grille=Seyes,Marge=1]
\EcrireLigne{L'utilisation des énergies carbonées produit du $CO_2$, un gaz à effet de serre qui modifie gravement le climat de la Terre. Ce sont de plus des produits polluants pour l'environnement et les humains, à l'extraction, à l'utilisation, et en rebut. Étant en quantité limitée sur Terre, elles ne sont pas renouvelables.}
\end{EnvQuadrillage}

\subsection{Quels problèmes résout l'énergie nucléaire parmi ceux soulevés plus haut~? Lesquels ne résout-elle pas~?}
\begin{EnvQuadrillage}[NbCarreaux=21x6,Grille=Seyes,Marge=1]
\EcrireLigne{L'énergie nucléaire ne produit que très peu de gaz à effet de serre. Elle peut cependant causer des pollutions lors de l'extraction minière et la gestion de ses déchets est complexe. L'uranium utilisé dans les centrales n'est de plus pas renouvelable.}
\end{EnvQuadrillage}

\subsection{En quoi les énergies renouvelables résolvent'elles ces problèmes~? Quels en sont cependant les inconvénients~?}
\begin{EnvQuadrillage}[NbCarreaux=21x6,Grille=Seyes,Marge=1]
\EcrireLigne{Les énergies renouvelables ne s'épuiseront pas (si correctement tgérées); elles peuvent toutefois avoir des effets négatifs sur l'environnement (les machines électriques consomment des matériaux rares, les barrages hydroélectriques bioulversent les écosystèmes, etc.), et surtout ne sont aujourd'hui pas en mesure de fournir la quantité d'énergie dont l'humanité a besoin actuellement.}
\end{EnvQuadrillage}


\AMCcleardoublepage
}

\csvreader[head to column names]{liste.csv}{}{\sujet}

\end{document}
