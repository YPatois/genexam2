\documentclass[12pt,a4paper]{article}
\usepackage{titlesec}
\usepackage{circuitikz}
\usepackage{siunitx}

\usepackage{csvsimple}%

\usepackage[francais,bloc]{automultiplechoice}

\titleformat{\section}
  {\centering\hrule\vspace{2mm}}
  {\thesection}
  {1em}
  {}
  [\vspace{1mm}\hrule]

%\begin{center}
%\hrule\vspace{2mm} {\small{ \bf Titre}\\}
%\vspace{1mm}\hrule
%\end{center}~

\newcommand{\sujet}{
\exemplaire{1}{%

\AMCsetFoot{\niveau{}°\classe{} -- \prenom{}~\nom{} -- \thepage}



%%% debut de l'en-tête des copies :  
\begin{center}
\vfill
\noindent{\large \bf Classe de \niveau{}°\classe{}}

\vspace*{3mm}

{\Large\bf Évaluation sur table - Électricité 27/05/2024}

\vfill
\noindent{}\fbox{\vspace*{3mm}
         \Huge\bf\prenom{}~\nom{}\normalsize{}% 
         \vspace*{3mm}
      }
\end{center}
\vfill

\begin{center}\em
Durée : 30 minutes.

  Aucun document n'est autorisé.
  L'usage de la calculatrice est interdit.

  Les questions faisant apparaitre le symbole \multiSymbole{} peuvent
  présenter zéro, une ou plusieurs bonnes réponses. Les autres ont
  une unique bonne réponse.

  Des points négatifs pourront être affectés à de \emph{très
  mauvaises} réponses.
\end{center}
\vspace{1ex}
\vfill
\pagebreak
%%% fin de l'en-tête

\restituegroupe{groupes}
%\restituegroupe{securite}



\AMCassociation{\id}

%\AMCaddpagesto{3}
	  }
}

%%%%§§§§§§§§§§§§§§§§§§§§§§§§§§§§§§§§§

\begin{document}
%%%Options
\AMCrandomseed{10}

\def\AMCformQuestion#1{{\sc Question #1 :}}

\setdefaultgroupmode{withoutreplacement}
%%% Fin Options

\element{securite}{
  \begin{questionmult}{pref}    
    Pourquoi le courant électrique est dangereux pour les humains~?
    \begin{reponses}
      \bonne{Il peut paralyser les muscles du corps}
      \mauvaise{Il peut effacer les contacts de mon téléphone}
      \bonne{Il peut arrêter la respiration et le c{\oe}ur}
    \end{reponses}
  \end{questionmult}
}

\element{securite}{
  \begin{questionmult}{pref}    
    Quelles sources de courants sont dangereuses pour les humains~?
    \begin{reponses}
      \bonne{Les prises du secteur}
      \mauvaise{Les piles-boutons}
      \bonne{Les fils haute-tension sur les poteaux électriques}
      \bonne{Les éclairs}
      \mauvaise{L'électricité statique sur un pull en laine}
    \end{reponses}
  \end{questionmult}
}

\element{generalites}{
  \begin{questionmult}{pref}
  	 
\begin{circuitikz}[european]
 \draw (0,0) -- (4,0);
 \draw (0,3) -- (4,3);

 \draw (0,0)
 to [ american voltage source, invert, i_>=${i_3}$, o-o] (0,3);
 \draw (2,0)
 to [ lamp=$L_178$, i<_=${i_2=\SI{140}{\mA}}$, o-o]  (2,3);
 \draw (4,0)
 to [ lamp=$L_179$, i^<=${i_1=\SI{100}{\mA}}$, o-o]  (4,3);
\end{circuitikz}


    Ce montage est-il en série ou en dérivation~?
    \begin{reponses}
      \bonne{Le montage est branché en série}
      \mauvaise{Le montage est branché en dérivation}
    \end{reponses}
  \end{questionmult}
}

\element{generalites}{
  \begin{questionmult}{pref}    
    Ce montage est-il en série ou en dérivation~?
    \begin{reponses}
      \mauvaise{Le montage est branché en série}
      \bonne{Le montage est branché en dérivation}
    \end{reponses}
  \end{questionmult}
}


\element{groupes}{
\section{Sécurité}
\restituegroupe{securite}

\section{Généralités}
\restituegroupe{generalites}
}


\csvreader[head to column names]{liste.csv}{}{\sujet}

\end{document}

