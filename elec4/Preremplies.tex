\documentclass[12pt,a4paper]{article}

\usepackage{csvsimple}%

\usepackage[francais,bloc]{automultiplechoice}

\newcommand{\sujet}{
\exemplaire{1}{%

\AMCsetFoot{\niveau{}°\classe{} -- \prenom{}~\nom{} -- \thepage}

%%% debut de l'en-tête des copies :  
\begin{center}
\vfill
\noindent{\large \bf Classe de \niveau{}°\classe{}}

\vspace*{3mm}

{\Large\bf Évaluation sur table - Électricité 27/05/2024}

\vfill
\noindent{}\fbox{\vspace*{3mm}
         \Huge\bf\prenom{}~\nom{}\normalsize{}% 
         \vspace*{3mm}
      }
\end{center}
\vfill

\begin{center}\em
Durée : 30 minutes.

  Aucun document n'est autorisé.
  L'usage de la calculatrice est interdit.

  Les questions faisant apparaitre le symbole \multiSymbole{} peuvent
  présenter zéro, une ou plusieurs bonnes réponses. Les autres ont
  une unique bonne réponse.

  Des points négatifs pourront être affectés à de \emph{très
  mauvaises} réponses.
\end{center}
\vspace{1ex}
\vfill
\pagebreak
%%% fin de l'en-tête

\restituegroupe{general}
 


\AMCassociation{\id}

%\AMCaddpagesto{3}
	  }
}

%%%%§§§§§§§§§§§§§§§§§§§§§§§§§§§§§§§§§

\begin{document}

%%%Options
\AMCrandomseed{10}

\def\AMCformQuestion#1{{\sc Question #1 :}}

\setdefaultgroupmode{withoutreplacement}
%%% Fin Options

%%% groupes

\element{general}{
  \begin{question}{prez}
  \input{"|/home/ypatois/unison/work/enseignement/college_nelson_mandela/evaluations/elec4/test-script.sh"}
    Parmi les personnalités suivantes, laquelle a été présidente de la république française~?
    \begin{reponses}
      \bonne{René Coty}
      \mauvaise{Alain Prost}
      \mauvaise{Marcel Proust}
      \mauvaise{Claude Monet}
    \end{reponses}
  \end{question}
}

\element{general}{
  \begin{questionmult}{pref}    
    Parmi les villes suivantes, lesquelles sont des préfectures~?
    \begin{reponses}
      \bonne{Poitiers}
      \mauvaise{Sainte-Menehould}
      \bonne{Avignon}
    \end{reponses}
  \end{questionmult}
}

\element{general}{
  \begin{question}{nb-ue}
    Combien d'états sont membres de l'Union Européenne en janvier 2009~?
    \begin{reponseshoriz}[o]
      \mauvaise{15}
      \mauvaise{21}
      \mauvaise{25}
      \bonne{27}
      \mauvaise{31}
    \end{reponseshoriz}
  \end{question}
}

%%%% fin des groupes

\csvreader[head to column names]{liste.csv}{}{\sujet}

\end{document}
