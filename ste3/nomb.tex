
\element{nomb}{
  \begin{question}{nomb1}
    Quel est le nom de l'alcane dont la formule est \(\ce{CH4}\)~?
    \begin{reponses}
      \mauvaise{éthane}
      \mauvaise{propane}
      \mauvaise{butane}
      \bonne{méthane}
    \end{reponses}
    \explain{Le méthane est le premier alcane de la série. Sa formule chimique est \(\ce{CH4}\), ce qui correspond à un seul atome de carbone et à quatre atomes d'hydrogène.}
  \end{question}
}

\element{nomb}{
  \begin{question}{nomb2}
    Quelle est la formule du méthane~?
    \begin{reponses}
      \mauvaise{\(\ce{CH3}\)}
      \mauvaise{\(\ce{C2H6}\)}
      \mauvaise{\(\ce{C3H8}\)}
      \bonne{\(\ce{CH4}\)}
    \end{reponses}
    \explain{Le méthane est le plus simple des alcanes. Il est composé d'un seul atome de carbone lié à quatre atomes d'hydrogène, ce qui donne la formule \(\ce{CH4}\).}
  \end{question}
}

\element{nomb}{
  \begin{question}{nomb3}
    Quel est le nom de l'alcane dont la formule est \(\ce{C2H6}\)~?
    \begin{reponses}
      \mauvaise{méthane}
      \mauvaise{propane}
      \mauvaise{butane}
      \bonne{éthane}
    \end{reponses}
    \explain{L'éthane est le deuxième alcane de la série. Sa formule chimique est \(\ce{C2H6}\), ce qui correspond à deux atomes de carbone et à six atomes d'hydrogène.}
  \end{question}
}

\element{nomb}{
  \begin{question}{nomb4}
    Quelle est la formule de l'éthane~?
    \begin{reponses}
      \mauvaise{\(\ce{C2H4}\)}
      \mauvaise{\(\ce{C3H8}\)}
      \mauvaise{\(\ce{C2H8}\)}
      \bonne{\(\ce{C2H6}\)}
    \end{reponses}
    \explain{L'éthane est composé de deux atomes de carbone et de six atomes d'hydrogène, ce qui donne la formule \(\ce{C2H6}\).}
  \end{question}
}

\element{nomb}{
  \begin{question}{nomb5}
    Quel est le nom de l'alcane dont la formule est \(\ce{C3H8}\)~?
    \begin{reponses}
      \mauvaise{éthane}
      \mauvaise{butane}
      \mauvaise{pentane}
      \bonne{propane}
    \end{reponses}
    \explain{Le propane est le troisième alcane de la série. Sa formule chimique est \(\ce{C3H8}\), ce qui correspond à trois atomes de carbone et à huit atomes d'hydrogène.}
  \end{question}
}

\element{nomb}{
  \begin{question}{nomb6}
    Quelle est la formule du propane~?
    \begin{reponses}
      \mauvaise{\(\ce{C3H6}\)}
      \mauvaise{\(\ce{C4H10}\)}
      \mauvaise{\(\ce{C2H8}\)}
      \bonne{\(\ce{C3H8}\)}
    \end{reponses}
    \explain{Le propane est composé de trois atomes de carbone et de huit atomes d'hydrogène, ce qui donne la formule \(\ce{C3H8}\).}
  \end{question}
}

\element{nomb}{
  \begin{question}{nomb7}
    Quel est le nom de l'alcane dont la formule est \(\ce{C4H10}\)~?
    \begin{reponses}
      \mauvaise{propane}
      \mauvaise{pentane}
      \mauvaise{hexane}
      \bonne{butane}
    \end{reponses}
    \explain{Le butane est le quatrième alcane de la série. Sa formule chimique est \(\ce{C4H10}\), ce qui correspond à quatre atomes de carbone et à dix atomes d'hydrogène.}
  \end{question}
}

\element{nomb}{
  \begin{question}{nomb8}
    Quelle est la formule du butane~?
    \begin{reponses}
      \mauvaise{\(\ce{C4H8}\)}
      \mauvaise{\(\ce{C5H12}\)}
      \mauvaise{\(\ce{C3H10}\)}
      \bonne{\(\ce{C4H10}\)}
    \end{reponses}
    \explain{Le butane est composé de quatre atomes de carbone et de dix atomes d'hydrogène, ce qui donne la formule \(\ce{C4H10}\).}
  \end{question}
}

\element{nomb}{
  \begin{question}{nomb9}
    Quel est le nom de l'alcane dont la formule est \(\ce{C5H12}\)~?
    \begin{reponses}
      \mauvaise{butane}
      \mauvaise{hexane}
      \mauvaise{heptane}
      \bonne{pentane}
    \end{reponses}
    \explain{Le pentane est le cinquième alcane de la série. Sa formule chimique est \(\ce{C5H12}\), ce qui correspond à cinq atomes de carbone et à douze atomes d'hydrogène.}
  \end{question}
}

\element{nomb}{
  \begin{question}{nomb10}
    Quelle est la formule du pentane~?
    \begin{reponses}
      \mauvaise{\(\ce{C5H10}\)}
      \mauvaise{\(\ce{C6H14}\)}
      \mauvaise{\(\ce{C4H12}\)}
      \bonne{\(\ce{C5H12}\)}
    \end{reponses}
    \explain{Le pentane est composé de cinq atomes de carbone et de douze atomes d'hydrogène, ce qui donne la formule \(\ce{C5H12}\).}
  \end{question}
}

