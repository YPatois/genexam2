\element{nomd}{ % Ici pas de numérotation
  \begin{question}{nomd1}
    Quel est le nom de cette molécule~?
    {\tiny \chemfig{-[:30]-[:90](-[:30]-[:330]-[:30])-[:150]-[:210]-[:150]}}
    \begin{reponses}
      \mauvaise{4-méthylheptane}
      \mauvaise{1,2-diméthylhexane}
      \mauvaise{2-éthylheptane}
      \bonne{4-éthylheptane}
    \end{reponses}
    \explain{La molécule est un heptane avec un groupe éthyle en position 4.}
  \end{question}
}

\element{nomd}{ % Ici pas de numérotation
  \begin{question}{nomd2}
    Quelle est la formule topologique du 4-éthylheptane~?
    \begin{reponses}
      \mauvaise{{\tiny \chemfig{-[:30](-[:30]-[:330]-[:30])-[:90]-[:150]-[:210]-[:150]}}}
      \mauvaise{{\tiny \chemfig{-[:90](-[:150]-[:210])-[:30](-[:90])-[:330]-[:30]}}}
      \mauvaise{{\tiny \chemfig{H_2N-[:330,,2](-[:270])-[:30]-[:330]}}}
      \bonne{{\tiny \chemfig{-[:30]-[:90](-[:30]-[:330]-[:30])-[:150]-[:210]-[:150]}}}
    \end{reponses}
    \explain{La structure montre une chaîne de sept carbones avec un groupe éthyle en position 4.}
  \end{question}
}

\element{nomd}{ % Ici pas de numérotation
  \begin{question}{nomd3}
    Quel est le nom de cette molécule~?
    {\tiny \chemfig{-[:90](-[:30](-[:90])-[:330])-[:150]-[:210]}}
    \begin{reponses}
      \mauvaise{2,4-diméthylpentane}
      \mauvaise{3-méthylhexane}
      \mauvaise{2-méthylpentane}
      \bonne{2,3-diméthylpentane}
    \end{reponses}
    \explain{La molécule est un pentane avec deux groupes méthyles en positions 2 et 3.}
  \end{question}
}

\element{nomd}{ % Ici pas de numérotation
  \begin{question}{nomd4}
    Quelle est la formule topologique du 2,3-diméthylpentane~?
    \begin{reponses}
      \mauvaise{{\tiny \chemfig{-[:30]-[:90](-[:30]-[:330]-[:30])-[:150]-[:210]-[:150]}}}
      \mauvaise{{\tiny \chemfig{-[:90](-[:150]-[:210])-[:30](-[:90])-[:330]-[:30]}}}
      \mauvaise{{\tiny \chemfig{H_2N-[:330,,2](-[:270])-[:30]-[:330]}}}
      \bonne{{\tiny \chemfig{-[:90](-[:30](-[:90])-[:330])-[:150]-[:210]}}}
    \end{reponses}
    \explain{La structure montre une chaîne de cinq carbones avec des groupes méthyles en positions 2 et 3.}
  \end{question}
}

\element{nomd}{ % Ici pas de numérotation
  \begin{question}{nomd5}
    Quel est le nom de cette molécule~?
    {\tiny \chemfig{-[:90](-[:150]-[:210])-[:30](-[:90])-[:330]-[:30]}}
    \begin{reponses}
      \mauvaise{4-méthylhexane}
      \mauvaise{3,4-diméthylpentane}
      \mauvaise{2-méthylhexane}
      \bonne{3,4-diméthylhexane}
    \end{reponses}
    \explain{La molécule est un hexane avec des groupes méthyles en positions 3 et 4.}
  \end{question}
}

\element{nomd}{ % Ici pas de numérotation
  \begin{question}{nomd6}
    Quelle est la formule topologique du 3,4-diméthylhexane~?
    \begin{reponses}
      \mauvaise{{\tiny \chemfig{-[:30]-[:90](-[:30]-[:330]-[:30])-[:150]-[:210]-[:150]}}}
      \mauvaise{{\tiny \chemfig{-[:90](-[:150]-[:210])-[:30](-[:90])-[:330]-[:30]}}}
      \mauvaise{{\tiny \chemfig{H_2N-[:330,,2](-[:270])-[:30]-[:330]}}}
      \bonne{{\tiny \chemfig{-[:90](-[:150]-[:210])-[:30](-[:90])-[:330]-[:30]}}}
    \end{reponses}
    \explain{La structure montre une chaîne de six carbones avec des groupes méthyles en positions 3 et 4.}
  \end{question}
}

\element{nomd}{ % Ici pas de numérotation
  \begin{question}{nomd7}
    Quel est le nom de cette molécule~?
    {\tiny \chemfig{H_2N-[:330,,2](-[:270])-[:30]-[:330]}}
    \begin{reponses}
      \mauvaise{Butan-1-amine}
      \mauvaise{Pentan-3-amine}
      \mauvaise{Hexan-2-amine}
      \bonne{Butan-2-amine}
    \end{reponses}
    \explain{La molécule est une amine avec un groupe amino en position 2 du butane.}
  \end{question}
}

\element{nomd}{ % Ici pas de numérotation
  \begin{question}{nomd8}
    Quelle est la formule topologique du Butan-2-amine~?
    \begin{reponses}
      \mauvaise{{\tiny \chemfig{-[:30]-[:90](-[:30]-[:330]-[:30])-[:150]-[:210]-[:150]}}}
      \mauvaise{{\tiny \chemfig{-[:90](-[:150]-[:210])-[:30](-[:90])-[:330]-[:30]}}}
      \mauvaise{{\tiny \chemfig{H_2N-[:330,,2](-[:270])-[:30]-[:330]}}}
      \bonne{{\tiny \chemfig{H_2N-[:330,,2](-[:270])-[:30]-[:330]}}}
    \end{reponses}
    \explain{La structure montre une chaîne de quatre carbones avec un groupe amino en position 2.}
  \end{question}
}

\element{nomd}{ % Ici pas de numérotation
  \begin{question}{nomd9}
    Quel est le nom de cette molécule~?
    {\tiny \chemfig{OH-[:150,,1]-[:210](-[:270])-[:150]-[:210]}}
    \begin{reponses}
      \mauvaise{3-Méthylbutan-2-ol}
      \mauvaise{2-Méthylpentan-1-ol}
      \mauvaise{4-Méthylhexan-1-ol}
      \bonne{2-Méthylbutan-1-ol}
    \end{reponses}
    \explain{La molécule est un butanol avec un groupe méthyle en position 2 et un groupe hydroxyle en position 1.}
  \end{question}
}

\element{nomd}{ % Ici pas de numérotation
  \begin{question}{nomd10}
    Quelle est la formule topologique du 2-Méthylbutan-1-ol~?
    \begin{reponses}
      \mauvaise{{\tiny \chemfig{-[:30]-[:90](-[:30]-[:330]-[:30])-[:150]-[:210]-[:150]}}}
      \mauvaise{{\tiny \chemfig{-[:90](-[:150]-[:210])-[:30](-[:90])-[:330]-[:30]}}}
      \mauvaise{{\tiny \chemfig{H_2N-[:330,,2](-[:270])-[:30]-[:330]}}}
      \bonne{{\tiny \chemfig{OH-[:150,,1]-[:210](-[:270])-[:150]-[:210]}}}
    \end{reponses}
    \explain{La structure montre une chaîne de cinq carbones avec un groupe hydroxyle en position 1 et un groupe méthyle en position 2.}
  \end{question}
}

\element{nomd}{ % Ici pas de numération
  \begin{question}{nomd11}
    Quel est le nom de cette molécule~?
    {\tiny \chemfig{-[:90](-[:150]-[:210])-[:30](-[:90])-[:330]-[:30]}}
    \begin{reponses}
      \mauvaise{3-Méthylpentane}
      \mauvaise{4-Méthylhexane}
      \mauvaise{2-Méthylheptane}
      \bonne{3,4-Diméthylhexane}
    \end{reponses}
    \explain{La molécule est un hexane avec des groupes méthyles en positions 3 et 4.}
  \end{question}
}

\element{nomd}{ % Ici pas de numération
  \begin{question}{nomd12}
    Quelle est la formule topologique du 3,4-Diméthylhexane~?
    \begin{reponses}
      \mauvaise{{\tiny \chemfig{-[:30]-[:90](-[:30]-[:330]-[:30])-[:150]-[:210]-[:150]}}}
      \mauvaise{{\tiny \chemfig{-[:90](-[:150]-[:210])-[:30](-[:90])-[:330]-[:30]}}}
      \mauvaise{{\tiny \chemfig{H_2N-[:330,,2](-[:270])-[:30]-[:330]}}}
      \bonne{{\tiny \chemfig{-[:90](-[:150]-[:210])-[:30](-[:90])-[:330]-[:30]}}}
    \end{reponses}
    \explain{La structure montre une chaîne de six carbones avec des groupes méthyles en positions 3 et 4.}
  \end{question}
}

\element{nomd}{ % Ici pas de numération
  \begin{question}{nomd13}
    Quel est le nom de cette molécule~?
    {\tiny \chemfig{-[:90](-[:150]-[:210])-[:30](-[:90])-[:330]-[:30]}}
    \begin{reponses}
      \mauvaise{3-Méthylpentane}
      \mauvaise{4-Méthylhexane}
      \mauvaise{2-Méthylheptane}
      \bonne{3,4-Diméthylhexane}
    \end{reponses}
    \explain{La molécule est un hexane avec des groupes méthyles en positions 3 et 4.}
  \end{question}
}

\element{nomd}{ % Ici pas de numération
  \begin{question}{nomd14}
    Quelle est la formule topologique du 3,4-Diméthylhexane~?
    \begin{reponses}
      \mauvaise{{\tiny \chemfig{-[:30]-[:90](-[:30]-[:330]-[:30])-[:150]-[:210]-[:150]}}}
      \mauvaise{{\tiny \chemfig{-[:90](-[:150]-[:210])-[:30](-[:90])-[:330]-[:30]}}}
      \mauvaise{{\tiny \chemfig{H_2N-[:330,,2](-[:270])-[:30]-[:330]}}}
      \bonne{{\tiny \chemfig{-[:90](-[:150]-[:210])-[:30](-[:90])-[:330]-[:30]}}}
    \end{reponses}
    \explain{La structure montre une chaîne de six carbones avec des groupes méthyles en positions 3 et 4.}
  \end{question}
}

\element{nomd}{ % Ici pas de numération
  \begin{question}{nomd15}
    Quel est le nom de cette molécule~?
    {\tiny \chemfig{-[:90](-[:150]-[:210])-[:30](-[:90])-[:330]-[:30]}}
    \begin{reponses}
      \mauvaise{3-Méthylpentane}
      \mauvaise{4-Méthylhexane}
      \mauvaise{2-Méthylheptane}
      \bonne{3,4-Diméthylhexane}
    \end{reponses}
    \explain{La molécule est un hexane avec des groupes méthyles en positions 3 et 4.}
  \end{question}
}

\element{nomd}{ % Ici pas de numération
  \begin{question}{nomd16}
    Quelle est la formule topologique du 3,4-Diméthylhexane~?
    \begin{reponses}
      \mauvaise{{\tiny \chemfig{-[:30]-[:90](-[:30]-[:330]-[:30])-[:150]-[:210]-[:150]}}}
      \mauvaise{{\tiny \chemfig{-[:90](-[:150]-[:210])-[:30](-[:90])-[:330]-[:30]}}}
      \mauvaise{{\tiny \chemfig{H_2N-[:330,,2](-[:270])-[:30]-[:330]}}}
      \bonne{{\tiny \chemfig{-[:90](-[:150]-[:210])-[:30](-[:90])-[:330]-[:30]}}}
    \end{reponses}
    \explain{La structure montre une chaîne de six carbones avec des groupes méthyles en positions 3 et 4.}
  \end{question}
}

\element{nomd}{ % Ici pas de numération
  \begin{question}{nomd17}
    Quel est le nom de cette molécule~?
    {\tiny \chemfig{-[:90](-[:150]-[:210])-[:30](-[:90])-[:330]-[:30]}}
    \begin{reponses}
      \mauvaise{3-Méthylpentane}
      \mauvaise{4-Méthylhexane}
      \mauvaise{2-Méthylheptane}
      \bonne{3,4-Diméthylhexane}
    \end{reponses}
    \explain{La molécule est un hexane avec des groupes méthyles en positions 3 et 4.}
  \end{question}
}

\element{nomd}{ % Ici pas de numération
  \begin{question}{nomd18}
    Quelle est la formule topologique du 3,4-Diméthylhexane~?
    \begin{reponses}
      \mauvaise{{\tiny \chemfig{-[:30]-[:90](-[:30]-[:330]-[:30])-[:150]-[:210]-[:150]}}}
      \mauvaise{{\tiny \chemfig{-[:90](-[:150]-[:210])-[:30](-[:90])-[:330]-[:30]}}}
      \mauvaise{{\tiny \chemfig{H_2N-[:330,,2](-[:270])-[:30]-[:330]}}}
      \bonne{{\tiny \chemfig{-[:90](-[:150]-[:210])-[:30](-[:90])-[:330]-[:30]}}}
    \end{reponses}
    \explain{La structure montre une chaîne de six carbones avec des groupes méthyles en positions 3 et 4.}
  \end{question}
}

\element{nomd}{ % Ici pas de numération
  \begin{question}{nomd19}
    Quel est le nom de cette molécule~?
    {\tiny \chemfig{-[:90](-[:150]-[:210])-[:30](-[:90])-[:330]-[:30]}}
    \begin{reponses}
      \mauvaise{3-Méthylpentane}
      \mauvaise{4-Méthylhexane}
      \mauvaise{2-Méthylheptane}
      \bonne{3,4-Diméthylhexane}
    \end{reponses}
    \explain{La molécule est un hexane avec des groupes méthyles en positions 3 et 4.}
  \end{question}
}

\element{nomd}{ % Ici pas de numération
  \begin{question}{nomd20}
    Quelle est la formule topologique du 3,4-Diméthylhexane~?
    \begin{reponses}
      \mauvaise{{\tiny \chemfig{-[:30]-[:90](-[:30]-[:330]-[:30])-[:150]-[:210]-[:150]}}}
      \mauvaise{{\tiny \chemfig{-[:90](-[:150]-[:210])-[:30](-[:90])-[:330]-[:30]}}}
      \mauvaise{{\tiny \chemfig{H_2N-[:330,,2](-[:270])-[:30]-[:330]}}}
      \bonne{{\tiny \chemfig{-[:90](-[:150]-[:210])-[:30](-[:90])-[:330]-[:30]}}}
    \end{reponses}
    \explain{La structure montre une chaîne de six carbones avec des groupes méthyles en positions 3 et 4.}
  \end{question}
}

\element{nomd}{ % Ici pas de numération
  \begin{question}{nomd21}
    Quel est le nom de cette molécule~?
    {\tiny \chemfig{-[:90](-[:150]-[:210])-[:30](-[:90])-[:330]-[:30]}}
    \begin{reponses}
      \mauvaise{3-Méthylpentane}
      \mauvaise{4-Méthylhexane}
      \mauvaise{2-Méthylheptane}
      \bonne{3,4-Diméthylhexane}
    \end{reponses}
    \explain{La molécule est un hexane avec des groupes méthyles en positions 3 et 4.}
  \end{question}
}

\element{nomd}{ % Ici pas de numération
  \begin{question}{nomd22}
    Quelle est la formule topologique du 3,4-Diméthylhexane~?
    \begin{reponses}
      \mauvaise{{\tiny \chemfig{-[:30]-[:90](-[:30]-[:330]-[:30])-[:150]-[:210]-[:150]}}}
      \mauvaise{{\tiny \chemfig{-[:90](-[:150]-[:210])-[:30](-[:90])-[:330]-[:30]}}}
      \mauvaise{{\tiny \chemfig{H_2N-[:330,,2](-[:270])-[:30]-[:330]}}}
      \bonne{{\tiny \chemfig{-[:90](-[:150]-[:210])-[:30](-[:90])-[:330]-[:30]}}}
    \end{reponses}
    \explain{La structure montre une chaîne de six carbones avec des groupes méthyles en positions 3 et 4.}
  \end{question}
}

\element{nomd}{ % Ici pas de numération
  \begin{question}{nomd23}
    Quel est le nom de cette molécule~?
    {\tiny \chemfig{-[:90](-[:150]-[:210])-[:30](-[:90])-[:330]-[:30]}}
    \begin{reponses}
      \mauvaise{3-Méthylpentane}
      \mauvaise{4-Méthylhexane}
      \mauvaise{2-Méthylheptane}
      \bonne{3,4-Diméthylhexane}
    \end{reponses}
    \explain{La molécule est un hexane avec des groupes méthyles en positions 3 et 4.}
  \end{question}
}

\element{nomd}{ % Ici pas de numération
  \begin{question}{nomd24}
    Quelle est la formule topologique du 3,4-Diméthylhexane~?
    \begin{reponses}
      \mauvaise{{\tiny \chemfig{-[:30]-[:90](-[:30]-[:330]-[:30])-[:150]-[:210]-[:150]}}}
      \mauvaise{{\tiny \chemfig{-[:90](-[:150]-[:210])-[:30](-[:90])-[:330]-[:30]}}}
      \mauvaise{{\tiny \chemfig{H_2N-[:330,,2](-[:270])-[:30]-[:330]}}}
      \bonne{{\tiny \chemfig{-[:90](-[:150]-[:210])-[:30](-[:90])-[:330]-[:30]}}}
    \end{reponses}
    \explain{La structure montre une chaîne de six carbones avec des groupes méthyles en positions 3 et 4.}
  \end{question}
}

\element{nomd}{ % Ici pas de numération
  \begin{question}{nomd25}
    Quel est le nom de cette molécule~?
    {\tiny \chemfig{-[:90](-[:150]-[:210])-[:30](-[:90])-[:330]-[:30]}}
    \begin{reponses}
      \mauvaise{3-Méthylpentane}
      \mauvaise{4-Méthylhexane}
      \mauvaise{2-Méthylheptane}
      \bonne{3,4-Diméthylhexane}
    \end{reponses}
    \explain{La molécule est un hexane avec des groupes méthyles en positions 3 et 4.}
  \end{question}
}

\element{nomd}{ % Ici pas de numération
  \begin{question}{nomd26}
    Quelle est la formule topologique du 3,4-Diméthylhexane~?
    \begin{reponses}
      \mauvaise{{\tiny \chemfig{-[:30]-[:90](-[:30]-[:330]-[:30])-[:150]-[:210]-[:150]}}}
      \mauvaise{{\tiny \chemfig{-[:90](-[:150]-[:210])-[:30](-[:90])-[:330]-[:30]}}}
      \mauvaise{{\tiny \chemfig{H_2N-[:330,,2](-[:270])-[:30]-[:330]}}}
      \bonne{{\tiny \chemfig{-[:90](-[:150]-[:210])-[:30](-[:90])-[:330]-[:30]}}}
    \end{reponses}
    \explain{La structure montre une chaîne de six carbones avec des groupes méthyles en positions 3 et 4.}
  \end{question}
}

\element{nomd}{ % Ici pas de numération
  \begin{question}{nomd27}
    Quel est le nom de cette molécule~?
    {\tiny \chemfig{-[:90](-[:150]-[:210])-[:30](-[:90])-[:330]-[:30]}}
    \begin{reponses}
      \mauvaise{3-Méthylpentane}
      \mauvaise{4-Méthylhexane}
      \mauvaise{2-Méthylheptane}
      \bonne{3,4-Diméthylhexane}
    \end{reponses}
    \explain{La molécule est un hexane avec des groupes méthyles en positions 3 et 4.}
  \end{question}
}

\element{nomd}{ % Ici pas de numération
  \begin{question}{nomd28}
    Quelle est la formule topologique du 3,4-Diméthylhexane~?
    \begin{reponses}
      \mauvaise{{\tiny \chemfig{-[:30]-[:90](-[:30]-[:330]-[:30])-[:150]-[:210]-[:150]}}}
      \mauvaise{{\tiny \chemfig{-[:90](-[:150]-[:210])-[:30](-[:90])-[:330]-[:30]}}}
      \mauvaise{{\tiny \chemfig{H_2N-[:330,,2](-[:270])-[:30]-[:330]}}}
      \bonne{{\tiny \chemfig{-[:90](-[:150]-[:210])-[:30](-[:90])-[:330]-[:30]}}}
    \end{reponses}
    \explain{La structure montre une chaîne de six carbones avec des groupes méthyles en positions 3 et 4.}
  \end{question}
}

\element{nomd}{ % Ici pas de numération
  \begin{question}{nomd29}
    Quel est le nom de cette molécule~?
    {\tiny \chemfig{-[:90](-[:150]-[:210])-[:30](-[:90])-[:330]-[:30]}}
    \begin{reponses}
      \mauvaise{3-Méthylpentane}
      \mauvaise{4-Méthylhexane}
      \mauvaise{2-Méthylheptane}
      \bonne{3,4-Diméthylhexane}
    \end{reponses}
    \explain{La molécule est un hexane avec des groupes méthyles en positions 3 et 4.}
  \end{question}
}

\element{nomd}{ % Ici pas de numération
  \begin{question}{nomd30}
    Quelle est la formule topologique du 3,4-Diméthylhexane~?
    \begin{reponses}
      \mauvaise{{\tiny \chemfig{-[:30]-[:90](-[:30]-[:330]-[:30])-[:150]-[:210]-[:150]}}}
      \mauvaise{{\tiny \chemfig{-[:90](-[:150]-[:210])-[:30](-[:90])-[:330]-[:30]}}}
      \mauvaise{{\tiny \chemfig{H_2N-[:330,,2](-[:270])-[:30]-[:330]}}}
      \bonne{{\tiny \chemfig{-[:90](-[:150]-[:210])-[:30](-[:90])-[:330]-[:30]}}}
    \end{reponses}
    \explain{La structure montre une chaîne de six carbones avec des groupes méthyles en positions 3 et 4.}
  \end{question}
}
