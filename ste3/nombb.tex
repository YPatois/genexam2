\element{nombb}{
  \begin{question}{hexa1}
    Quel est le nom de l'alcane dont la formule est $\ce{C6H14}$ ?
    \begin{reponses}
      \mauvaise{heptane}
      \mauvaise{octane}
      \mauvaise{nonane}
      \bonne{hexane}
    \end{reponses}
    \explain{L'hexane est un alcane avec six atomes de carbone et 14 atomes 
    d'hydrogène, ce qui correspond à la formule $\ce{C6H14}$.}
  \end{question}
}


\element{nombb}{
  \begin{question}{hexa2}
    Quelle est la formule de l'hexane ?
    \begin{reponses}
      \mauvaise{$\ce{C5H12}$}
      \mauvaise{$\ce{C6H12}$}
      \mauvaise{$\ce{C7H16}$}
      \bonne{$\ce{C6H14}$}
    \end{reponses}
    \explain{L'hexane se compose de six atomes de carbone et de 14 atomes d'hydrogène, 
    donnant la formule $\ce{C6H14}$.}
  \end{question}
}

\element{nombb}{
  \begin{question}{hepta1}
    Quel est le nom de l'alcane dont la formule est $\ce{C7H16}$ ?
    \begin{reponses}
      \mauvaise{hexane}
      \mauvaise{octane}
      \mauvaise{nonane}
      \bonne{heptane}
    \end{reponses}
    \explain{Le heptane est un alcane avec sept atomes de carbone et 16 atomes d'hydrogène, 
    ce qui correspond à la formule $\ce{C7H16}$.}
  \end{question}
}

\element{heptnombba2}{
  \begin{question}{hepta2}
    Quelle est la formule du heptane ?
    \begin{reponses}
      \mauvaise{$\ce{C6H14}$}
      \mauvaise{$\ce{C7H14}$}
      \mauvaise{$\ce{C8H18}$}
      \bonne{$\ce{C7H16}$}
    \end{reponses}
    \explain{Le heptane se compose de sept atomes de carbone et de 16 atomes d'hydrogène, 
    donnant la formule $\ce{C7H16}$.}
  \end{question}
}

\element{nombb}{
  \begin{question}{octa1}
    Quel est le nom de l'alcane dont la formule est $\ce{C8H18}$ ?
    \begin{reponses}
      \mauvaise{heptane}
      \mauvaise{nonane}
      \mauvaise{décane}
      \bonne{octane}
    \end{reponses}
    \explain{L'octane est un alcane avec huit atomes de carbone et 18 atomes d'hydrogène, 
    ce qui correspond à la formule $\ce{C8H18}$.}
  \end{question}
}

\element{nombb}{
  \begin{question}{octa2}
    Quelle est la formule de l'octane ?
    \begin{reponses}
      \mauvaise{$\ce{C7H16}$}
      \mauvaise{$\ce{C8H16}$}
      \mauvaise{$\ce{C9H20}$}
      \bonne{$\ce{C8H18}$}
    \end{reponses}
    \explain{L'octane se compose de huit atomes de carbone et de 18 atomes d'hydrogène, 
    donnant la formule $\ce{C8H18}$.}
  \end{question}
}

\element{nombb}{
  \begin{question}{nona1}
    Quel est le nom de l'alcane dont la formule est $\ce{C9H20}$ ?
    \begin{reponses}
      \mauvaise{octane}
      \mauvaise{décane}
      \mauvaise{hexane}
      \bonne{nonane}
    \end{reponses}
    \explain{Le nonane est un alcane avec neuf atomes de carbone et 20 atomes d'hydrogène, 
    ce qui correspond à la formule $\ce{C9H20}$.}
  \end{question}
}

\element{nombb}{
  \begin{question}{nona2}
    Quelle est la formule du nonane ?
    \begin{reponses}
      \mauvaise{$\ce{C8H18}$}
      \mauvaise{$\ce{C9H18}$}
      \mauvaise{$\ce{C10H22}$}
      \bonne{$\ce{C9H20}$}
    \end{reponses}
    \explain{Le nonane se compose de neuf atomes de carbone et de 20 atomes d'hydrogène, 
    donnant la formule $\ce{C9H20}$.}
  \end{question}
}

\element{nombb}{
  \begin{question}{deca1}
    Quel est le nom de l'alcane dont la formule est $\ce{C10H22}$ ?
    \begin{reponses}
      \mauvaise{nonane}
      \mauvaise{octane}
      \mauvaise{heptane}
      \bonne{décane}
    \end{reponses}
    \explain{Le décane est un alcane avec dix atomes de carbone et 22 atomes d'hydrogène, 
    ce qui correspond à la formule $\ce{C10H22}$.}
  \end{question}
}

\element{nombb}{
  \begin{question}{deca2}
    Quelle est la formule du décane ?
    \begin{reponses}
      \mauvaise{$\ce{C9H20}$}
      \mauvaise{$\ce{C10H20}$}
      \mauvaise{$\ce{C11H24}$}
      \bonne{$\ce{C10H22}$}
    \end{reponses}
    \explain{Le décane se compose de dix atomes de carbone et de 22 atomes d'hydrogène, 
    donnant la formule $\ce{C10H22}$.}
  \end{question}
}
