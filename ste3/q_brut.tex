\element{brut}{ % Ici pas de numérotation  
  \begin{question}{brut1}  
    Quel est la formule brute du Rouge para~?  
    {\small 
\chemfig{HO-[:30,,2]=^[:330](-[:30]-[:90](=_[:30.9,1.042]-[:330.4,1.042]=_[:270,1.042]-[:209.6,1.042]=_[:149.1,1.042])-[:150]=^[:210]-[:270])-[:270]N=[:210]N-[:270]=_[:330]-[:270]=_[:210](-[:270]\mcfbelow{N}{^{\mcfplus}}(=[:330]O)-[:210,,,2]^{\mcfminus}O)-[:150]=_[:\chemfig{HO-[:30,,2]=^[:330](-[:30]-[:90](=_[:30.9,1.042]-[:330.4,1.042]=_[:270,1.042]-[:209.6,1.042]=_[:149.1,1.042])-[:150]=^[:210]-[:270])-[:270]N=[:210]N-[:270]=_[:330]-[:270]=_[:210](-[:270]\mcfbelow{N}{^{\mcfplus}}(=[:330]O)-[:210,,,2]^{\mcfminus}O-[:150]=_[:90](-[:30])}}}
    \begin{reponses}  
      \mauvaise{\ce{C16H11O3N2}}  
      \mauvaise{\ce{C16H10O3N3}}  
      \mauvaise{\ce{C15H11O3N3}}  
      \bonne{\ce{C16H11O3N3}}  
    \end{reponses}  
    \explain{La formule brute correspond au nombre exact de chaque type d'atome dans la molécule. Ici, il y a 16 atomes de carbone, 11 d'hydrogène, 3 d'oxygène et 3 d'azote.}  
  \end{question}  

  \begin{question}{brut2}  
    Quel est la formule brute du Paracétamol~?  
    {\small \chemfig{OH-[:90,,1]C=^[:30,,,1]CH-[:90,,1,1]CH=^[:150,,1]C(-[:210,,,2]HC=^[:270,,2,2]HC-[:330,,2]\phantom{C})-[:90,,,2]HN-[:30,,2]C(=[:330]O)-[:90,,,1]CH_3}  }
    \begin{reponses}  
      \mauvaise{\ce{C8H8NO2}}  
      \mauvaise{\ce{C7H7NO2}}  
      \mauvaise{\ce{C7H9NO2}}  
      \bonne{\ce{C8H9NO2}}  
    \end{reponses}  
    \explain{La formule brute correspond au nombre exact de chaque type d'atome dans la molécule. Ici, il y a 8 atomes de carbone, 9 d'hydrogène, 1 d'azote et 2 d'oxygène.}  
  \end{question}  

  \begin{question}{brut3}  
    Quel est la formule brute de la Kétamine~?  
    {\small 
\chemfig{Cl-[:100]C-[:160]C(=_[:100,,,2]HC-[:40,,2]\mcfabove{C}{H}=_[:340,,,1]CH-[:280,,1,1]CH=_[:220,,1]\phantom{C})-[:220]C(-[:120]\mcfabove{N}{H}-[:180,,,2]H_3C)-[:330,,,1]CH_2-[:270,,1,1]CH_2-[:210,,1]\mcfbelow{C}{\mcfright{H}{_2}}-[:150,,,2]H_2C-[:90,,2]C(-[:30\chemfig{Cl-[:100]C-[:160]C(=_[:100,,,2]HC-[:40,,2]\mcfabove{C}{H}=_[:340,,,1]CH-[:280,,1,1]CH=_[:220,,1]\phantom{C})-[:220]C(-[:120]\mcfabove{N}{H}-[:180,,,2]H_3C)-[:330,,,1]CH_2-[:270,,1,1]CH_2-[:210,,1]\mcfbelow{C}{\mcfright{H}{_2}}-[:150,,,2]H_2C-[:9,,2]C(-[:30]\phantom{C})=[:150]O} }}
    \begin{reponses}  
      \mauvaise{\ce{C13H15ClNO}}  
      \mauvaise{\ce{C12H16ClNO}}  
      \mauvaise{\ce{C13H16ClN}}  
      \bonne{\ce{C13H16ClNO}}  
    \end{reponses}  
    \explain{La formule brute correspond au nombre exact de chaque type d'atome dans la molécule. Ici, il y a 13 atomes de carbone, 16 d'hydrogène, 1 de chlore, 1 d'azote et 1 d'oxygène.}  
  \end{question}  

  \begin{question}{brut4}  
    Quel est la formule brute de l'Ibuprofène~?  
    {\small \chemfig{OH-[:270,,1](=[:330]O)-[:210](-[:150])-[:270]=_[:330]-[:270]=_[:210](-[:150]=_[:90]-[:30])-[:270]-[:210](-[:270])-[:150]}   }
    \begin{reponses}  
      \mauvaise{\ce{C13H18O}}  
      \mauvaise{\ce{C14H18O2}}  
      \mauvaise{\ce{C12H18O2}}  
      \bonne{\ce{C13H18O2}}  
    \end{reponses}  
    \explain{La formule brute correspond au nombre exact de chaque type d'atome dans la molécule. Ici, il y a 13 atomes de carbone, 18 d'hydrogène et 2 d'oxygène.}  
  \end{question}  
}
\element{brut}{ % Ici pas de numérotation
  \begin{question}{brut5}
    Quel est la formule brute de l'acide citrique~?
    {\small \chemfig{OH-[:120,,1](-[:210]-[:270](=[:330]O)-[:210,,,2]HO)(-[:30](=[:90]O)-[:330,,,1]OH)-[:120]-[:180](-[:120,,,2]HO)=[:240]O} }
    \begin{reponses}
      \mauvaise{\ce{C6H8O7}}
      \mauvaise{\ce{C6H7O7}}
      \mauvaise{\ce{C6H6O7}}
      \bonne{\ce{C6H8O7}}
    \end{reponses}
    \explain{L'addition ajoute les deux quantités, ici 2 et 2.}
  \end{question}
}
