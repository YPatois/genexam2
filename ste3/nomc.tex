\element{nomc}{  
  \begin{question}{nomen1}  
    Quel est le nom de la molécule \ce{CH3-CH2-CH2-OH} ?  
    \begin{reponses}  
      \mauvaise{Propan-2-ol}  
      \mauvaise{Butan-1-ol}  
      \mauvaise{Ethan-2-ol}  
      \bonne{Propan-1-ol}  
    \end{reponses}  
    \explain{La chaîne principale contient trois carbones et le groupe hydroxyle (\ce{-OH}) est en position 1. Le nom correct est donc propan-1-ol.}  
  \end{question}  
}

\element{nomc}{
  \begin{question}{nomen2}  
    Quelle est la formule du propan-2-ol ?  
    \begin{reponses}  
      \mauvaise{\ce{CH3-CHOH-CH3}}  
      \mauvaise{\ce{CH3-CH2-OH}}  
      \mauvaise{\ce{CH2OH-CH2-CH3}}  
      \bonne{\ce{CH3-CHOH-CH3}}  
    \end{reponses}  
    \explain{Le propan-2-ol a une chaîne de trois carbones avec le groupe hydroxyle en position 2. La formule correcte est donc \ce{CH3-CHOH-CH3}.}  
  \end{question}  
}

\element{nomc}{
  \begin{question}{nomen3}  
    Quel est le nom de la molécule \ce{CH3-CH(OH)-CH2-CH3} ?  
    \begin{reponses}  
      \mauvaise{Butan-1-ol}  
      \mauvaise{Pentan-2-ol}  
      \mauvaise{Ethan-1-ol}  
      \bonne{Butan-2-ol}  
    \end{reponses}  
    \explain{La chaîne principale contient quatre carbones et le groupe hydroxyle (\ce{-OH}) est en position 2. Le nom correct est donc butan-2-ol.}  
  \end{question}  
}

\element{nomc}{  
    \begin{question}{nomen4}  
    Quelle est la formule du butan-1-ol ?  
    \begin{reponses}  
      \mauvaise{\ce{CH3-CH2-CHOH-CH3}}  
      \mauvaise{\ce{CH3-CH2-CH2-OH}}  
      \mauvaise{\ce{HO-CH2-CH2-CH2-CH3}}  
      \bonne{\ce{CH3-CH2-CH2-CH2-OH}}  
    \end{reponses}  
    \explain{Le butan-1-ol a une chaîne de quatre carbones avec le groupe hydroxyle en position 1. La formule correcte est donc \ce{CH3-CH2-CH2-CH2-OH}.}  
  \end{question}  
}

\element{nomc}{
  \begin{question}{nomen5}  
    Quel est le nom de la molécule \ce{CH3-CH2-OH} ?  
    \begin{reponses}  
      \mauvaise{Propan-1-ol}  
      \mauvaise{Ethan-2-ol}  
      \mauvaise{Méthanol}  
      \bonne{Ethan-1-ol}  
    \end{reponses}  
    \explain{La chaîne principale contient deux carbones et le groupe hydroxyle (\ce{-OH}) est en position 1. Le nom correct est donc éthan-1-ol.}  
  \end{question}  
}

\element{nomc}{
  \begin{question}{nomen6}  
    Quelle est la formule du pentan-3-ol ?  
    \begin{reponses}  
      \mauvaise{\ce{CH3-CHOH-CH2-CH2-CH3}}  
      \mauvaise{\ce{HO-CH2-CH2-CH2-CH2-CH3}}  
      \mauvaise{\ce{CH3-CH2-OH-CH2-CH3}}  
      \bonne{\ce{CH3-CHOH-CH2-CH2-CH3}}  
    \end{reponses}  
    \explain{Le pentan-3-ol a une chaîne de cinq carbones avec le groupe hydroxyle en position 3. La formule correcte est donc \ce{CH3-CHOH-CH2-CH2-CH3}.}  
  \end{question}  
}

\element{nomc}{
  \begin{question}{nomen7}  
    Quel est le nom de la molécule \ce{HO-CH2-CH2-CH2-CH2-CH3} ?  
    \begin{reponses}  
      \mauvaise{Pentan-2-ol}  
      \mauvaise{Butan-1-ol}  
      \mauvaise{Ethan-1-ol}  
      \bonne{Pentan-1-ol}  
    \end{reponses}  
    \explain{La chaîne principale contient cinq carbones et le groupe hydroxyle (\ce{-OH}) est en position 1. Le nom correct est donc pentan-1-ol.}  
  \end{question}  
}

\element{nomc}{
  \begin{question}{nomen8}  
    Quelle est la formule du propan-1-ol ?  
    \begin{reponses}  
      \mauvaise{\ce{CH3-CHOH-CH3}}  
      \mauvaise{\ce{CH3-CH2-OH}}  
      \mauvaise{\ce{HO-CH2-CH2-CH3}}  
      \bonne{\ce{CH3-CH2-OH}}  
    \end{reponses}  
    \explain{Le propan-1-ol a une chaîne de trois carbones avec le groupe hydroxyle en position 1. La formule correcte est donc \ce{CH3-CH2-OH}.}  
  \end{question}  
}
