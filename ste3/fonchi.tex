\element{fonchi}{  
  \begin{question}{hydrox1}  
    Quel est le nom du groupe \ce{-OH} ?  
    \begin{reponses}  
      \mauvaise{groupe carbonyle}  
      \mauvaise{groupe carboxyle}  
      \mauvaise{groupe amine}  
      \bonne{groupe hydroxyle}  
    \end{reponses}  
    \explain{Le groupe \ce{-OH} est appelé groupe hydroxyle. Il est caractéristique des alcools et des phénols.}  
  \end{question}  
}

\element{fonchi}{  
    \begin{question}{hydrox2}  
    Quelle est la structure chimique du groupe hydroxyle ?  
    \begin{reponses}  
      \mauvaise{\ce{-C=O}}  
      \mauvaise{\ce{-COOH}}  
      \mauvaise{\ce{-NH2}}  
      \bonne{\ce{-OH}}  
    \end{reponses}  
    \explain{Le groupe hydroxyle se compose d'un atome d'oxygène covalent lié à un atome d'hydrogène (\ce{-O-H}).}  
  \end{question}  
}  

\element{fonchi}{
    \begin{question}{carbon1}  
    Quel est le nom du groupe \ce{-C=O} ?  
    \begin{reponses}  
      \mauvaise{groupe hydroxyle}  
      \mauvaise{groupe carboxyle}  
      \mauvaise{groupe amine}  
      \bonne{groupe carbonyle}  
    \end{reponses}  
    \explain{Le groupe \ce{-C=O} est appelé groupe carbonyle. Il est caractéristique des aldéhydes et des céttones.}  
  \end{question}  
}

\element{fonchi}{
    \begin{question}{carbon2}  
    Quelle est la structure chimique du groupe carbonyle ?  
    \begin{reponses}  
      \mauvaise{\ce{-OH}}  
      \mauvaise{\ce{-COOH}}  
      \mauvaise{\ce{-NH2}}  
      \bonne{\ce{-C=O}}  
    \end{reponses}  
    \explain{Le groupe carbonyle se compose d'un atome de carbone doublement lié à un atome d'oxygène (\ce{-C=O}).}  
  \end{question}  
}

\element{fonchi}{
    \begin{question}{carbon3}  
    Quelle est la différence entre un aldéhyde et une cétone ?  
    \begin{reponses}  
      \mauvaise{L'aldéhyde a un groupe hydroxyle, tandis que la cétone a un groupe carbonyle.}  
      \mauvaise{L'aldéhyde est un acide, tandis que la cétone est une base.}  
      \mauvaise{L'aldéhyde a un groupe amine, tandis que la cétone a un groupe carboxyle.}  
      \bonne{L'aldéhyde a le groupe carbonyle à l'extrémité de la chaîne carbonée, tandis que la cétone l'a dans la chaîne.}  
    \end{reponses}  
    \explain{L'aldéhyde a le groupe carbonyle (\ce{-C=O}) à l'extrémité de la chaîne carbonée, tandis que la cétone l'a dans la chaîne.}  
  \end{question}  
}  

\element{fonchi}{  
  \begin{question}{carbox1}  
    Quel est le nom du groupe \ce{-COOH} ?  
    \begin{reponses}  
      \mauvaise{groupe hydroxyle}  
      \mauvaise{groupe carbonyle}  
      \mauvaise{groupe amine}  
      \bonne{groupe carboxyle}  
    \end{reponses}  
    \explain{Le groupe \ce{-COOH} est appelé groupe carboxyle. Il est caractéristique des acides carboxyliques.}  
  \end{question}  
}

\element{fonchi}{ 
  \begin{question}{carbox2}  
    Quelle est la structure chimique du groupe carboxyle ?  
    \begin{reponses}  
      \mauvaise{\ce{-OH}}  
      \mauvaise{\ce{-C=O}}  
      \mauvaise{\ce{-NH2}}  
      \bonne{\ce{-COOH}}  
    \end{reponses}  
    \explain{Le groupe carboxyle se compose d'un atome de carbone doublement lié à un atome d'oxygène et d'un groupe hydroxyle (\ce{-C(=O)-OH}).}  
  \end{question}  
}  

\element{fonchi}{
  \begin{question}{amine1}  
    Quel est le nom du groupe \ce{-NH2} ?  
    \begin{reponses}  
      \mauvaise{groupe hydroxyle}  
      \mauvaise{groupe carbonyle}  
      \mauvaise{groupe carboxyle}  
      \bonne{groupe amine}  
    \end{reponses}  
    \explain{Le groupe \ce{-NH2} est appelé groupe amine. Il est caractéristique des amines.}  
  \end{question}  
}

\element{fonchi}{
  \begin{question}{amine2}  
    Quelle est la structure chimique du groupe amine ?  
    \begin{reponses}  
      \mauvaise{\ce{-OH}}  
      \mauvaise{\ce{-C=O}}  
      \mauvaise{\ce{-COOH}}  
      \bonne{\ce{-NH2}}  
    \end{reponses}  
    \explain{Le groupe amine se compose d'un atome de azote covalent lié à deux atomes d'hydrogène (\ce{-N-H2}).}  
  \end{question}  
}

\element{fonchi}{
  \begin{question}{amine3}  
    Quelles sont les différentes formes des amines ?  
    \begin{reponses}  
      \mauvaise{Amine primaire, amine secondaire et amine tertiaire.}  
      \mauvaise{Amine aliphatique et amine aromatique.}  
      \mauvaise{Amine acide et amine base.}  
      \bonne{Amine primaire, amine secondaire et amine tertiaire.}  
    \end{reponses}  
    \explain{Les amines peuvent être primaires (un seul groupe alkyle lié au azote), secondaires (deux groupes alkyles liés au azote) ou tertiaires (trois groupes alkyles liés au 
azote).}  
  \end{question}  
}
