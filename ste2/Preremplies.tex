\documentclass[10pt,a4paper]{article}
\usepackage[a4paper, margin=2cm]{geometry}
\usepackage{needspace}
\usepackage[nobottomtitles]{titlesec}
\usepackage{multicol}
\usepackage{xcolor}
\usepackage{fp}
\usepackage{xfp}
\usepackage{numprint}
\usepackage{siunitx}
\usepackage{enumitem}
\usepackage[version=4]{mhchem}
\usepackage{WriteOnGrid}
\usepackage{frcursive}
\usepackage{csvsimple}%
\usepackage[francais,bloc]{automultiplechoice}

\DeclareSIUnit{\nothing}{\relax}

\FPseed=10

\baremeDefautS{e=0,v=0,b=1,m=-.1}
\baremeDefautM{e=0,v=0,b=.5,m=-.1}

\graphicspath{ {./images/} }

\newenvironment{reponsesd}{
    \begin{multicols}{2}
    \begin{reponses} }{
    \end{reponses}
    \end{multicols}
}

\setlength{\columnseprule}{1pt}
\def\columnseprulecolor{\color{lightgray}}%

\let\oexplain\explain
\renewcommand{\explain}[1]{\oexplain{\textcolor{red}{#1}}}

\titleformat{\section}
  {\centering\hrule\vspace{2mm}}
  {\thesection}
  {1em}
  {}
  [\vspace{1mm}\hrule]

\titleformat{\subsection}
  {\em}
  {\thesubsection}
  {1em}
  {}

\newcommand{\sujet}{
\exemplaire{1}{%

\AMCsetFoot{\niveau{} \classe{} -- \prenom{}~\nom{} -- \thepage}

%%% debut de l'en-tête des copies :
\begin{center}
\vfill
\noindent{\large \bf Classe de \niveau{}°\classe{}}

\vspace*{3mm}

{\Large\bf Évaluation de compétences (non notée) STL - Physique-Chime 05/11/2025}

\vfill
\namefield{\noindent{}\fbox{\vspace*{3mm}
         \Huge\bf\prenom{}~\nom{}\normalsize{}%
         \vspace*{3mm}
      }}
\end{center}
\vfill

\begin{center}
\textbf{Durée : 5 minutes.}
\vspace*{5mm}

  Aucun document n'est autorisé.
  L'usage de la calculatrice est interdit.

  Les questions faisant apparaitre le symbole \multiSymbole{} peuvent
  présenter une ou plusieurs bonnes réponses. Les autres ont
  une unique bonne réponse.

  {\color{white} Des points négatifs seront affectés aux mauvaises réponses.}
  \vspace*{5mm}

  \textbf{IMPORTANT: Utilisez un crayon de papier bien noir pour cocher les cases, et une gomme
  pour effacer délicatement en cas d'erreur. Ne raturez pas les cases.
  Si vous effacez le pourtour de la case, ne le redessinez pas!}

  \vspace*{5mm}

  Ne pas faire comme ceci (raturé, pas centré, trop pâle, ...):\\
  \includegraphics[width=4cm]{checkbox_bad.png}

  Mais comme cela (bien centré, bien foncé):\\
  \includegraphics[width=2cm]{checkbox_good.png}

  Si vous faites une erreur et que vous ne pouvez effacer, raturez la case bien ostensiblement.

\end{center}
\vspace{1ex}
\vfill
\pagebreak
%%% fin de l'en-tête

\restituegroupe{groupes}


\AMCassociation{\id}

	  } % End \exemplaire{1}{%
} % End \newcommand{\sujet}{


%%%%§§§§§§§§§§§§§§§§§§§§§§§§§§§§§§§§§
\newcommand{\allq}{}
\newcommand{\nball}[1]{#1}

\begin{document}
%%%Options
\AMCrandomseed{10}

\def\AMCformQuestion#1{{\sc Question #1 :}}

\setdefaultgroupmode{withoutreplacement}
%%% Fin Options

%%% elements

\element{penti}{
  \begin{question}{penti1}
    Quelle est la valeur de \(2^3\)?
    \begin{reponses}
      \mauvaise{$6$}
      \mauvaise{$4$}
      \mauvaise{$10$}
      \bonne{$8$}
    \end{reponses}
    \explain{\(2^3 = 2 \times 2 \times 2 = 8\)}
  \end{question}
}

\element{penti}{
  \begin{question}{penti2}
    Quelle est la valeur de \(2^{-2}\)?
    \begin{reponses}
      \mauvaise{$\frac{1}{8}$}
      \mauvaise{$-4$}
      \mauvaise{$\frac{1}{2}$}
      \bonne{\(\frac{1}{4}\)}
    \end{reponses}
    \explain{\(2^{-2} = \frac{1}{2^2} = \frac{1}{4}\)}
  \end{question}
}

\element{penti}{
  \begin{question}{penti3}
    Quelle est la valeur de \(a^0\) pour tout \(a \neq 0\)?
    \begin{reponses}
      \mauvaise{$0$}
      \mauvaise{$-1$}
      \mauvaise{Indéfini}
      \bonne{$1$}
    \end{reponses}
    \explain{Pour tout \(a \neq 0\), \(a^0 = 1\)}
  \end{question}
}

\element{penti}{
  \begin{question}{penti4}
    Quelle est la valeur de $3^{-2}$?
    \begin{reponses}
      \mauvaise{$-2$}
      \mauvaise{$9$}
      \mauvaise{$\frac{-1}{9}$}
      \bonne{$\frac{1}{9}$}
    \end{reponses}
    \explain{Il s'agit de la règle générale des opérations avec les exponents}
  \end{question}
}


\element{p10}{
  \begin{question}{p101}
    Quelle est la valeur de \(10^3\)?
    \begin{reponses}
      \mauvaise{\numprint{100}}
      \mauvaise{\numprint{10000}}
      \mauvaise{\numprint{1000000}}
      \bonne{\numprint{1000}}
    \end{reponses}
    \explain{\(10^3 = 1000\)}
  \end{question}
}

\element{p10}{
  \begin{question}{p102}
    Quelle est la valeur de \(10^{-2}\)?
    \begin{reponses}
      \mauvaise{\numprint{100}}
      \mauvaise{\numprint{-2}}
      \mauvaise{Un nombre imaginaire}
      \bonne{\numprint{0,01}}
    \end{reponses}
    \explain{\(10^{-2} = 0,01\)}
  \end{question}
}

\element{p10}{
  \begin{question}{p103}
    Comment écrire \numprint{0,001} sous forme de puissance de 10?
    \begin{reponses}
      \mauvaise{$10^3$}
      \mauvaise{$-10^2$}
      \mauvaise{C'est impossible}
      \bonne{$10^{-3}$}
    \end{reponses}
    \explain{\(0,001 = 10^{-3}\)}
  \end{question}
}

\element{p10}{
  \begin{question}{p104}
    Que vaut $\frac{10^2}{10^{-2}}$ ?
    \begin{reponses}
      \mauvaise{$0$}
      \mauvaise{$1$}
      \mauvaise{Indéfini}
      \bonne{$10^4$}
    \end{reponses}
    \explain{\(\frac{10^2}{10^{-2}} = 10^4\)}
  \end{question}
}

\element{p10}{
  \begin{question}{p105}
    Quelle est la valeur de $10^4$?
    \begin{reponses}
      \mauvaise{\numprint{100}}
      \mauvaise{\numprint{1000}}
      \mauvaise{\numprint{1000000}}
      \bonne{\numprint{10000}}
    \end{reponses}
    \explain{$10^3 = 10000$}
  \end{question}
}

\element{p10}{
  \begin{question}{p106}
    Quelle est la valeur de $10^{-3}$?
    \begin{reponses}
      \mauvaise{\numprint{1000}}
      \mauvaise{\numprint{-3}}
      \mauvaise{\numprint{0,01}}
      \bonne{\numprint{0,001}}
    \end{reponses}
    \explain{$10^{-2} = 0,01$}
  \end{question}
}

\element{p10}{
  \begin{question}{p107}
    Comment écrire \numprint{0,0001} sous forme de puissance de 10?
    \begin{reponses}
      \mauvaise{$10^4$}
      \mauvaise{$-10^3$}
      \mauvaise{$10^{-3}$}
      \bonne{$10^{-4}$}
    \end{reponses}
    \explain{\numprint{0,0001} = $10^{-4}$}
  \end{question}
}

\element{p10}{
  \begin{question}{p108}
    Que vaut $\frac{10^3}{10^{-2}}$ ?
    \begin{reponses}
      \mauvaise{$0$}
      \mauvaise{$1$}
      \mauvaise{$10^4$}
      \bonne{$10^5$}
    \end{reponses}
    \explain{$\frac{10^3}{10^{-2}} = 10^5$}
  \end{question}
}


\element{psi}{
  \begin{question}{psi1}
    Quel est le symbole du préfixe représentant \(10^3\)~?
    \begin{reponses}
      \mauvaise{\si{\mega\nothing}}
      \mauvaise{\si{\giga\nothing}}
      \mauvaise{\si{\tera\nothing}}
      \bonne{\si{\kilo\nothing}}
    \end{reponses}
    \explain{Le symbole est \si{\kilo\nothing}}
  \end{question}
}

\element{psi}{
  \begin{question}{psi2}
    Quelle est la valeur de \qty{1}{\micro\second}~?
    \begin{reponses}
      \mauvaise{\qty{1 e6}{\second}}
      \mauvaise{\qty{1 e-3}{\second}}
      \mauvaise{\qty{1 e3}{\second}}
      \bonne{\qty{1 e-6}{\second}}
    \end{reponses}
    \explain{La valeur de \qty{1}{\micro\second} est \qty{1 e-6}{\second}}
  \end{question}
}

\element{psi}{
  \begin{question}{psi3}
    Quel préfixe correspond à \(10^{-6}\)~?
    \begin{reponses}
      \mauvaise{\si{\milli\nothing}}
      \mauvaise{\si{\kilo\nothing}}
      \mauvaise{\si{\nano\nothing}}
      \bonne{\si{\micro\nothing}}
    \end{reponses}
    \explain{Le préfixe est \si{\micro\nothing}}
  \end{question}
}

\element{psi}{
  \begin{question}{psi4}
    Quelle quantité en unités de base correspond à \qty{5}{\kilo\metre}~?
    \begin{reponses}
      \mauvaise{\qty{5}{\metre}}
      \mauvaise{\qty{5}{\milli\metre}}
      \mauvaise{\qty{500}{\metre}}
      \bonne{\qty{5000}{\metre}}
    \end{reponses}
    \explain{\(5 \, \si{\kilo\metre} = 5 000 \, \text{m}\)}
  \end{question}
}


% Composition d'un atome
\element{atcomp}{
  \begin{question}{atcomp1}
    Quelles sont les deux principales parties constitutives d'un atome~?
    \begin{reponses}
      \mauvaise{Électrons et photons}
      \mauvaise{Neutrons et protons}
      \mauvaise{Protons et électrons}
      \bonne{Noyau et électrons}
    \end{reponses}
    \explain{Un atome est constitué d'un noyau central et d'électrons en orbite autour de celui-ci.}
  \end{question}
}

\element{atcomp}{
  \begin{question}{atcomp2}
    Qu'est-ce qu'un nucléon~?
    \begin{reponses}
      \mauvaise{Un électron}
      \mauvaise{Un photon}
      \mauvaise{Une particule virtuelle}
      \bonne{Un proton ou un neutron}
    \end{reponses}
    \explain{Les nucléons sont les constituants du noyau atomique : protons et neutrons.}
  \end{question}
}

% Composition des noyaux
\element{atcomp}{
  \begin{question}{atcomp3}
    Quelles particules constituent principalement un noyau atomique~?
    \begin{reponses}
      \mauvaise{Électrons et neutrons}
      \mauvaise{Protons et photons}
      \mauvaise{Neutrons et photons}
      \bonne{Protons et neutrons}
    \end{reponses}
    \explain{Le noyau atomique est composé de protons et de neutrons.}
  \end{question}
}

% Ordre de grandeur
\element{atcomp}{
  \begin{question}{atcomp4}
    Quelle est l'ordre de grandeur d'un atome~?
    \begin{reponses}
      \mauvaise{$10^{-12}$ mètres}
      \mauvaise{$10^{-6}$ mètres}
      \mauvaise{$10^{-9}$ mètres}
      \bonne{$10^{-10}$ mètres}
    \end{reponses}
    \explain{La taille d'un atome est de l'ordre de $10^{-10}$ mètres.}
  \end{question}
}

\element{atcomp}{
  \begin{question}{atcomp5}
    Quelle est approximativement la taille d'un noyau atomique par rapport à un atome~?
    \begin{reponses}
      \mauvaise{$10^{10}$ fois plus petit}
      \mauvaise{$10^3$ fois plus petit}
      \mauvaise{$10$ fois plus petit}
      \bonne{$10^5$ fois plus petit}
    \end{reponses}
    \explain{Un noyau atomique est environ $10^5$ fois plus petit qu'un atome.}
  \end{question}
}


% Symboles chimiques
\element{sym}{
  \begin{question}{sym1}
    Quel est le symbole chimique du carbone ?
    \begin{reponses}
      \mauvaise{\ce{H}}
      \mauvaise{\ce{O}}
      \mauvaise{\ce{N}}
      \bonne{\ce{C}}
    \end{reponses}
    \explain{Le symbole chimique du carbone est \ce{C}.}
  \end{question}
}

\element{sym}{
  \begin{question}{sym2}
    Quel est le symbole chimique de l'oxygène ?
    \begin{reponses}
      \mauvaise{\ce{C}}
      \mauvaise{\ce{H}}
      \mauvaise{\ce{N}}
      \bonne{\ce{O}}
    \end{reponses}
    \explain{Le symbole chimique de l'oxygène est \ce{O}.}
  \end{question}
}

\element{sym}{
  \begin{question}{sym3}
    Quel est le symbole chimique de l'azote ?
    \begin{reponses}
      \mauvaise{\ce{C}}
      \mauvaise{\ce{H}}
      \mauvaise{\ce{O}}
      \bonne{\ce{N}}
    \end{reponses}
    \explain{Le symbole chimique de l'azote est \ce{N}.}
  \end{question}
}

\element{sym}{
  \begin{question}{sym4}
    Quel est le symbole chimique de l'hydrogène ?
    \begin{reponses}
      \mauvaise{\ce{C}}
      \mauvaise{\ce{O}}
      \mauvaise{\ce{N}}
      \bonne{\ce{H}}
    \end{reponses}
    \explain{Le symbole chimique de l'hydrogène est \ce{H}.}
  \end{question}
}

% Notation des symboles chimiques
\element{iso}{
  \begin{question}{iso1}
    Dans le symbole \ce{^A_ZX}, qu'est-ce que représente Z ?
    \begin{reponses}
      \mauvaise{Le nombre de neutrons}
      \mauvaise{Le nombre de nucléons}
      \mauvaise{Le numéro de masse}
      \bonne{Le nombre de protons}
    \end{reponses}
    \explain{Dans le symbole \ce{^A_ZX}, Z représente le nombre de protons, qui correspond au numéro atomique de l'élément.}
  \end{question}
}

\element{iso}{
  \begin{question}{iso2}
    Dans le symbole \ce{^A_ZX}, qu'est-ce que représente A ?
    \begin{reponses}
      \mauvaise{Le nombre de protons}
      \mauvaise{Le nombre de neutrons}
      \mauvaise{La charge électrique de l'atome}
      \bonne{Le nombre de nucléons}
    \end{reponses}
    \explain{Dans le symbole \ce{^A_ZX}, A représente le nombre total de nucléons, c'est-à-dire la somme des protons et des neutrons.}
  \end{question}
}

\element{iso}{
  \begin{question}{iso3}
    Dans le symbole \ce{^A_ZX}, comment peut-on déterminer le nombre de neutrons ?
    \begin{reponses}
      \mauvaise{En additionnant Z et A}
      \mauvaise{En soustrayant A de Z (Z-A)}
      \mauvaise{En multipliant Z par A}
      \bonne{En soustrayant Z de A (A-Z)}
    \end{reponses}
    \explain{Le nombre de neutrons est obtenu en soustrayant le nombre de protons (Z) du nombre total de nucléons (A).}
  \end{question}
}

\element{iso}{
  \begin{question}{iso4}
    Dans le symbole \ce{^14_6C}, combien y a-t-il de neutrons ?
    \begin{reponses}
      \mauvaise{6}
      \mauvaise{10}
      \mauvaise{12}
      \bonne{8}
    \end{reponses}
    \explain{Le nombre de protons Z est 6, et le nombre de nucléons A est 14. Donc, le nombre de neutrons est 14 - 6 = 8.}
  \end{question}
}


%%% fin des elements
\element{groupes}{
%\section{Notion de puissances}
\begin{multicols}{2}
\insertgroup[2]{penti}
\insertgroup[2]{p10}
%\end{multicols}

%\section{Préfixes SI (Système International des Unités)}
%\begin{multicols}{2}
\insertgroup[2]{psi}
%\end{multicols}

%\section{Atomes}
%\begin{multicols}{2}
\insertgroup[2]{atcomp}
%\end{multicols}

%\section{Chimie}
%\begin{multicols}{2}
\insertgroup[2]{sym}
\insertgroup[2]{iso}
\end{multicols}


\AMCcleardoublepage
}

\csvreader[head to column names]{liste.csv}{}{\sujet}

\end{document}
