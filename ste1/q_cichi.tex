
\element{cichi}{
  \begin{question}{cichi2}
    Qu'est-ce que la constante de vitesse ($k$) représente dans une équation chimique ?
    \begin{reponses}
      \mauvaise{C'est la concentration initiale du réactif.}
      \mauvaise{C'est le temps nécessaire pour que la réaction se termine.}
      \mauvaise{C'est la température à laquelle la réaction se produit.}
      \bonne{C'est un facteur de proportionnalité qui dépend de la température et de la nature des réactifs.}
    \end{reponses}
    \explain{La constante de vitesse ($k$) est un facteur dans la loi de vitesse qui reflète combien vite la réaction se déroule dans des conditions spécifiques, influencé par des facteurs tels que la température et les espèces chimiques impliquées.}
  \end{question}
}

\element{cichi}{
  \begin{question}{cichi4}
    Quelle est l'expression de la demi-vie ($t_{\frac{1}{2}}$) pour une réaction du premier ordre (avec $[A]_0$ concentration initiale) ?
    \begin{reponses}
      \mauvaise{$t_{\frac{1}{2}} = [A]_0 / k$}
      \mauvaise{$t_{\frac{1}{2}} = 1 / (k [A]_0)$}
      \mauvaise{$t_{\frac{1}{2}} = k \times [A]_0$}
      \bonne{$t_{1/2} = \frac{\ln(2)}{k}$, et il est indépendant de la concentration initiale.}
    \end{reponses}
    \explain{Pour les réactions du premier ordre, la demi-vie est donnée par $t_{1/2} = \frac{\ln(2)}{k}$ et ne dépend pas de la concentration initiale du réactif.}
  \end{question}
}
