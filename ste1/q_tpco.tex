
\element{tpco}{
  \begin{questionmult}{tpco1}
    Quelle est la principale fonction d'une solution tampon ?
    \begin{reponses}
      \mauvaise{Neutraliser complètement les acides et les bases forts.}
      \mauvaise{Agir comme catalyseur dans les réactions chimiques.}
      \mauvaise{Augmenter la conductivité électrique de l'eau.}
      \bonne{Résister aux changements de pH lorsqu'on ajoute de petites quantités
      d'acide ou de base.}
    \end{reponses}
    \explain{Les solutions tampon résistent aux variations du pH 
    en neutralisant les ions ajoutés grâce à leurs composants acido-basiques.}
  \end{questionmult}
}

\element{tpco}{
  \begin{questionmult}{tpco2}
    Comment les solutions tampon sont-elles généralement préparées ?
    \begin{reponses}
      \mauvaise{En mélangeant un acide fort avec une base forte.}
      \mauvaise{En ayant un certificat dûment tamponné.}
      \mauvaise{En dissolvant un solide dans l'eau pour créer une solution saturée.}
      \mauvaise{En chauffant un mélange d'acide et de base jusqu'à ébullition.}
      \bonne{En mélangeant un acide faible avec sa base conjuguée ou une base 
      faible avec son acide conjugué.}
    \end{reponses}
    \explain{Les solutions tampon sont préparées en combinant un 
    acide faible et sa base conjuguée.}
  \end{questionmult}
}

\element{tpco}{
  \begin{questionmult}{tpco3}
    Quel est le rôle du système de tampon bicarbonaté dans le sang ?
    \begin{reponses}
      \mauvaise{Réguler la température corporelle.}
      \mauvaise{Réguler la pression du sang.}
      \mauvaise{Transporter l'oxygène dans tout le corps.}
      \mauvaise{Digérer les aliments dans le sang.}
      \bonne{Maintenir le pH sanguin en neutralisant les ions 
      hydrogène en excès grâce à l'ion bicarbonate.}
    \end{reponses}
    \explain{Le système bicarbonaté aide à maintenir un pH stable
    dans le sang en neutralisant les excès d'acide ou de base.}
  \end{questionmult}
}

\element{tpco}{
  \begin{questionmult}{tpco4}
    Que se passe-t-il lorsque le CO2 se dissout dans l'eau ?
    \begin{reponses}
      \mauvaise{Il forme un acide fort qui se dissocie complètement.}
      \mauvaise{Il ne réagit pas et reste sous forme de molécules de \ce{CO2}.}
      \mauvaise{Il ne retrouve plus sa maison.}
      \mauvaise{Il réagit avec l'eau pour produire de l'oxygène.}
      \bonne{Il se dissout partiellement pour former de l'acide carbonique,
       qui se dissocie ensuite en ions \ce{H+} et bicarbonate (\ce{HCO3-}).}
    \end{reponses}
    \explain{Le \ce{CO2} se dissous réagit avec l'eau pour former une petite 
    quantité d'acide carbonique, lequel se dissocie en ions \ce{H+} et \ce{HCO3-}.}
  \end{questionmult}
}

\element{tpco}{
  \begin{questionmult}{tpco5}
    Comment les solutions tampon maintiennent-elles la stabilité du 
    pH lorsque des acides ou des bases sont ajoutés ?
    \begin{reponses}
      \mauvaise{Elles neutralisent complètement l'acide ou la base ajouté.}
      \mauvaise{Elles jettent les acides et les bases par-dessus bord.}
      \mauvaise{Elles augmentent leur concentration pour dominer la substance ajoutée.}
      \mauvaise{Elles changent de couleur pour indiquer les variations du pH.}
      \bonne{Elles réagissent avec l'acide ou 
      la base ajouté pour consommer les ions \ce{H+} ou \ce{OH-} en excès.}
    \end{reponses}
    \explain{Les solutions tampon neutralisent les ajouts d'acide 
    ou de base en réagissant avec les ions \ce{H+} ou \ce{OH-} en excédentaires.}
  \end{questionmult}
}

\element{tpco}{
  \begin{questionmult}{tpco6}
    Pourquoi le système de tampon bicarbonaté est-il important dans les systèmes biologiques ?
    \begin{reponses}
      \mauvaise{Il est essentiel pour la photosynthèse chez les plantes.}
      \mauvaise{Il amortit les chocs lors d'une chutte.}
      \mauvaise{Il facilite la digestion des aliments dans l'estomac.}
      \mauvaise{Il est impliqué dans la transmission des signaux nerveux.}
      \bonne{Il aide à maintenir un pH stable dans le sang et 
      d'autres fluides corporels.}
    \end{reponses}
    \explain{Le système bicarbonaté est essentiel pour maintenir 
    l'équilibre acido-basique nécessaire aux processus biologiques vitaux.}
  \end{questionmult}
}

