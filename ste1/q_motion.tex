\element{motion}{
  \begin{question}{motion1}
    Qu'est-ce que la force électrostatique~?
    \begin{reponses}
      \mauvaise{Une force qui pousse les objets vers le bas.}
      \mauvaise{Une force qui fait tourner les objets.}
      \mauvaise{Une force qui fait cuire les aliments.}
      \bonne{Une force qui attire ou repousse les objets chargés électriquement.}
    \end{reponses}
    \explain{La force électrostatique est une interaction entre particules chargées, décrite par la loi de Coulomb. Elle peut attirer ou repousser selon les charges.}
  \end{question}
}

\element{motion}{
  \begin{question}{motion2}
    Qu'est-ce que le champ électrostatique~?
    \begin{reponses}
      \mauvaise{Un endroit où les particules ne s'approchent jamais.}
      \mauvaise{Une région où les champs magnétiques dominent.}
      \mauvaise{Un lieu idéal pour piqueniquer avec des pizzas.}
      \bonne{ Une zone autour d'un objet chargé où la force électrostatique peut être détectée.}
    \end{reponses}
    \explain{Le champ électrostatique représente la distribution de forces autour d'un objet chargé, mesuré en newtons par coulomb.}
  \end{question}
}

\element{motion}{
  \begin{question}{motion3}
    Pourquoi est-il important de dresser un bilan des forces~?
    \begin{reponses}
      \mauvaise{Pour savoir combien de force il faut pour soulever une montagne.}
      \mauvaise{Pour mesurer la résistance de l'air dans les tunnels.}
      \mauvaise{Pour vérifier si les objets sont vraiment inanimés.}
      \bonne{Pour déterminer les forces qui agissent sur un objet et leur équilibre.}
    \end{reponses}
    \explain{Le bilan des forces permet de comprendre comment les interactions physiques affectent un objet, essentiel pour analyser ses mouvements ou son équilibre.}
  \end{question}
}

\element{motion}{
  \begin{question}{motion4}
    Quelle est la première loi de Newton~?
    \begin{reponses}
      \mauvaise{Un objet en mouvement arrête de se déplacer sans force extérieure.}
      \mauvaise{La gravité attire tout vers le centre de la Terre.}
      \mauvaise{Tout ce qui est lisse ne peut rouiller.}
      \bonne{Sans force extérieure, un objet au repos reste au repos, et un objet en mouvement uniforme continue.}
    \end{reponses}
    \explain{La première loi de Newton décrit l'inertie, montrant que les objets maintiennent leur état de mouvement ou de repos à moins qu'une force ne les modifie.}
  \end{question}
}

\element{motion}{
  \begin{question}{motion5}
    Comment la chute verticale avec frottement visqueux affecte-t-elle un objet~?
    \begin{reponses}
      \mauvaise{En augmentant sa vitesse jusqu'à ce qu'il explose.}
      \mauvaise{En le faisant disparaître dans l'air.}
      \mauvaise{En le ralentissant progressivement jusqu'à l'arrêt.}
      \bonne{En exerçant une force de frottement proportionnelle à sa vitesse.}
    \end{reponses}
    \explain{Le frottement visqueux oppose la motion, proportionnellement à la vitesse, ralentissant l'objet et l'empêchant d'accélérer indéfiniment.}
  \end{question}
}

\element{motion}{
  \begin{question}{motion6}
    Qu'est-ce que le régime permanent~?
    \begin{reponses}
      \mauvaise{Un état où tout est en mouvement sans arrêt.}
      \mauvaise{Une situation où les forces s'équilibrent.}
      \mauvaise{Un moment où l'univers entier est en paix.}
      \bonne{ Une situation où la vitesse de l'objet devient constante malgré des forces opposées.}
    \end{reponses}
    \explain{Au régime permanent, les forces en jeu s'équilibrent, conduisant à une vitesse constante sans accélération.}
  \end{question}
}

\element{motion}{
  \begin{question}{motion7}
    Qu'est-ce que la vitesse en régime permanent~?
    \begin{reponses}
      \mauvaise{La vitesse maximale qu'un objet peut atteindre.}
      \mauvaise{La vitesse minimum pour éviter de tomber.}
      \mauvaise{La vitesse idéale pour cuisiner des œufs.}
      \bonne{ La vitesse constante atteinte quand la force motrice équilibre la résistance.}
    \end{reponses}
    \explain{En régime permanent, l'objet se déplace à une vitesse constante où la somme des forces est nulle, équilibrant poussée et frottement.}
  \end{question}
}
