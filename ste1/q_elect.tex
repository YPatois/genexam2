\element{elect}{
  \begin{question}{elect1}
    Qu'est-ce que la Loi des Nœuds~?
    \begin{reponses}
      \mauvaise{La somme des courants dans un circuit est égale à la résistance.}
      \mauvaise{Le courant net dans un nœud est nul.}
      \mauvaise{La tension et le courant sont inversement liés dans un nœud.}
      \bonne{La somme des courants entrants dans un nœud est égale à la somme des courants sortants.}
    \end{reponses}
    \explain{La Loi des Nœuds de Kirchhoff stipule que le courant total entrant dans un nœud doit être égal au courant total sortant, assurant ainsi la conservation de la charge électrique.}
    \end{question}
  }

\element{elect}{
  \begin{question}{elect2}
    Qu'est-ce que la Loi des Mailles~?
    \begin{reponses}
      \mauvaise{La somme de toutes les tensions dans une boucle est égale à la résistance multipliée par le courant.}
      \mauvaise{La somme des courants autour d'une boucle fermée est nulle.}
      \mauvaise{Les sources de tension s'annulent dans une boucle de circuit.}
      \bonne{La somme des variations de tension autour d'une boucle fermée est nulle.}
    \end{reponses}
    \explain{La Loi des Mailles de Kirchhoff stipule que la somme algébrique des tensions autour d'un chemin fermé doit être nulle, assurant ainsi la conservation de l'énergie.}
    \end{question}
  }

\element{elect}{
  \begin{question}{elect3}
    Qu'est-ce que la Loi d'Ohm ?
    \begin{reponses}
      \mauvaise{La puissance est égale au courant carré multiplié par la résistance (P=I²R).}
      \mauvaise{La tension est égale à la résistance plus le courant (U=R+I).}
      \mauvaise{Le courant est égal à la tension multipliée par la résistance (I=UR).}
      \bonne{La tension est égale au courant multiplié par la résistance (U=RI).}
    \end{reponses}
    \explain{La Loi d'Ohm stipule que U=RI, reliant ainsi la tension, le courant et la résistance dans un circuit électrique.}
    \end{question}
  }

\element{elect}{
  \begin{question}{elect4}
    Comment est calculée la puissance électrique dans un circuit ?
    \begin{reponses}
      \mauvaise{La puissance est égale à la tension carrée (P=V²).}
      \mauvaise{La puissance est égale au courant carré divisé par la résistance (P=I²/R).}
      \mauvaise{La puissance est égale à la résistance au carré multipliée par le courant (P=R²I).}
      \bonne{La puissance est égale à la tension multipliée par le courant (P=UI).}
    \end{reponses}
    \explain{La puissance électrique est calculée à l'aide de P=UI, où V est la tension et I est le courant.}
    \end{question}
  }


\element{elect}{
  \begin{question}{elect5}
    Qu'est-ce que l'Effet Joule ?
    \begin{reponses}
      \mauvaise{Il décrit la conservation de l'énergie dans les circuits.}
      \mauvaise{Il est lié à l'intensité des champs magnétiques.}
      \mauvaise{Il provoque un court-circuit.}
      \bonne{Il fait référence à la chaleur générée par le courant électrique.}
    \end{reponses}
    \explain{L'Effet Joule est le chauffage d'un conducteur dû au courant électrique, calculé par P=RI² ou P=U²/R.}
    \end{question}
  }

