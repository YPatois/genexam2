\element{radio}{
  \begin{question}{radio1}
    Qu'est-ce qu'une émission alpha ($\alpha$)~?
    \begin{reponses}
      \mauvaise{Une émission alpha consiste en l'éjection d'un photon gamma haute énergie.}
      \mauvaise{Une émission alpha est la libération d'un neutron excité.}
      \mauvaise{Une émission alpha se produit lors de la capture d'un électron par un proton.}
      \bonne{Une émission alpha est le dégagement d'un noyau d'hélium (2 protons et 2 neutrons).}
    \end{reponses}
    \explain{Une émission alpha implique l'éjection d'un noyau d'hélium. Les autres options décrivent d'autres types de radiation.}
  \end{question}
}

\element{radio}{
  \begin{question}{radio2}
    Quelles sont les principales différences entre une émission bêta moins ($\beta^-$) et une émission bêta plus ($\beta^+$) ?
    \begin{reponses}
      \mauvaise{Une émission $\beta^-$ libère un positron, tandis qu'une émission $\beta^+$ libère un neutron.}
      \mauvaise{Une émission $\beta^-$ se produit dans des noyaux riches en neutrons.}
      \mauvaise{Une émission $\beta^-$ est moins pénétrante qu'une émission $\beta^+$.}
      \bonne{ Une émission $\beta^-$ libère un électron, tandis qu'une émission $\beta^+$ libère un positron.}
    \end{reponses}
    \explain{L'émission $\beta^-$ libère un électron et $\beta^+$ libère un positron (son antiparticule).}
  \end{question}
}

\element{radio}{
  \begin{question}{radio3}
    Qu'est-ce qu'une émission gamma ($\gamma$), et dans quel contexte se produit-elle ?
    \begin{reponses}
      \mauvaise{ Une émission $\gamma$ est l'éjection d'un noyau d'hélium.}
      \mauvaise{ Une émission $\gamma$ se produit lors de la capture d'un électron par un proton.}
      \mauvaise{ Une émission $\gamma$ libère un neutron excité.}
      \bonne{Une émission $\gamma$ est l'émission d'un photon haute énergie suite à une transition nucléaire.}
    \end{reponses}
    \explain{L'émission $\gamma$ implique l'émission de photons haute énergie lors d'une transition nucléaire.}
  \end{question}
}

\element{radio}{
  \begin{question}{radio4}
    Quelle loi physique est conservée lors d'une désintégration radioactive ?
    \begin{reponses}
      \mauvaise{ La masse atomique totale.}
      \mauvaise{ Le spin des particules émises.}
      \mauvaise{ L'énergie cinétique des produits de désintégration.}
      \bonne{ Le nombre de nucléions et l'énergie totale sont conservés.}
    \end{reponses}
    \explain{Lors d'une désintégration radioactive, le nombre de nucléons et l'énergie totale sont conservés. }
  \end{question}
}

\element{radio}{
  \begin{question}{radio5}
    Comment évolue la population moyenne d’un ensemble de noyaux radioactifs au fil du temps ?
    \begin{reponses}
      \mauvaise{ Elle augmente exponentiellement.}
      \mauvaise{ Elle reste constante.}
      \mauvaise{ Elle suit une décroissance logarithmique.}
      \bonne{ Elle diminue suivant une loi exponentielle.}
    \end{reponses}
    \explain{La population de noyaux radioactifs diminue suivant une décroissance exponentielle.}
  \end{question}
}

\element{radio}{
  \begin{question}{radio6}
    Qu'est-ce que la constante de désintégration $\lambda$, et quelle unité a-t-elle ?
    \begin{reponses}
      \mauvaise{ C'est la vitesse de décroissance en noyaux par seconde.}
      \mauvaise{ C'est la probabilité de désintégration par seconde, en pourcentage.}
      \mauvaise{ C'est la masse critique d'un matériau.}
      \bonne{ C'est la probabilité de désintégration par seconde.}
    \end{reponses}
    \explain{$\lambda$ représente la probabilité de désintégration par unité de temps (unité : $\text{s}^{-1}$).}
  \end{question}
}

\element{radio}{
  \begin{question}{radio7}
    Qu'est-ce que le temps de demi-vie d'un isotope radioactif ?
    \begin{reponses}
      \mauvaise{ Le temps de demi-vie est la durée pendant laquelle l'isotope est dangereux.}
      \mauvaise{ Le temps de demi-vie est la période après laquelle l'isotope disparaît complètement.}
      \mauvaise{ Le temps de demi-vie est une mesure de radioactivité.}
      \bonne{ Le temps de demi-vie est la durée après laquelle 50\% des noyaux radioactifs ont subi une désintégration.}
    \end{reponses}
    \explain{La demi-vie est la durée après laquelle 50\% des noyaux radioactifs ont subi une désintégration.}
  \end{question}
}

\element{radio}{
  \begin{question}{radio8}
    Qu'est-ce que l'activité d'un échantillon radioactif ?
    \begin{reponses}
      \mauvaise{ La quantité de rayonnement $\gamma$ émis.}
      \mauvaise{ Le nombre total de désintégrations depuis la formation de l'échantillon.}
      \mauvaise{ La dose de radiation reçue par un être humain à proximité.}
      \bonne{ Le nombre de désintégrations par seconde dans l'échantillon.}
    \end{reponses}
    \explain{L'activité correspond au nombre de désintégrations par unité de temps.}
  \end{question}
}
