\element{acide}{
  \begin{questionmult}{acide1}
    Qu'est-ce qu'une constante d'équilibre acido-basique ?
    \begin{reponses}
      \mauvaise{Un paramètre qui mesure la vitesse d'un réaction chimique.}
      \mauvaise{Une valeur qui indique la température à laquelle une substance se décompose.}
      \mauvaise{C'est une unité de mesure pour les échelles de pH.}
      \bonne{Une constante qui décrit l'équilibre entre une espèce acide et 
      sa base conjuguée en solution aqueuse.}
    \end{reponses}
    \explain{La constante d'équilibre acido-basique est utilisée pour 
    quantifier la force relative d'un acide ou d'une base.}
  \end{questionmult}
}

\element{acide}{
  \begin{questionmult}{acide2}
    Qu'est-ce que le pKa, et à quoi sert-il ?
    \begin{reponses}
      \mauvaise{C'est un indicateur de la basicité d'une substance.}
      \mauvaise{C'est une mesure de la solubilité d'un composé dans l'eau.}
      \mauvaise{C'est l'abréviation pour 'pourquoi Kévin aime les acides'.}
      \bonne{C'est le logarithme négatif de la constante de dissociation acide, 
      servant à indiquer la force de l'acide.}
    \end{reponses}
    \explain{Un pKa élevé correspond à un faible acide, tandis qu'un pKa bas 
    indique un acide fort.}
  \end{questionmult}
}

\element{acide}{
  \begin{questionmult}{acide3}
    Comment le coefficient de dissociation est-il lié à la force d'un acide ?
    \begin{reponses}
      \mauvaise{Il n'y a pas de relation directe entre les deux.}
      \mauvaise{Un grand coefficient de dissociation indique un acide faible.}
      \mauvaise{Il est utilisé pour mesurer la quantité d'électricité dans une solution.}
      \bonne{Un grand coefficient de dissociation signifie que l'acide
      se dissocie plus facilement, ce qui en fait un acide fort.}
    \end{reponses}
    \explain{Le coefficient de dissociation reflète la proportion dans
    laquelle l'acide se sépare en ions en solution. Plus il est élevé, plus l'acide est fort.}
  \end{questionmult}
}

\element{acide}{
  \begin{questionmult}{acide4}
    Si un acide a un pKa de 5, qu'en déduisons-nous ?
    \begin{reponses}
      \mauvaise{Il s'agit d'un acide fort.}
      \mauvaise{Il est incapable de réagir avec une base.}
      \mauvaise{C'est un acide qui ne réagit qu'avec des bases naturelles.}
      \bonne{C'est un acide modérément faible}
    \end{reponses}
    \explain{Un pKa moyen comme 5 indique que l'acide n'est que très partiellement dissocié,
    ce qui est typique des acides faibles.}
  \end{questionmult}
}

\element{acide}{
  \begin{questionmult}{acide5}
    Quelle relation existe-t-il entre le pKa et la force d'un acide ?
    \begin{reponses}
      \mauvaise{Plus le pKa est élevé, plus l'acide est fort.}
      \mauvaise{Il n'y a pas de relation directe.}
      \mauvaise{Plus le pKa est élevé, plus l'acide est faible.}
      \bonne{Plus le pKa est bas, plus l'acide est fort.}
    \end{reponses}
    \explain{Un pKa bas signifie que l'acide se dissocie plus 
    facilement, ce qui en fait un acide fort.}
  \end{questionmult}
}

\element{acide}{
  \begin{questionmult}{acide6}
    Pourquoi les acides faibles ont-ils un coefficient de dissociation inférieur à 1 ?
    \begin{reponses}
      \mauvaise{Parce qu'ils ne sont pas capables de se dissocier.}
      \mauvaise{Parce que leur pKa est trop élevé.}
      \mauvaise{Parce qu'ils sont timides et n'osent pas se dissocier complètement.}
      \bonne{Parce qu'ils ne se dissocient pas complètement en solution aqueuse}
    \end{reponses}
    \explain{Les acides faibles partiellement se dissociie, donc leur coefficient
     de dissociation reste inférieur à 1.}
  \end{questionmult}
}
