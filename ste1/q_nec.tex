\element{nec}{
  \begin{question}{nec1}
    À quoi servent les règles de Cahn, Ingold et Prelog en nomenclature chimique ?
    \begin{reponses}
      \mauvaise{Déterminer la conformation d'une molécule}
      \mauvaise{Établir les propriétés physiques d'un composé}
      \mauvaise{Déterminer si une molécule est chirale}
      \bonne{Donner la priorité aux substituents pour nommer les composés}
    \end{reponses}
    \explain{Les règles de Cahn-Ingold-Prelog permettent d'attribuer un ordre de priorité aux substituents afin de nommer correctement les composés chimiques.}
  \end{question}
}

\element{nec}{
  \begin{question}{nec2}
    Selon les règles de Cahn-Ingold-Prelog, comment détermine-t-on la priorité des substituents ?
    \begin{reponses}
      \mauvaise{En fonction du nombre de lettres dans le nom du substituant}
      \mauvaise{En fonction de la taille du substituant}
      \mauvaise{En fonction de la charge électrique du substituant}
      \bonne{En fonction du numéro atomique du premier atome du substituant}
    \end{reponses}
    \explain{La priorité est déterminée par le numéro atomique du premier atome du substituant. Si nécessaire, on examine les atomes suivants dans le substituant.}
  \end{question}
}

\element{nec}{
  \begin{question}{nec3}
    Qu'est-ce que la configuration R/S d'un centre chiral ?
    \begin{reponses}
      \mauvaise{Une classification des molécules en fonction de leur poids moléculaire}
      \mauvaise{Un système de désignation des isomères Z/E}
      \mauvaise{Une échelle de mesure de la polarité d'une molécule}
      \bonne{Un moyen de décrire l'arrangement spatial des substituents autour d'un atome}
    \end{reponses}
    \explain{La configuration R/S décrit l'arrangement tridimensionnel des quatre substituants différents autour d'un atome de carbone asymétrique.}
  \end{question}
}

\element{nec}{
  \begin{question}{nec4}
    Qu'est-ce que les désignations Z et E pour une double liaison ?
    \begin{reponses}
      \mauvaise{Elles indiquent si la molécule est droite ou gauche}
      \mauvaise{Elles classifient les molécules en fonction de leur solubilité}
      \mauvaise{Elles indiquent si la molécule est liquide ou gazeuse}
      \bonne{Elles décrivent la répartition des substituents de haute priorité autour d'une double liaison}
    \end{reponses}
    \explain{Z (zusammen) signifie que les substituents de plus haute priorité sont du même côté, tandis que E (entgegen) signifie qu'ils sont des côtés opposés.}
  \end{question}
}
