\element{echi}{
  \begin{question}{echi1}
    Qu’est-ce qu’un diagramme d’état d’un corps pur ?
    \begin{reponses}
      \mauvaise{Un graphique montrant l’évolution des prix d’un corps sur le marché.}
      \mauvaise{Un schéma représentant les différents états d’humeur d’une personne.}
      \mauvaise{Un tableau périodique des éléments chimiques.}
      \bonne{LUn graphique des différentes phases en fonction de la température et de la pression.}
    \end{reponses}
    \explain{Un diagramme d’état montre les conditions de température et de pression 
    auxquelles un corps pur se trouve sous différentes phases (solide, liquide ou gazeux). 
    Il est essentiel pour comprendre les transitions de phase.}
  \end{question}
}

\element{echi}{
  \begin{question}{echi2}
    Comment déterminez-vous le point triple sur un diagramme d’état ?
    \begin{reponses}
      \mauvaise{En suivant la ligne des températures négatives.}
      \mauvaise{En trouvant l’intersection des axes X et Y.}
      \mauvaise{En comptant les étoiles sur le graphique.}
      \bonne{À l’intersection des trois courbes de fusion, sublimation et condensation.}
    \end{reponses}
    \explain{Le point triple est le point spécifique sur un diagramme d’état où les pressions 
    et températures permettent aux trois phases (solide, liquide et gazeux) de coexister en équilibre.}
  \end{question}
}

\element{echi}{
  \begin{question}{echi3}
    Définissez l’enthalpie de changement d’état.
    \begin{reponses}
      \mauvaise{L’énergie libérée par une réaction chimique à pression constante.}
      \mauvaise{La quantité de chaleur perdue par un corps en mouvement.}
      \mauvaise{Un type de cuisson à basse température.}
      \bonne{L'énergie transférées lors d’un changement d'état (à T° constante).}
    \end{reponses}
    \explain{L’enthalpie de changement d’état représente la quantité de chaleur échangée 
    lorsqu’une substance change de phase à température constante, comme la fusion de la glace en eau.}
  \end{question}
}

\element{echi}{
  \begin{question}{echi4}
    Comment calculez-vous la chaleur requise pour un changement d’état en utilisant une enthalpie de changement d'état (en J/kg)?
    \begin{reponses}
      \mauvaise{En multipliant cette enthalpie par le volume du corps.}
      \mauvaise{En divisant cette enthalpie par la température ambiante.}
      \mauvaise{En soustrayant cette enthalpie de celle de la température initiale.}
      \bonne{En multipliant la masse du corps par cette enthalpie.}
    \end{reponses}
    \explain{La chaleur requise pour un changement d’état est calculée en multipliant 
    la masse de la substance par l'enthalpie de changement d'état du processus.}
  \end{question}
}

\element{echi}{
  \begin{question}{echi5}
    Définissez l’enthalpie standard de formation.
    \begin{reponses}
      \mauvaise{L’énergie libérée par la combustion d’un combustible.}
      \mauvaise{La quantité de chaleur perdue lors du refroidissement d’une substance.}
      \mauvaise{Un type de réaction chimique exothermique.}
      \bonne{L’énergie échangée pour former un composé à partir de ses éléments constitutifs.}
    \end{reponses}
    \explain{L’enthalpie standard de formation est la variation d’enthalpie qui accompagne 
    la formation d’un mole d’un composé à partir de ses éléments constitutifs dans leurs états 
    standards, sous pression atmospérique.}
  \end{question}
}

\element{echi}{
  \begin{question}{echi6}
    Comment utilisez-vous la loi de Hess pour calculer l’enthalpie standard de réaction ?
    \begin{reponses}
      \mauvaise{En additionnant toutes les enthalpies de formation des réactifs.}
      \mauvaise{En soustrayant l’enthalpie des produits de celle des réactifs.}
      \mauvaise{En multipliant l’enthalpie par le nombre de moles.}
      \bonne{En soustrayant la somme des enthalpies de formation des 
      réactifs de celle des produits.}
    \end{reponses}
    \explain{La loi de Hess permet de calculer l’enthalpie standard d’une réaction en soustrayant 
    la somme des enthalpies standard de formation des réactifs de celle des produits.}
  \end{question}
}

\element{echi}{
  \begin{question}{echi7}
    Qu’est-ce que la capacité thermique~?
    \begin{reponses}
      \mauvaise{La capacité à résister aux hautes températures.}
      \mauvaise{La quantité de chaleur requise pour chauffer un corps sans changement de phase.}
      \mauvaise{Un type de thermostat pour réguler la température.}
      \bonne{La quantité de chaleur nécessaire pour élever la température d’un corps.}
    \end{reponses}
    \explain{La capacité thermique représente la chaleur requise pour modifier la température d’une unité de masse d'un corps. 
    Elle est exprimée en J/K.}
  \end{question}
}

