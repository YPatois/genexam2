\element{energ}{
  \begin{question}{energ1}
    Qu'est-ce qu'une chaîne énergétique ?
    \begin{reponses}
      \mauvaise{Un ensemble de moteurs qui se relayent pour éviter la fatigue.}
      \mauvaise{Des outils utilisés en cuisine pour découper les légumes.}
      \mauvaise{Une série de transformations où l'énergie se multiplie à chaque étape.}
      \bonne{ Une succession d'étapes où l'énergie est transférée et transformée dans un système.}
    \end{reponses}
    \explain{Une chaîne énergétique décrit les différents processus par lesquels l'énergie est transférée et transformée à travers un système, de sa source à ses utilisations finales.}
  \end{question}
}

\element{energ}{
  \begin{question}{energ2}
    Quelle est la loi fondamentale de la conservation de l’énergie ?
    \begin{reponses}
      \mauvaise{L'énergie se crée et se détruit constamment dans la nature.}
      \mauvaise{La vitesse de l'énergie augmente avec la distance.}
      \mauvaise{Toute énergie finit par disparaître complètement.}
      \bonne{ La quantité totale d'énergie dans un système isolé reste constante, bien qu'elle puisse changer de forme.}
    \end{reponses}
    \explain{La conservation de l'énergie indique que l'énergie ne peut être créée ni détruite, seulement transformée ou transférée d'une forme à une autre.}
  \end{question}
}

\element{energ}{
  \begin{question}{energ3}
    Qu'est-ce que le rendement énergétique ?
    \begin{reponses}
      \mauvaise{Le rapport entre l'énergie perdue et l'énergie utile.}
      \mauvaise{La vitesse à laquelle l'énergie est transférée dans un système.}
      \mauvaise{Un type de transformation énergétique sans perte.}
      \bonne{ Le rapport entre l'énergie utile produite et l'énergie totale consommée.}
    \end{reponses}
    \explain{Le rendement énergétique est la proportion d'énergie utilisée de manière efficace par rapport à l'énergie totale disponible, souvent exprimé en pourcentage : $\eta = \frac{E_{utile}}{E_{totale}}$.}
  \end{question}
}

\element{energ}{
  \begin{question}{energ4}
    %\needspace{12cm}
    Qu’est-ce que la dissipation énergétique ?
    \begin{reponses}
      \mauvaise{Le processus par lequel l'énergie est stockée dans un système.}
      \mauvaise{La transformation totale de l'énergie en travail utile.}
      \mauvaise{ Une technique pour augmenter l'efficacité énergétique.}
      \bonne{ La perte d'énergie sous forme de chaleur ou de mouvement aléatoire, généralement due à la résistance ou au frottement.}
    \end{reponses}
    \explain{La dissipation énergétique correspond aux pertes d'énergie qui ne peuvent être récupérées ni utilisées pour un travail utile, souvent sous forme de chaleur.}
  \end{question}
}

\element{energ}{
  %\needspace{12cm}
  \begin{question}{energ5}
    %\needspace{12cm}
    Qu'est-ce qu’un transfert thermique ?
    \begin{reponses}
      \mauvaise{Un processus où la chaleur est générée mais ne se déplace jamais.}
      \mauvaise{ Une méthode pour stocker l'énergie chimique dans les aliments.}
      \mauvaise{Un type de mouvement des particules quantiques.}
      \bonne{ Le passage de la chaleur d'un corps à un autre ou d'une partie à une autre du même corps.}
    \end{reponses}
    \explain{Le transfert thermique se produit lorsque de la chaleur est échangée entre deux systèmes ou parties d'un système à des températures différentes.}
  \end{question}
}

\element{energ}{
  \begin{question}{energ6}
    Quelles sont les principales formes de transfert thermique ?
    \begin{reponses}
      \mauvaise{La conduction, la convection et la conversion.}
      \mauvaise{La radiation, l'isolation et la réflexion.}
      \mauvaise{La conduction, la convection et la créativité.}
      \bonne{ La conduction, la convection et la radiation.}
    \end{reponses}
    \explain{Les trois principales formes de transfert thermique sont la conduction (d'un corps à un autre en contact direct), la convection (via un fluide) et la radiation (par ondes électromagnétiques).}
  \end{question}
}


