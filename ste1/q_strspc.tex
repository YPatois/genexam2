\element{strspc}{
  \begin{questionmult}{strspc1}
    Une molécule est dite chirale si~:
    \begin{reponses}
      \mauvaise{ses atomes sont alignés}
      \mauvaise{elle est superposable à son image dans un miroir}
      \mauvaise{ses énantiomères sont diastéréoisométriques}
      \bonne{elle n'est pas superposable à son image dans un miroir}
    \end{reponses}
    \explain{un composé est dit chiral s'il n'est pas superposable à son 
    image dans un miroir plan.}
  \end{questionmult}
}

\element{strspc}{
  \begin{questionmult}{strspc2}
    La conformation d'une molécule se défini par~:
    \begin{reponses}
      \mauvaise{ses caractéristiques isomérique}
      \mauvaise{son respect des spécifications de son espèce chimique}
      \mauvaise{la qualité de son image dans un miroir plan}
      \bonne{les arangement des atomes tournant autour de liaisons simples}
    \end{reponses}
    \explain{La conformation d'une molécule correspond à l'orientation des atomes autour
    de liaisons simples. Celle-ci pouvant la plupart du temps tourner librement, 
    le même composé chimique peut passer de une à l'autre facilement.}
  \end{questionmult}
  }

\element{strspc}{
  \begin{questionmult}{strspc3}
    La configuration d'une molécule est~:
    \begin{reponses}
      \mauvaise{la version de système d'exploitation installé par défaut}
      \mauvaise{les arangement des atomes tournant autour de liaisons simples}
      \mauvaise{son image dans un miroir plan}
      \bonne{la disposition de ses atomes dans l'espace indépendamment des rotations
      autour des liaisons simples}
    \end{reponses}
    \explain{La configuration d'une molécule correspond à l'orientation des atomes dans l'espace,
    indépendamment des rotations autour des liaisons simples. Pour changer de configuration,
    il faut casser des liaisons pour en reformer d'autres.}
  \end{questionmult}
}

\element{strspc}{
  \begin {questionmult}{strspc4}
    Deux molécules sont des stéréo-isomères chimiques, si~:
    \begin{reponses}
      \mauvaise{elles ont une formule brute différente mais les mêmes propriétés chimiques}
      \mauvaise{elles ont la même conformation}
      \mauvaise{elles ont la meme configuration mais une formule semi-développée differente}
      \bonne{elles ont même formule semi-développée mais une configuration différente}
    \end{reponses}
    \explain{Les stéroisomères chimiques sont des molécules ayant la même formule
    semi-développée mais une conformation ou une configuration différente.}
  \end {questionmult}
}

\element{strspc}{
  \begin {questionmult}{strspc5}
    Deux molécules sont des énantiomères chimiques, si~:
    \begin{reponses}
      \mauvaise{elles ont même formule semi-développée mais ne sont pas images de l'une de l'autre dans un miroir plan}
      \mauvaise{elles ont la même mère}
      \mauvaise{la rotation de leur image dans un miroir dessine une ligne de symétrie}
      \bonne{elles sont images de l'une de l'autre dans un miroir plan}
    \end{reponses}
    \explain{Les enantiomères chimiques sont des molécules qui sont images de l'une de
    l'autre dans un miroir plan.}
  \end {questionmult}
}

\element{strspc}{
  \begin {questionmult}{strspc6}
    Deux molécules sont des diastéréoisométriques chimiques, si~:
    \begin{reponses}
      \mauvaise{elles ont une formule brute differente mais sont images de l'une de l'autre dans un miroir plan}
      \mauvaise{elles sont très très malades}
      \mauvaise{elles sont images de l'une de l'autre dans un miroir plan}
      \bonne{elles ont même formule semi-développée mais ne sont pas images de l'une de l'autre dans un miroir plan}
    \end{reponses}
    \explain{Les diastéréoisométriques .}
  \end {questionmult}
}
