
\element{meca}{
  \begin{question}{meca1}
    Qu'est-ce que le travail élémentaire d'une force ?
    \begin{reponses}
      \mauvaise{Le travail effectué par une force pour déplacer un objet.}
      \mauvaise{L'énergie potentielle stockée dans un champ de force.}
      \mauvaise{La quantité de café nécessaire pour bouger un objet lourd.}
      \bonne{L'énergie transférée lorsqu'une force fait déplacer un objet sur une distance.}
    \end{reponses}
    \explain{Le travail élémentaire d'une force correspond à l'énergie transférée 
    lorsque la force agit sur un objet et le déplace. Cela se calcule comme 
    $W = F \cdot d \cdot \cos(\theta)$, où $F$ est la force, $d$ est la distance et $\theta$ l'angle entre la force 
et le déplacement.}
  \end{question}
}


\element{meca}{
  \begin{question}{meca2}
    Qu'est-ce que le travail du poids ?
    \begin{reponses}
      \mauvaise{Le travail effectué par une personne pour soulever des haltères.}
      \mauvaise{L'énergie cinétique d'un objet en mouvement.}
      \mauvaise{La force nécessaire pour voler comme un super-héros.}
      \bonne{L'énergie transférée par la gravité lorsque l'objet change de hauteur.}
    \end{reponses}
    \explain{Le travail du poids est le travail effectué par la force gravitationnelle, 
    $W = m \cdot g \cdot h$, où $m$ est la masse, $g$ l'accélération gravitationnelle et $h$ la hauteur.}
  \end{question}
}

\element{meca}{
  \begin{question}{meca3}
    Qu'est-ce que l'énergie potentielle de pesanteur ?
    \begin{reponses}
      \mauvaise{L'énergie utilisée pour courir sur une pente.}
      \mauvaise{L'énergie stockée dans un ressort comprimé.}
      \mauvaise{La quantité d'énergie nécessaire pour ne plus jamais boire de café.}
      \bonne{L'énergie associée à la présence d'un objet dans un champ gravitationnel.}
    \end{reponses}
    \explain{L'énergie potentielle de pesanteur est donnée par $E_p = m \cdot g \cdot h$, où $m$ est la masse, 
    $g$ l'accélération gravitationnelle et $h$ la hauteur. Elle représente l'énergie stockée due à la position de l'objet.}
  \end{question}
}

\element{meca}{
  \begin{question}{meca4}
    Qu'est-ce que l'énergie mécanique ?
    \begin{reponses}
      \mauvaise{L'énergie utilisée pour réfléchir à une question difficile.}
      \mauvaise{La somme de l'énergie cinétique et de l'énergie thermique.}
      \mauvaise{La quantité d'énergie nécessaire pour faire fonctionner une machine à remonter le temps.}
      \bonne{La somme de l'énergie cinétique et de l'énergie potentielle d'un système.}
    \end{reponses}
    \explain{L'énergie mécanique est la combinaison de l'énergie cinétique 
    ($E_c = \frac{1}{2}mv^2$) et de l'énergie potentielle ($E_p = mgh$), représentant le total de l'énergie dans un système.}
  \end{question}
}

\element{meca}{
  \begin{question}{meca5}
    Qu'est-ce que la conservation de l'énergie ?
    \begin{reponses}
      \mauvaise{L'idée que l'énergie peut être créée ou détruite.}
      \mauvaise{La capacité à stocker de l'énergie pour une utilisation future.}
      \mauvaise{Un concept qui permet de faire disparaître l'énergie lorsqu'elle n'est plus utile.}
      \bonne{Le principe selon lequel l'énergie ne peut être créée ni détruite, seulement transformée.}
    \end{reponses}
    \explain{La conservation de l'énergie signifie que la quantité totale d'énergie
    dans un système isolé reste constante, même si elle change de forme (par exemple,
    de l'énergie potentielle en énergie cinétique).}
  \end{question}
}

\element{meca}{
  \begin{question}{meca6}
    Comment la puissance et l'énergie sont-elles liées ?
    \begin{reponses}
      \mauvaise{La puissance est une forme d'énergie stockée.}
      \mauvaise{L'énergie est la puissance divisée par le temps.}
      \mauvaise{La puissance est la quantité d'énergie disponible pour faire du travail.}
      \bonne{La puissance exprime la vitesse à laquelle l'énergie est utilisée ou transférée.}
    \end{reponses}
    \explain{La puissance représente le taux d'utilisation ou de transfert d'énergie,
    tandis que l'énergie est la quantité totale d'énergie disponible.
    Par exemple, $P = \frac{E}{t}$ montre comment la puissance est liée à l'énergie et au temps.}
  \end{question}
}

\element{meca}{
  \begin{question}{meca7}
    Qu'est-ce que l'énergie mécanique disponible ?
    \begin{reponses}
      \mauvaise{L'énergie perdue sous forme de chaleur.}
      \mauvaise{L'énergie potentielle stockée dans un champ magnétique.}
      \mauvaise{La quantité d'énergie nécessaire pour faire fonctionner une machine à remonter le temps.}
      \bonne{L'énergie totale (cinétique et potentielle) qui peut être utilisée pour effectuer un travail.}
    \end{reponses}
    \explain{L'énergie mécanique disponible fait référence à l'énergie totale
    d'un système qui peut être utilisée pour accomplir un travail, y compris à la fois l'énergie cinétique et potentielle.}
  \end{question}
}

\element{meca}{
  \begin{question}{meca8}
    Pourquoi la conservation de l'énergie est-elle importante ?
    \begin{reponses}
      \mauvaise{Parce qu'elle permet de créer de l'énergie gratuitement.}
      \mauvaise{Parce qu'elle explique pourquoi nous devons manger pour avoir de l'énergie.}
      \mauvaise{Parce qu'elle prédit que l'univers finira par manquer d'énergie.}
      \bonne{Parce qu'elle décrit comment l'énergie se transforme mais ne disparaît pas, aidant à comprendre les systèmes physiques.}
    \end{reponses}
    \explain{La conservation de l'énergie est cruciale car elle explique que
    toute énergie reste présente sous une forme ou une autre dans un système isolé,
    permettant de prédire et d'étudier divers phénomènes naturels.}
  \end{question}
}

