\element{logs}{
    \begin{question}{logs1}
        Le logarithme décimal de $10^{-3}$ est :
        \begin{reponses}
            \mauvaise{$3$}
            \mauvaise{$0$}
            \mauvaise{$-1$}
            \bonne{$-3$}
        \end{reponses}
        \explain{Le logarithme décimal de $10^{-3}$ est $-3$ car $\log(10^{-3}) = -3 \cdot \log(10) = -3 \cdot 1 = -3$.}
    \end{question}
}

\element{logs}{
    \begin{question}{logs2}
        Si $\log x = 2$, alors $x$ est :
        \begin{reponses}
            \mauvaise{$0{,}01$}
            \mauvaise{$0{,}1$}
            \mauvaise{$1$}
            \bonne{$100$}
        \end{reponses}
        \explain{Si $\log x = 2$, alors $x = 10^2 = 100$.}
    \end{question}
}

\element{logs}{
    \begin{question}{logs3}
        $\log(ab)$ est égal à :
        \begin{reponses}
            \mauvaise{$\log a - \log b$}
            \mauvaise{$\log a \times \log b$}
            \mauvaise{$\frac{\log a}{\log b}$}
            \bonne{$\log a + \log b$}
        \end{reponses}
        \explain{La propriété des logarithmes stipule que $\log(ab) = \log a + \log b$.}
    \end{question}
}

\element{logs}{
    \begin{question}{logs4}
        $\log\left(\frac{a}{b}\right)$ est égal à :
        \begin{reponses}
            \mauvaise{$\log a + \log b$}
            \mauvaise{$\log a \times \log b$}
            \mauvaise{$\log(a \times b)$}
            \bonne{$\log a - \log b$}
        \end{reponses}
        \explain{La propriété des logarithmes stipule que $\log\left(\frac{a}{b}\right) = \log a - \log b$.}
    \end{question}
}

\element{logs}{
    \begin{question}{logs5}
        $\log(10^n)$ est égal à :
        \begin{reponses}
            \mauvaise{$1$}
            \mauvaise{$0$}
            \mauvaise{$10 \times n$}
            \bonne{$n$}
        \end{reponses}
        \explain{La propriété des logarithmes stipule que $\log(10^n) = n \cdot \log(10) = n$.}
    \end{question}
}

\element{logs}{
    \begin{question}{logs6}
        $\log(1)$ est égal à :
        \begin{reponses}
            \mauvaise{$10$}
            \mauvaise{$-1$}
            \mauvaise{$0{,}1$}
            \bonne{$0$}
        \end{reponses}
        \explain{Le logarithme de 1 est 0 car $10^0 = 1$.}
    \end{question}
}

\element{logs}{
    \begin{question}{logs7}
        Si $\log x = -1$, alors $x$ est :
        \begin{reponses}
            \mauvaise{$1$}
            \mauvaise{$10$}
            \mauvaise{$0{,}01$}
            \bonne{$0{,}1$}
        \end{reponses}
        \explain{Si $\log x = -1$, alors $x = 10^{-1} = 0{,}1$.}
    \end{question}
}

\element{logs}{
    \begin{question}{logs8}
        $\log(10^{-5})$ est égal à :
        \begin{reponses}
            \mauvaise{$5$}
            \mauvaise{$1$}
            \mauvaise{$0$}
            \bonne{$-5$}
        \end{reponses}
        \explain{Le logarithme de $10^{-5}$ est $-5$ car $\log(10^{-5}) = -5 \cdot \log(10) = -5$.}
    \end{question}
}
