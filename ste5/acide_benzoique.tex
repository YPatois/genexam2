\documentclass[12pt]{article}
\usepackage[utf8]{inputenc}
\usepackage[T1]{fontenc}
\usepackage[french]{babel}
\usepackage{amsmath}
\usepackage{WriteOnGrid}
\usepackage{siunitx}
\usepackage[version=4]{mhchem}
\usepackage{chemfig}
\usepackage[francais,bloc]{automultiplechoice}

\title{Exercice : Extraction de l'acide benzoïque}
\author{}
\date{}

\begin{document}

\maketitle

\section*{Introduction}
L'acide benzoïque, de formule \ce{C6H5COOH}, et le benzoate de sodium sont des conservateurs antimicrobiens respectivement identifiés par les codes E210 et E211. Ils sont présents dans de nombreux produits alimentaires et notamment dans certaines boissons gazeuses sucrées.

\chemfig{
OH-[:120,,1](=[:60]O)-[:180]=_[:240]-[:180]=_[:120]-[:60]=_(-[:300])
}

\section*{Données}
\begin{itemize}
    \item Masse molaire de l'acide benzoïque : $M = 122 \, \mathrm{g\cdot mol^{-1}}$.
    \item Température de fusion de l'acide benzoïque : $\theta_f = 122,4 \, \mathrm{°C}$.
    \item pKa du couple acide benzoïque/ion benzoate : $pKa = 4,2$.
    \item Le benzoate de sodium est soluble dans l’eau.
\end{itemize}

\section*{Questions}

\begin{question}{ouverte1}
Quelle est la formule topologique de l'acide benzoïque ?
\begin{EnvQuadrillage}[NbCarreaux=21x2,Grille=Seyes,Marge=1]
\end{EnvQuadrillage}
\end{question}

\begin{question}{ouverte2}
Quels sont les groupes caractéristiques présents dans la formule de l'acide benzoïque ?
\begin{EnvQuadrillage}[NbCarreaux=21x2,Grille=Seyes,Marge=1]
\end{EnvQuadrillage}
\end{question}

\begin{question}{ouverte3}
Expliquer pourquoi l'acide benzoïque est peu soluble dans l'eau mais très soluble dans l'éther éthylique.
\begin{EnvQuadrillage}[NbCarreaux=21x2,Grille=Seyes,Marge=1]
\end{EnvQuadrillage}
\end{question}

\begin{question}{ouverte4}
Écrire l'équation de la réaction de l'acide benzoïque avec l'eau.
\begin{EnvQuadrillage}[NbCarreaux=21x2,Grille=Seyes,Marge=1]
\end{EnvQuadrillage}
\end{question}

\begin{question}{ouverte5}
Quelle est l'espèce prédominante du couple acide benzoïque/ion benzoate à un pH de 3 ?
\begin{EnvQuadrillage}[NbCarreaux=21x2,Grille=Seyes,Marge=1]
\end{EnvQuadrillage}
\end{question}

\end{document}
