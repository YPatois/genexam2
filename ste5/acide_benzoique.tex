
\section*{Introduction}
L'acide benzoïque, de formule \ce{C6H5COOH}, et le benzoate de sodium sont des conservateurs antimicrobiens 
respectivement identifiés par les codes E210 et E211. Ils sont présents dans de nombreux produits alimentaires 
et notamment dans certaines boissons gazeuses sucrées. À température ambiante, l’acide benzoïque est un solide blanc.

\textbf{Source du sujet :} Baccalauréat de mars 2023 – exercice 4 du sujet de Métropole.

\begin{figure}[h]
    \centering
    \chemfig{OH-[:120,,1](=[:60]O)-[:180]=_[:240]-[:180]=_[:120]-[:60]=_(-[:300])}
    \caption{Formule topologique de l’acide benzoïque}
\end{figure}

\textbf{Données :}
\begin{itemize}
    \item Masse molaire de l'acide benzoïque : $M = \SI{122}{\gram\per\mole}$.
    \item Température de fusion de l'acide benzoïque : $\theta_f = \SI{122,4}{\degreeCelsius}$.
    \item pKa du couple acide benzoïque/ion benzoate : $pKa = 4,2$.
    \item Le benzoate de sodium est soluble dans l’eau.
\end{itemize}

\begin{table}[h]
\centering
\begin{tabular}{|c|c|c|c|c|}
\hline
 & Densité & Eau & Éthanol & Éther éthylique \\
\hline
Acide benzoïque & - & Peu soluble & Très soluble & Très soluble \\
\hline
Eau & 1,0 & - & Miscible & Non miscible \\
\hline
Éthanol & 0,76 & Miscible & - & - \\
\hline
Éther éthylique & 0,71 & Non miscible & - & - \\
\hline
\end{tabular}
\caption{Tableau de solubilité des substances}
\end{table}

L’objectif de cet exercice est de vérifier l’indication d’une étiquette de boisson gazeuse concernant la
présence d’un conservateur.

\section*{Questions}

\begin{question}{acidbenzo1}
    Entourer et nommer le groupe caractéristique présents dans la formule topologique de l'acide benzoïque ci-dessous.\\
        \centering\chemfig{OH-[:120,,1](=[:60]O)-[:180]=_[:240]-[:180]=_[:120]-[:60]=_(-[:300])}
    \AMCOpen{lines=3}{
        \mauvaise[F]{Faux}\scoring{b=-1}
        \mauvaise[X]{Non réponse}\scoring{b=0}
        \bonne[I]{Groupe identifié}\scoring{b=.5}
        \bonne[N]{Groupe nommé}\scoring{b=.5}
    }
    \explain{Le groupes caractéristique est le groupe carboxyle (–COOH)}
\end{question}

\begin{question}{acidbenzo2}
Quels sont les groupes caractéristiques présents dans la formule de l'acide benzoïque ?
\begin{EnvQuadrillage}[NbCarreaux=20x4,Grille=Seyes,Marge=1]
\end{EnvQuadrillage}
\AMCOpen{lines=4}{\wrongchoice[F]{Faux}\scoring{0}\correctchoice[J]{Juste}\scoring{2}}
\explain{
Les groupes caractéristiques de l'acide benzoïque sont :
- Un groupe carboxyle (–COOH)
- Un noyau benzénique (cycle aromatique)
}
\end{question}

\begin{question}{ouverte3}
Expliquer pourquoi l'acide benzoïque est peu soluble dans l'eau mais très soluble dans l'éther éthylique.
\begin{EnvQuadrillage}[NbCarreaux=20x2,Grille=Seyes,Marge=1]
\end{EnvQuadrillage}
\AMCOpen{lines=2}{\wrongchoice[F]{Faux}\scoring{0}\correctchoice[J]{Juste}\scoring{2}}
\explain{
L'acide benzoïque est peu soluble dans l'eau en raison de son caractère hydrophobe (cycle benzénique) et de 
la formation de liaisons hydrogène intramoléculaires. En revanche, il est très soluble dans l'éther éthylique, 
un solvant organique apolaire, en raison de sa structure aromatique.
}
\end{question}

\begin{question}{ouverte4}
Écrire l'équation de la réaction de l'acide benzoïque avec l'eau.
\begin{EnvQuadrillage}[NbCarreaux=20x2,Grille=Seyes,Marge=1]
\end{EnvQuadrillage}
\AMCOpen{lines=2}{\wrongchoice[F]{Faux}\scoring{0}\correctchoice[J]{Juste}\scoring{2}}
\explain{
L'équation de la réaction est :
\ce{C6H5COOH + H2O <=> C6H5COO- + H3O+}
}
\end{question}

\begin{question}{ouverte5}
Quelle est l'espèce prédominante du couple acide benzoïque/ion benzoate à un pH de 3 ?
\begin{EnvQuadrillage}[NbCarreaux=20x2,Grille=Seyes,Marge=1]
\end{EnvQuadrillage}
\AMCOpen{lines=2}{\wrongchoice[F]{Faux}\scoring{0}\correctchoice[J]{Juste}\scoring{2}}
\explain{
À un pH de 3, l'espèce prédominante est l'acide benzoïque (\ce{C6H5COOH}), car le pH est inférieur 
au pKa du couple.
}
\end{question}
