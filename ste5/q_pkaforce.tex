
\element{pkaforce}{
    \begin{question}{pkaforce1}
        Le pKa est défini comme~:
        \begin{reponses}
            \mauvaise{$\log(\text{[acide]})$}
            \mauvaise{$\text{pH}_{\text{équivalence}}$}
            \mauvaise{$K_a$}
            \bonne{$\text{p}K_a = -\log(K_a)$}
        \end{reponses}
    \end{question}
}


\element{pkaforce}{
    \begin{question}{pkaforce2}
        Plus le pKa d'un acide est faible, plus~:
        \begin{reponses}
            \mauvaise{L'acide est faible}
            \mauvaise{L'acide est neutre}
            \mauvaise{L'acide est instable}
            \bonne{L'acide est fort}
        \end{reponses}
    \end{question}
}

\element{pkaforce}{
    \begin{question}{pkaforce3}
        Si le pKa d'un acide est 4, alors à pH 4~:
        \begin{reponses}
            \mauvaise{L'acide est entièrement dissocié}
            \mauvaise{La base est majoritaire}
            \mauvaise{L'acide est neutre}
            \bonne{Les formes acide et base sont en concentrations égales}
        \end{reponses}
    \end{question}
}

\element{pkaforce}{
    \begin{question}{pkaforce4}
        Un acide de pKa 2 est~:
        \begin{reponses}
            \mauvaise{Plus faible qu'un acide de pKa 5}
            \mauvaise{De même force qu'un acide de pKa 5}
            \mauvaise{Un acide très faible}
            \bonne{Plus fort qu'un acide de pKa 5}
        \end{reponses}
    \end{question}
}

\element{pkaforce}{
    \begin{question}{pkaforce5}
        La relation de Henderson-Hasselbalch est~:
        \begin{reponses}
            \mauvaise{$\ce{pH = pKa + log(\frac{[Acide]}{[Base]})}$}
            \mauvaise{$\ce{pH = pKa - [Acide]}$}
            \mauvaise{$\ce{pH = [Base] - [Acide]}$}
            \bonne{$\ce{pH = pKa + log(\frac{[Base]}{[Acide]})}$}
        \end{reponses}
    \end{question}
}

\element{pkaforce}{
    \begin{question}{pkaforce6}
        Si le pH d'une solution est égal au pKa de l'acide, alors~:
        \begin{reponses}
            \mauvaise{La solution est neutre}
            \mauvaise{L'acide est entièrement dissocié}
            \mauvaise{La base est absente}
            \bonne{Les concentrations de l'acide et de sa base conjuguée sont égales}
        \end{reponses}
    \end{question}
}

\element{pkaforce}{
    \begin{question}{pkaforce7}
        Un acide de pKa 9 est~:
        \begin{reponses}
            \mauvaise{Un acide fort}
            \mauvaise{Un acide très fort}
            \mauvaise{Un acide neutre}
            \bonne{Un acide faible}
        \end{reponses}
    \end{question}
}

\element{pkaforce}{
    \begin{question}{pkaforce8}
        Le pKa du couple $\ce{CH3COOH/CH3COO-}$ est 4,8. À pH 6,8, la forme prédominante est~:
        \begin{reponses}
            \mauvaise{$\ce{CH3COOH}$}
            \mauvaise{Les deux formes sont absentes}
            \mauvaise{Les deux formes sont en concentrations égales}
            \bonne{$\ce{CH3COO-}$}
        \end{reponses}
    \end{question}
}
