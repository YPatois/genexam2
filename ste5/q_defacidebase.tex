\element{defacidebase}{
    \begin{question}{defacidebase1}
        Selon Brønsted-Lowry, un acide est~:
        \begin{reponses}
            \mauvaise{Un accepteur de paires d'électrons}
            \mauvaise{Un donneur de paires d'électrons}
            \mauvaise{Un accepteur de protons}
            \bonne{Un donneur de protons}
        \end{reponses}
        \explain{Un acide, selon Brønsted-Lowry, est une espèce chimique capable de donner un proton $H^+$ à une autre espèce.}
    \end{question}
}

\element{defacidebase}{
    \begin{question}{defacidebase2}
        Selon Brønsted-Lowry, une base est~:
        \begin{reponses}
            \mauvaise{Un donneur de protons}
            \mauvaise{Un accepteur de paires d'électrons}
            \mauvaise{Un donneur d'électrons}
            \bonne{Un accepteur de protons}
        \end{reponses}
        \explain{Une base, selon Brønsted-Lowry, est une espèce chimique capable d'accepter un proton $H^+$ d'une autre espèce.}
    \end{question}
}

\element{defacidebase}{
    \begin{question}{defacidebase3}
        Dans la réaction \ce{NH3 + H2O -> NH4+ + OH-}, \ce{NH3} joue le rôle de~:
        \begin{reponses}
            \mauvaise{Acide}
            \mauvaise{Solvant}
            \mauvaise{Oxydant}
            \bonne{Base}
        \end{reponses}
        \explain{\ce{NH3} accepte un proton $H^+$ de \ce{H2O} pour former \ce{NH4+}, ce qui en fait une base selon Brønsted-Lowry.}
    \end{question}
}

\element{defacidebase}{
    \begin{question}{defacidebase4}
        Dans la réaction \ce{CH3COOH + H2O -> CH3COO- + H3O+}, \ce{CH3COOH} joue le rôle de~:
        \begin{reponses}
            \mauvaise{Base}
            \mauvaise{Solvant}
            \mauvaise{Réducteur}
            \bonne{Acide}
        \end{reponses}
        \explain{\ce{CH3COOH} donne un proton $H^+$ à \ce{H2O} pour former \ce{H3O+}, ce qui en fait un acide selon Brønsted-Lowry.}
    \end{question}
}

\element{defacidebase}{
    \begin{question}{defacidebase5}
        Un couple acide/base est défini par~:
        \begin{reponses}
            \mauvaise{Deux espèces chimiques de même charge}
            \mauvaise{Deux espèces chimiques de même masse molaire}
            \mauvaise{Deux espèces chimiques de même formule brute}
            \bonne{Deux espèces chimiques qui diffèrent par un proton}
        \end{reponses}
        \explain{Un couple acide/base est constitué de deux espèces chimiques qui se transforment l'une en l'autre par gain ou perte d'un proton $H^+$.}
    \end{question}
}

\element{defacidebase}{
    \begin{question}{defacidebase6}
        Dans le couple \ce{H3O+/H2O}, \ce{H3O+} est~:
        \begin{reponses}
            \mauvaise{La base}
            \mauvaise{Un solvant}
            \mauvaise{Un oxydant}
            \bonne{L'acide}
        \end{reponses}
        \explain{\ce{H3O+} est un acide car il peut donner un proton $H^+$ pour former \ce{H2O}.}
    \end{question}
}

\element{defacidebase}{
    \begin{question}{defacidebase7}
        Dans le couple \ce{NH4+/NH3}, \ce{NH3} est~:
        \begin{reponses}
            \mauvaise{L'acide}
            \mauvaise{Un solvant}
            \mauvaise{Un réducteur}
            \bonne{La base}
        \end{reponses}
        \explain{\ce{NH3} est une base car il peut accepter un proton $H^+$ pour former \ce{NH4+}.}
    \end{question}
}

\element{defacidebase}{
    \begin{question}{defacidebase8}
        La réaction \ce{HCl + H2O -> Cl- + H3O+} montre que~:
        \begin{reponses}
            \mauvaise{\ce{H2O} est un acide}
            \mauvaise{\ce{Cl-} est un acide}
            \mauvaise{\ce{HCl} est une base}
            \bonne{\ce{HCl} est un acide}
        \end{reponses}
        \explain{\ce{HCl} donne un proton $H^+$ à \ce{H2O} pour former \ce{H3O+}, ce qui en fait un acide selon Brønsted-Lowry.}
    \end{question}
}
