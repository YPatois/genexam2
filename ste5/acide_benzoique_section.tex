\section {Caractéristique de l'acide benzoïque}
\subsection{Introduction}
L'acide benzoïque, de formule \ce{C6H5COOH}, et le benzoate de sodium sont des conservateurs antimicrobiens 
respectivement identifiés par les codes E210 et E211. Ils sont présents dans de nombreux produits alimentaires 
et notamment dans certaines boissons gazeuses sucrées. À température ambiante, l’acide benzoïque est un solide blanc.

\textbf{Source du sujet :} Baccalauréat de mars 2023 – exercice 4 du sujet de Métropole.

\begin{figure}[h]
    \centering
    \chemfig{OH-[:120,,1](=[:60]O)-[:180]=_[:240]-[:180]=_[:120]-[:60]=_(-[:300])}
    \caption{Formule topologique de l’acide benzoïque}
\end{figure}

\textbf{Données :}
\begin{itemize}
    \item Masse molaire de l'acide benzoïque : $M = \SI{122}{\gram\per\mole}$.
    \item Température de fusion de l'acide benzoïque : $\theta_f = \SI{122,4}{\degreeCelsius}$.
    \item pKa du couple acide benzoïque/ion benzoate : $pKa = 4,2$.
    \item Le benzoate de sodium est soluble dans l’eau.
\end{itemize}

\begin{table}[h]
\centering
\begin{tabular}{|c|c|c|c|c|}
\hline
 & Densité & Eau & Éthanol & Éther éthylique \\
\hline
Acide benzoïque & - & Peu soluble & Très soluble & Très soluble \\
\hline
Eau & 1,0 & - & Miscible & Non miscible \\
\hline
Éthanol & 0,76 & Miscible & - & - \\
\hline
Éther éthylique & 0,71 & Non miscible & - & - \\
\hline
\end{tabular}
\caption{Tableau de solubilité des substances}
\end{table}

L’objectif de cet exercice est de vérifier l’indication d’une étiquette de boisson gazeuse concernant la
présence d’un conservateur.

\subsection{Questions}
