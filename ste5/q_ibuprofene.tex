\section{Stéréochimie et réactions acido-basiques de l'ibuprofène}

\subsection{Introduction}
Au XIXe siècle, on utilisait déjà des principes actifs chiraux comme la morphine, administrée comme 
anti-douleur. Malgré les idées énoncées par Pasteur à la fin du XIXe siècle, les chimistes ont 
mis beaucoup de temps pour comprendre que la chiralité pouvait avoir un impact considérable sur 
les organismes vivants. Cette prise de conscience a eu lieu dans les années 1960 avec le drame de 
la thalidomide, médicament qui fut administré aux femmes enceintes comme anti-vomitif, et qui 
provoqua chez les nouveau-nés de graves malformations. On connaît aujourd'hui la raison de ce drame~: 
alors que l'énantiomère R est bien anti-vomitif, l'énantiomère S est tératogène.

L'ibuprofène est connu pour avoir un effet biologique anti-inflammatoire et antipyrétique sous sa 
forme S et sans effet thérapeutique notable sous sa forme R. Le produit commercial est généralement 
le mélange racémique. Cependant, seul l'énantiomère S est biologiquement actif et présente les effets 
thérapeutiques désirés. L'énantiomère R est très difficile à séparer du S, mais est heureusement 
inoffensif. L'énantiomère S seul commence à produire son effet 12 minutes après son absorption, 
alors que le mélange racémique n'est actif que 38 minutes après avoir été absorbé. Très curieusement, 
le corps humain possède la propriété de pouvoir transformer chimiquement l'énantiomère R inactif en 
énantiomère S.

\textbf{Note :} Quand un mélange contient en proportion égale les deux énantiomères R et S 
d'une molécule ayant un seul carbone asymétrique, le mélange est qualifié de racémique.

\subsection{Données}
\begin{itemize}
    \item Numéros atomiques : $Z(H) = 1$ ; $Z(C) = 6$ ; $Z(O) = 8$.
    \item pKa du couple acide/base de l'ibuprofène : $pKa = 4,5$ à $\SI{20}{\degreeCelsius}$.
\end{itemize}

\begin{figure}[h]
    \centering
    \chemfig{
    OH-[:210,,1](=[:270]O)-[:150](\mcfatomno{*}-[:90]CH3)-[:210]=_[:270]-[:210]=_[:150](-[:90]%
    =_[:30]-[:330])-[:210]-[:150](-[:210])-[:90]
    }
    \caption{Structure de l'ibuprofène}
\end{figure}

\begin{figure}[h]
    \centering
    \chemfig{
    -[:210]=_[:270]-[:210]=_[:150](-[:90]%
    =_[:30]-[:330])-[:210]-[:150](-[:210])-[:90]
    }
    \caption{Le groupe ci-dessus, pourra être noté MPP pour méthyl
phénylpropane, il peut simplifier vos représentations}
\end{figure}

\subsection{Questions}

\begin{question}{ibuprofene1}
    Représenter la formule topologique de l'ibuprofène.
    \begin{EnvQuadrillage}[NbCarreaux=20x4,Grille=Seyes,Marge=1]
    \end{EnvQuadrillage}
    \AMCOpen{}{
        \mauvaise[F]{NA}\scoring{b=0}
        \mauvaise[X]{somewhat correct}\scoring{b=.5}
        \bonne[C]{Formule correcte}\scoring{b=1}
    }
    \explain{
        \chemfig{
            CH_3>[:100](<:[:-30]H)(-[90]COOH)-[:210]=_[:270]-[:210]%
            =_[:150](-[:90]=_[:30]-[:330])-[:210]-[:150](-[:210])-[:90]}
        ou \chemfig{ CH_3>[:100](<:[:-30]H)(-[90]COOH)-[:210]MPP } \\
    }
\end{questionmult}

\begin{questionmult}{ibuprofene2}
    Entourer, sur la formule topologique, le groupe caractéristique présent dans la molécule 
    d'ibuprofène et nommer la famille fonctionnelle associée (répondre sur le schéma que vous avez dessiné au dessus).
    \begin{EnvQuadrillage}[NbCarreaux=20x2,Grille=Seyes,Marge=1]
    \end{EnvQuadrillage}
    \AMCOpen{}{
        \mauvaise[F]{NA}\scoring{b=0}
        \bonne[A1]{Groupe identifié}\scoring{b=.5}
        \bonne[A2]{Famille nommée}\scoring{b=.5}
    }
    \explain{
    Le groupe caractéristique est le groupe carboxyle (–COOH), et la famille fonctionnelle associée 
    est celle des acides carboxyliques.
    }
\end{questionmult}

\begin{questionmult}{ibuprofene3}
    Justifier que l'atome de carbone noté C* dans la représentation de la molécule d'ibuprofène 
    est asymétrique.
    \begin{EnvQuadrillage}[NbCarreaux=20x2,Grille=Seyes,Marge=1]
    \end{EnvQuadrillage}
    \AMCOpen{}{
        \mauvaise[F]{NA}\scoring{b=0}
        \bonne[A1]{Bases}\scoring{b=.5}
        \bonne[A2]{Justification correcte}\scoring{b=.5}
    }
    \explain{
    Le carbone C* est asymétrique car il est lié à quatre groupes d'atomes différents.
    }
\end{questionmult}

\begin{questionmult}{ibuprofene4}
    La représentation d'un des énantiomères A de l'ibuprofène est donnée ci-dessus. Représenter, 
    en perspective de Cram, l'autre énantiomère B de l'ibuprofène.
    
    \centering
    \chemfig{ CH_3>[:100](<:[:-30]H)(-[90]COOH)-[:210]MPP }
    \par\noindent\small\textit{Énantiomère A}
    
    \begin{EnvQuadrillage}[NbCarreaux=20x4,Grille=Seyes,Marge=1]
    \end{EnvQuadrillage}
    \AMCOpen{}{
        \mauvaise[F]{NA}\scoring{b=0}
        \bonne[A1]{Cram perspective}\scoring{b=.5}
        \bonne[A2]{Représentation correcte}\scoring{b=.5}
    }
    \explain{
        \chemfig{ H>[:100](<:[:-30]CH_3)(-[90]COOH)-[:210]MPP }
        L'énantiomère B est l'image miroir de l'énantiomère A en perspective de Cram.
    }
\end{questionmult}

\begin{questionmult}{ibuprofene5}
    Déterminer la configuration absolue, R ou S, de chaque énantiomère A et B en expliquant votre démarche.
    \begin{EnvQuadrillage}[NbCarreaux=20x6,Grille=Seyes,Marge=1]
    \end{EnvQuadrillage}
    \AMCOpen{}{
        \mauvaise[F]{NA}\scoring{b=0}
        \bonne[A1]{Raisonnement}\scoring{b=.5}
        \bonne[A2]{Résultat}\scoring{b=.5}
    }
    \explain{
    La configuration absolue est déterminée en utilisant les règles Cahn-Ingold-Prelog.\\
    mol2chemfig -w -a -60 -z -y delete -i pubchem 114864 : R \\
    \chemfig{OH-[:210,,1](=[:270]O)-[:150](<[:90])-[:210]=_[:270]-[:210]%
    =_[:150](-[:90]=_[:30]-[:330])-[:210]-[:150](-[:210])-[:90]}\\
    mol2chemfig -w -a -60 -z -y delete -i pubchem 6919043 : s \\
    \chemfig{\mcfright{O}{^{\mcfminus}}-[:210](=[:270]O)-[:150](<:[:90])-[:210]%
    =^[:150]-[:210]=^[:270](-[:330]=^[:30]-[:90])-[:210]-[:150](-[:210])-[:90]}
    }
\end{questionmult}

\begin{questionmult}{ibuprofene6}
    Expliquer en quoi un mélange racémique dans le cas de l'ibuprofène :
    \begin{itemize}
        \item n'est pas dangereux pour les humains ;
        \item retarde son action par rapport à l'énantiomère S seul.
    \end{itemize}
    \begin{EnvQuadrillage}[NbCarreaux=20x4,Grille=Seyes,Marge=1]
    \end{EnvQuadrillage}
    \AMCOpen{}{
        \mauvaise[F]{NA}\scoring{b=0}
        \bonne[A1]{Racémique}\scoring{b=.25}
        \bonne[A2]{Dangereux}\scoring{b=.25}
        \bonne[A3]{Retarde}\scoring{b=.25}
        \bonne[A4]{Overal}\scoring{b=.25}
    }
    \explain{
    Le mélange racémique n'est pas dangereux car l'énantiomère R est inactif mais inoffensif. 
    L'action est retardée car le corps doit convertir l'énantiomère R en S.
    }
\end{questionmult}

\begin{questionmult}{ibuprofene7}
    L'ibuprofène est un acide faible. Il appartient à un couple acide-base de pKa égal à 4,5 à $\SI{20}{\degreeCelsius}$.
    \begin{enumerate}
        \item Écrire la formule semi-développée de la base conjuguée de l'ibuprofène en notant 
        MPP le groupe méthyl phénylpropane.
        \item Indiquer si la forme basique de l'ibuprofène possède les mêmes propriétés 
        stéréochimiques que la forme acide.
    \end{enumerate}
    \begin{EnvQuadrillage}[NbCarreaux=20x8,Grille=Seyes,Marge=1]
    \end{EnvQuadrillage}
    \AMCOpen{}{
        \mauvaise[F]{NA}\scoring{b=0}
        \mauvaise[X]{Faux}\scoring{b=-1}
        \bonne[C]{Formule correcte}\scoring{b=1}
        \bonne[C]{Propriétés stéréochimiques}\scoring{b=1}
    }
    \explain{
    La base conjuguée de l'ibuprofène conserve le carbone asymétrique, donc elle possède les 
    mêmes propriétés stéréochimiques.
    }
\end{questionmult}
