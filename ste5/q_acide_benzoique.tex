\begin{questionmult}{acidbenzo1}
    Entourer et nommer le groupe caractéristique présents dans la formule topologique de l'acide benzoïque ci-dessous.\\
        \centering\chemfig{OH-[:120,,1](=[:60]O)-[:180]=_[:240]-[:180]=_[:120]-[:60]=_(-[:300])}\\
    \AMCOpen{}{
        \mauvaise[F]{NA}\scoring{b=0}
        \mauvaise[X]{Faux}\scoring{b=-1}
        \bonne[I]{Groupe identifié}\scoring{b=.5}
        \bonne[N]{Groupe nommé}\scoring{b=.5}
    }
    \explain{Le groupes caractéristique est le groupe carboxyle (–COOH)}
\end{questionmult}

\begin{question}{acidbenzo2}
    Écrire la formule topologique de l’ion benzoate, base conjuguée de l’acide benzoïque.\\
    \begin{EnvQuadrillage}[NbCarreaux=20x4,Grille=Seyes,Marge=1]
    \end{EnvQuadrillage}
    \AMCOpen{}{
        \mauvaise[F]{NA}\scoring{b=0}
        \mauvaise[X]{Faux}\scoring{b=-1}
        \mauvaise[B]{\approx}\scoring{b=.5}
        \bonne[A]{OK}\scoring{b=1}
    }
    \explain{
    La formule topologique de l’ion benzoate est :
    mcfright{O}{^{\mcfminus}}-[:210](=[:150]O)-[:270]=^[:210]-[:270]=^[:330]%
-[:30]=^[:90](-[:150])}
\end{question}

\begin{question}{acidbenzo3}
    Représenter le diagramme de prédominance du couple acide benzoïque/ion benzoate.\\
    \begin{EnvQuadrillage}[NbCarreaux=20x4,Grille=Seyes,Marge=1]
    \end{EnvQuadrillage}
    \AMCOpen{}{
        \mauvaise[F]{NA}\scoring{b=0}
        \mauvaise[X]{Faux}\scoring{b=-1}
        \mauvaise[B]{\approx}\scoring{b=1}
        \bonne[A]{OK}\scoring{b=2}
    }
    \explain{
    Le diagramme de prédominance du couple acide benzoïque/ion benzoate doit montrer que :
    - À un pH inférieur au pKa (4,2), l'acide benzoïque (\ce{C6H5COOH}) est prédominant.
    - À un pH supérieur au pKa (4,2), l'ion benzoate (\ce{C6H5COO-}) est prédominant.
    }
\end{question}

\subsection{Expériences}

Dans un premier temps, on réalise deux expériences permettant de mettre en évidence les propriétés de l’acide
benzoïque et de l’ion benzoate :
\begin{itemize}
\item Expérience (1) : Dans un tube à essais contenant une solution de benzoate de sodium, on ajoute quelques
gouttes d’acide chlorhydrique concentré. On observe qu’un solide blanc apparaît.
\item Expérience (2) : Dans un tube à essais contenant une solution de soude concentrée, on ajoute de l’acide
benzoïque solide. On observe que le solide introduit disparaît.
\end{itemize}

\begin{questionmult}{acidbenzo4}
    On donne l'équation chimique suivante :\\
    \ce{C6H5COO^-_{(aq)} + H3O^+_{(aq)} -> C6H5COOH_{(aq)} + H2O_{(l)}}\\
    La réaction qui correspond à l'équation ci-dessus modélise une transformation chimique ayant lieu lors de l'une 
    des expériences précédentes. Indiquer, en justifiant la réponse, s'il s'agit de l'expérience (1) ou de l'expérience (2).
    \begin{EnvQuadrillage}[NbCarreaux=20x6,Grille=Seyes,Marge=1]
    \end{EnvQuadrillage}
    \AMCOpen{}{
        \mauvaise[F]{NA}\scoring{b=0}
        \mauvaise[X]{Faux}\scoring{b=-1}
        \bonne[P]{Expérience identifiée}\scoring{b=0.5}
        \bonne[C]{Justification correcte}\scoring{b=0.5}
    }
    \explain{
    Il s'agit de l'expérience (1). Les réactifs sont l'ion benzoate et l'ion oxonium, qui correspondent aux réactifs de 
    l'expérience 1, les ions oxonium étant apportés par la solution d'acide chlorhydrique.
    }
\end{questionmult}

L’étiquette d’une boisson gazeuse indique qu’elle contient du benzoate de sodium comme conservateur alimentaire,
entre autres, et on souhaite vérifier cette indication. On réalise pour cela une extraction liquide-liquide.
On suit le protocole expérimental suivant :

\begin{enumerate}
    \item Verser 500 mL de boisson dans un grand bécher.
    \item Ajouter de l’acide chlorhydrique jusqu’à amener le pH à environ 2.
    \item Ajouter alors 40 mL d’éther éthylique, agiter et laisser reposer.
    \item Transvaser l’ensemble dans une ampoule à décanter.
    \item Agiter vigoureusement pendant deux minutes en prenant soin de dégazer régulièrement pour éviter toute surpression.
    \item Laisser décanter.
    \item Récupérer la phase aqueuse S dans un bécher et la phase organique dans un erlenmeyer.
\end{enumerate}


\begin{questionmult}{acidbenzo5}
    Préciser ce qu'il se passe lors de l'étape 2 du protocole.
    \begin{EnvQuadrillage}[NbCarreaux=20x5,Grille=Seyes,Marge=1]
    \end{EnvQuadrillage}
    \AMCOpen{}{
        \mauvaise[F]{NA}\scoring{b=0}
        \mauvaise[X]{Faux}\scoring{b=-1}
        \bonne[A]{Eq}\scoring{b=.5}
        \bonne[B]{Prd}\scoring{b=.5}
    }
    \explain{
    Lors de l'étape 2 du protocole, l'ion benzoate se transforme en acide benzoïque suite à l'ajout d'acide chlorhydrique. 
    En effet, on ajoute l'acide chlorhydrique jusqu'à pH = 2, c'est-à-dire un pH inférieur au pKa du couple. 
    Il s'agit donc de la zone de prédominance de l'acide benzoïque.
    }
\end{questionmult}

\begin{question}{acidbenzo6}
    À l'aide des données, expliquer pourquoi l'éther éthylique constitue un solvant extracteur plus adapté que 
    l'éthanol lors de la réalisation de l'étape 3 du protocole.
    \begin{EnvQuadrillage}[NbCarreaux=20x6,Grille=Seyes,Marge=1]
    \end{EnvQuadrillage}
    \AMCOpen{}{
        \mauvaise[F]{NA}\scoring{b=0}
        \mauvaise[X]{Faux}\scoring{b=-1}
        \mauvaise[C]{\approx}\scoring{b=.5}
        \bonne[C]{OK}\scoring{b=1}
    }
    \explain{
    L'éther éthylique est non miscible à l'eau et l'acide benzoïque y est très soluble, contrairement à l'éthanol qui 
    est miscible à l'eau. Cela en fait un solvant extracteur plus adapté.
    }
\end{question}

\begin{questionmult}{acidbenzo7}
    Compléter le schéma ci-dessous, en justifiant, à l'aide des données, la nature 
    et la position des différentes phases dans l'ampoule à décanter à l'issue de l'étape 6 du protocole.

    \begin{center}
        \begin{minipage}[t]{0.4\textwidth}
            \centering
            \includegraphics[height=5cm, keepaspectratio]{images/ste5_ampoule.png}
        \end{minipage}%
        \begin{minipage}[t]{0.6\textwidth}
            \begin{EnvQuadrillage}[NbCarreaux=12x6, Grille=Seyes,Marge=1]
            \end{EnvQuadrillage}
        \end{minipage}
    \end{center}

    \AMCOpen{}{
        \mauvaise[F]{NA}\scoring{b=0}
        \mauvaise[X]{Faux}\scoring{b=-1}
        \bonne[C]{ScBuoy}\scoring{b=.5}
        \bonne[C]{ScAcd}\scoring{b=.5}
        \bonne[C]{XpBuoy}\scoring{b=.5}
        \bonne[C]{Ovl}\scoring{b=.5}
   }
    \explain{
    L'éther éthylique (densité = 0,71) est moins dense que l'eau (densité = 1,0). La phase organique (éther) 
    constitue donc la phase supérieure et la phase aqueuse, la phase inférieure.
    }
\end{questionmult}

\begin{question}{acidbenzo8}
    Après évaporation du solvant extracteur, il reste une masse \( m = 10,0 \) mg de solide blanc. Proposer 
    une méthode expérimentale permettant de vérifier que ce solide blanc est bien de l'acide benzoïque.
    \begin{EnvQuadrillage}[NbCarreaux=20x4,Grille=Seyes,Marge=1]
    \end{EnvQuadrillage}
    \AMCOpen{}{
        \mauvaise[F]{NA}\scoring{b=0}
        \mauvaise[X]{Faux}\scoring{b=-1}
        \bonne[B]{\approx}\scoring{b=.5}
        \bonne[A]{OK}\scoring{b=1}
   }
    \explain{
    On peut mesurer son point de fusion au banc Kofler et vérifier qu'il est bien de \(122,4^\circ C\). Une 
    chromatographie sur couche mince ou un spectre infra-rouge/RMN peuvent aussi être réalisés pour confirmation.
    }
\end{question}

\begin{questionmult}{acidbenzo9}
    Sachant que la concentration $C$ en quantité de matière totale théorique d'ions benzoate dans cette 
    boisson est égale à \qty{4.0e-4}{\mol\per\litre}, montrer que la masse théorique d'acide 
    benzoïque que l'on devrait obtenir à l'issue de l'extraction est égale à \qty{24}{\mg}.

    \begin{EnvQuadrillage}[NbCarreaux=20x8,Grille=Seyes,Marge=1]
    \end{EnvQuadrillage}

    \AMCOpen{}{
        \mauvaise[F]{NA}\scoring{b=0}
        \mauvaise[X]{Faux}\scoring{b=-1}
        \bonne[C]{Idée}\scoring{b=.5}
        \bonne[B]{Forml}\scoring{b=.5}
        \bonne[A]{Nb}\scoring{b=.5}
        \bonne[A]{Ovl}\scoring{b=.5}
    }

    \explain{
    La masse théorique d'acide benzoïque est calculée comme suit :
    \[
    m = C \times V \times M = \SI{4.0e-4}{\mol\per\litre} \times \SI{0.5}{\litre} \times \SI{122}{\gram\per\mol} = \SI{24}{\milligram}
    \]
    }
\end{questionmult}

\begin{questionmult}{acidbenzo10}
    Le rendement d'une extraction étant défini comme le rapport de la masse de substance réellement extraite sur la 
    masse théorique de substance que l'on aurait pu extraire, calculer le rendement de cette extraction et 
    l'exprimer en pourcentage. Commenter ce résultat.
    \begin{EnvQuadrillage}[NbCarreaux=20x4,Grille=Seyes,Marge=1]
    \end{EnvQuadrillage}
    \AMCOpen{}{
        \mauvaise[F]{NA}\scoring{b=0}
        \mauvaise[X]{Faux}\scoring{b=-1}
        \bonne[C]{Idée}\scoring{b=.5}
        \bonne[B]{Forml}\scoring{b=.5}
        \bonne[A]{Nb}\scoring{b=.5}
        \bonne[A]{Ovl}\scoring{b=.5}
    }
    \explain{
    Le rendement est calculé comme suit :
    \[
    \eta = \frac{10,0}{24} \times 100 = 41,7\%
    \]
    Ce rendement est relativement faible. Pour l'améliorer, il faudrait extraire une deuxième fois la phase aqueuse avec
     40 mL d'éther éthylique puis rassembler les phases organiques.
    }
\end{questionmult}

