% symboles chimiques
\element{symchim}{
  \begin{question}{symchim1}
    Quel est le symbole chimique du carbone~?
    \begin{reponseshoriz}
      \mauvaise{\ce{H}}
      \mauvaise{\ce{O}}
      \mauvaise{\ce{N}}
      \bonne{\ce{C}}
    \end{reponseshoriz}
    \explain{Le symbole chimique du carbone est \ce{C}.}
  \end{question}
}

\element{symchim}{
  \begin{question}{symchim2}
    Quel est le symbole chimique de l'oxygène~?
    \begin{reponseshoriz}
      \mauvaise{\ce{C}}
      \mauvaise{\ce{H}}
      \mauvaise{\ce{N}}
      \bonne{\ce{O}}
    \end{reponseshoriz}
    \explain{Le symbole chimique de l'oxygène est \ce{O}.}
  \end{question}
}

\element{symchim}{
  \begin{question}{symchim3}
    Quel est le symbole chimique de l'azote~?
    \begin{reponseshoriz}
      \mauvaise{\ce{C}}
      \mauvaise{\ce{H}}
      \mauvaise{\ce{O}}
      \bonne{\ce{N}}
    \end{reponseshoriz}
    \explain{Le symbole chimique de l'azote est \ce{N}.}
  \end{question}
}

\element{symchim}{
  \begin{question}{symchim4}
    Quel est le symbole chimique de l'hydrogène~?
    \begin{reponseshoriz}
      \mauvaise{\ce{C}}
      \mauvaise{\ce{O}}
      \mauvaise{\ce{N}}
      \bonne{\ce{H}}
    \end{reponseshoriz}
    \explain{Le symbole chimique de l'hydrogène est \ce{H}.}
  \end{question}
}

% Notation des symboles chimiques
\element{symchim}{
  \begin{question}{symchim5}
    Dans le symbole \ce{^A_ZX}, qu'est-ce que représente Z~?
    \begin{reponses}
      \mauvaise{Le nombre de neutrons}
      \mauvaise{Le nombre de nucléons}
      \mauvaise{Le numéro de masse}
      \bonne{Le nombre de protons}
    \end{reponses}
    \explain{Dans le symbole \ce{^A_ZX}, Z représente le nombre de protons, qui correspond au numéro atomique de l'élément.}
  \end{question}
}

\element{symchim}{
  \begin{question}{symchim6}
    Dans le symbole \ce{^A_ZX}, qu'est-ce que représente A~?
    \begin{reponses}
      \mauvaise{Le nombre de protons}
      \mauvaise{Le nombre de neutrons}
      \mauvaise{La charge électrique de l'atome}
      \bonne{Le nombre de nucléons}
    \end{reponses}
    \explain{Dans le symbole \ce{^A_ZX}, A représente le nombre total de nucléons, c'est-à-dire la somme des protons et des neutrons.}
  \end{question}
}

\element{symchim}{
  \begin{question}{symchim7}
    Dans le symbole \ce{^A_ZX}, comment peut-on déterminer le nombre de neutrons~?
    \begin{reponses}
      \mauvaise{En additionnant Z et A}
      \mauvaise{En soustrayant A de Z (Z-A)}
      \mauvaise{En multipliant Z par A}
      \bonne{En soustrayant Z de A (A-Z)}
    \end{reponses}
    \explain{Le nombre de neutrons est obtenu en soustrayant le nombre de protons (Z) du nombre total de nucléons (A).}
  \end{question}
}

\element{symchim}{
  \begin{question}{symchim8}
    Dans le symbole \ce{^14_6C}, combien y a-t-il de neutrons~?
    \begin{reponseshoriz}
      \mauvaise{6}
      \mauvaise{10}
      \mauvaise{12}
      \bonne{8}
    \end{reponseshoriz}
    \explain{Le nombre de protons Z est 6, et le nombre de nucléons A est 14. Donc, le nombre de neutrons est 14 - 6 = 8.}
  \end{question}
}
